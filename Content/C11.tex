\newpage

\section[Day 11: Limits \& Special Sequences]{Limits and Special Sequences}

\subsection{ Upper and Lower Limits } 

{ \color{blue} Definition 11.1.1: No limit } 

	\begin{adjustbox}{minipage=14cm, right, vspace=0.1cm 0cm}
		Let \{$s_n$\} be a sequence of real numbers such that:

		\hspace{1cm}
		For every real M, there is a N $\in$ $\mathbb{Z}$ such that
		for n $\geq$ N, $s_n$ $\geq$ M.

		\hspace{2cm}
		Then, $s_n$ $\rightarrow$ $+\infty$.

		\hspace{1cm}
		For every real M, there is a N $\in$ $\mathbb{Z}$ such that
		for n $\geq$ N, $s_n$ $\leq$ M.

		\hspace{2cm}
		Then, $s_n$ $\rightarrow$ $-\infty$. \\
	\end{adjustbox}

{ \color{blue} Definition 11.1.2: Upper and Lower Limits } 

	\begin{adjustbox}{minipage=14cm, right, vspace=0.1cm 0cm}
		Let \{$s_n$\} be a sequence of real numbers.

		Let E contain all subsequential limits of \{$s_n$\}
		plus possibly $\pm$$\infty$.

		Then, the upper limit of \{$s_n$\}:

		\hspace{1cm}
		$s^*$ = sup(E) \hspace{1cm} $\lim_{n \rightarrow \infty}$ sup($s_n$) = $s^*$

		Then, the lower limit of \{$s_n$\}:

		\hspace{1cm}
		$s_*$ = inf(E) \hspace{1.2cm} $\lim_{n \rightarrow \infty}$ inf($s_n$) = $s_*$ \\
	\end{adjustbox}

{ \color{red} Theorem 11.1.3: Upper and Lower limits are unique }

	\begin{adjustbox}{minipage=14cm, right, vspace=0.1cm 0.1cm}
		Let \{$s_n$\} be a sequence of real numbers.
		Let E be the set of subsequential limits and
		$s^*$ be the upper limit of \{$s_n$\}. Then:
	\end{adjustbox}

	\begin{enumerate}[label=(\alph*), leftmargin=2cm, itemsep=0.1cm]
		\item $s^*$ $\in$ E
		
			{ \color{magenta} \underline{Proof} }

			\fbox{
			\begin{minipage}{13cm}
				If $s^*$ = $+\infty$, then there is a
				\{$s_{n_k}$\} $\rightarrow$ $+\infty$ so E is not bounded above.
				
				If $s^*$ $\in$ $\mathbb{R}$, then E is bounded above so
				$s^*$ $\in$ E'.

				Then by {\color{red} theorem 10.2.4}, $s^*$ $\in$ E.

				If $s^*$ = $-\infty$, then there are no subsequential limits
				in E.
				Thus, for every M, there is a N such that for n $\geq$ N,
				$s_n$ $\leq$ M so $-\infty$ $\in$ E.
			\end{minipage} }

		\item If x $>$ $s^*$, there is a N such that for n $\geq$ N, $s_n$ $<$ x
		
			{ \color{magenta} \underline{Proof} }

			\fbox{
			\begin{minipage}{13cm}
				Suppose there is a x $>$ $s^*$ such that $s_n$ $\geq$ x
				for infinitely many n.

				Then, there is a y $\in$ E where y $\geq$ x $>$ $s^*$
				which contradicts $s^*$ = sup(E).
			\end{minipage} }

		\item $s^*$ is the only number that satisfies (a) and (b)
		
			{ \color{magenta} \underline{Proof} }

			\fbox{
			\begin{minipage}{13cm}
				Suppose p,q satisfy part a and b where p $<$ q.
				
				Choose x where p $<$ x $<$ q.
				Since p satisfies b, then $s_n$ $<$ x for n $\geq$ N.
				
				Thus, q cannot satisfy a.
			\end{minipage} }
	\end{enumerate}

	\vspace{0.2cm}

	\hspace{1cm}
	The same properties are analogous for $s_*$. \\

\newpage

{ \color{red} Theorem 11.1.4: Inf \& Sup of $s_n$ $\leq$ $t_n$ }

	\begin{adjustbox}{minipage=14cm, right, vspace=0.1cm 0cm}
		If $s_t$ $\leq$ $t_n$ for n $\geq$ fixed N, then

		\hspace{1cm}
		$\lim_{n \rightarrow \infty}$ inf($s_n$)
		$\leq$
		$\lim_{n \rightarrow \infty}$ inf($t_n$)

		\hspace{1cm}
		$\lim_{n \rightarrow \infty}$ sup($s_n$)
		$\leq$
		$\lim_{n \rightarrow \infty}$ sup($t_n$)
	\end{adjustbox}

{ \color{magenta} \underline{Proof} }

	\fbox{
	\begin{minipage}{15cm}
		Let $E_1$ be the set of extended reals x such that
		\{$s_{n_k}$\} $\rightarrow$ x for some \{$s_{n_K}$\}.

		Let $E_2$ be the set of extended reals y such that
		\{$t_{n_k}$\} $\rightarrow$ y for some \{$s_{n_K}$\}.

		Let $s^*$ = sup($E_1$), $s_*$ = inf($E_1$),
		$t^*$ = sup($E_2$), and $t_*$ = inf($E_2$).

		Since there is a N such that $s_n$ $\leq$ $t_n$ for n $\geq$ N,
		then:
		
		\hspace{1cm}
		x $\leftarrow$ \{$s_{N}, s_{N+1}, ... $\}
		$\leq$ \{$t_{N}, t_{N+1}, ... $\} $\rightarrow$ y

		Thus, for n $\geq$ N, inf($s_n$) $\leq$ inf($t_n$)
		and sup($s_n$) $\leq$ sup($t_n$).		
	\end{minipage} }	





\subsection{ Special Sequences }

{ \color{red} Theorem 11.2.1: Special Sequences }

	\begin{enumerate}[label=(\alph*), leftmargin=2cm, itemsep=0.1cm]
		\item If p $>$ 0, then $\lim_{n \rightarrow \infty}$ $\frac{1}{n^p}$ = 0
		
			{ \color{magenta} \underline{Proof} }

				\fbox{
				\begin{minipage}{13cm}
					For $\epsilon$ $>$ 0, let N $>$ $\sqrt[p]{\frac{1}{\epsilon}}$.

					Then for n $\geq$ N,
					$\lim_{n \rightarrow \infty}$ $\frac{1}{n^p}$
					$\leq$ $\frac{1}{N^p}$
					$<$ $\frac{1}{\sqrt[p]{\frac{1}{\epsilon}}^p}$ = $\epsilon$
				\end{minipage} }

		\item If p $>$ 0, then $\lim_{n \rightarrow \infty}$ $\sqrt[n]{p}$ = 1
		
			{ \color{magenta} \underline{Proof} }

				\fbox{
				\begin{minipage}{13cm}
					If p $>$ 1, then let $x_n$ = $\sqrt[n]{p}$ - 1 $>$ 0.

					\hspace{1cm}
					p = $(x_n + 1)^n$
					= $x_n^n + nx_n^{n-1} + ... + nx_n + 1$
					$\geq$ $nx_n + 1$

					Thus, 0 $<$ $x_n$ $\leq$ $\frac{p-1}{n}$ so
					\{$x_n$\} $\rightarrow$ 0 and thus,
					\{$\sqrt[n]{p}$\} $\rightarrow$ 1.

					If p = 1, then $\lim_{n \rightarrow \infty}$ $\sqrt[n]{p}$
					= $\lim_{n \rightarrow \infty}$ 1 = 1.

					If 0 $<$ p $<$ 1, then $\frac{1}{p}$ $>$ 1.
					From the proof above for p $>$ 1,
					\{$\sqrt[n]{\frac{1}{p}}$\} $\rightarrow$ 1.

					Thus, \{$\frac{1}{\sqrt[n]{p}}$\} $\rightarrow$ 1
					so \{$\sqrt[n]{p}$\} $\rightarrow$ 1.
				\end{minipage} }

		\item $\lim_{n \rightarrow \infty}$ $\sqrt[n]{n}$ = 1

			{ \color{magenta} \underline{Proof} }

				\fbox{
				\begin{minipage}{13cm}
					Let $x_n$ = $\sqrt[n]{n}$ - 1 $\geq$ 0.

					\hspace{1cm}
					n = ($x_n$ + 1)$^n$ $\geq$ $\frac{n(n-1)}{2} x_n^2$

					Thus, 0 $\leq$ $x_n$ $\leq$ $\sqrt{\frac{2}{n-1}}$
					so \{$x_n$\} $\rightarrow$ 0 and thus,
					\{$\sqrt[n]{n}$\} $\rightarrow$ 1.
				\end{minipage} }

		\item If p $>$ 0 and $\alpha$ $\in$ $\mathbb{R}$, then
		$\lim_{n \rightarrow \infty}$ $\frac{n^{\alpha}}{(1+p)^n}$ = 0

			{ \color{magenta} \underline{Proof} }

				\fbox{
				\begin{minipage}{13cm}
					Let k $\in$ $\mathbb{Z}$ such that k $>$ $\alpha$ and k $>$ 0.
					For n $>$ 2k:

					\hspace{1cm}
					(1+p)$^n$ $>$ $\binom{n}{k} p^k$
					= $\frac{n(n-1)...(n-k+1)}{k!} p^k$
					$>$ $\frac{n^k p^k}{2^k k!}$

					Thus, 0 $<$ $\frac{n^{\alpha}}{(1+p)^n}$
					$<$ $\frac{2^k k!}{p^k} n^{\alpha - k}$.
					
					Since $\alpha$ - k $<$ 0, then \{$n^{\alpha - k}$\} $\rightarrow$ 0
					so \{$\frac{n^{\alpha}}{(1+p)^n}$\} $\rightarrow$ 0.
				\end{minipage} }

		\item If $|x|$ $<$ 1, then $\lim_{n \rightarrow \infty}$ $x^n$ = 0

			{ \color{magenta} \underline{Proof} }

				\fbox{
				\begin{minipage}{13cm}
					From part d, let $\alpha$ = 0.

					Thus, $\lim_{n \rightarrow \infty}$ $\frac{1}{(1+p)^n}$ = 0
					and since p $>$ 0, then $\frac{1}{(1+p)^n}$ = $(\frac{1}{1+p})^n$ $<$ 1.

					Also, $-\lim_{n \rightarrow \infty}$ $\frac{1}{(1+p)^n}$
					= $\lim_{n \rightarrow \infty}$ $\frac{-1}{(1+p)^n}$ = 0
					so $\frac{-1}{(1+p)^n}$ = $(\frac{-1}{1+p})^n$ $>$ -1.
				\end{minipage} }
	\end{enumerate}




