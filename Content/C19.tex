\newpage

\section[Day 19: $L^p$ Spaces]{ $L^p$ Spaces }

\subsection{ $L^2$: Square Integrable Functions }

    \begin{definition}{Square Integrable}{14cm}
        Measurable function f: [a,b] $\rightarrow$ [$-\infty,\infty$]
        is {\color{lblue} square integrable} if $f^2$ is integrable.

        Let $L^2[a,b]$ be the set of all square integrable functions on [a,b].
        
        Then define the norm of f $\in$ $L^2[a,b]$,
        $||f||$ = $(\int f^2 d\mu)^{\frac{1}{2}}$.
    \end{definition}

    \vspace{0.5cm}



    \begin{wtheorem}{$L^p$ norm: Scalar Multiplication Property}{14cm}
        For any c $\in$ $\mathbb{R}$ and f $\in$ $L^2[a,b]$:

        \hspace{0.5cm}
        $||cf||$ = $|c|$ $||f||$

        Also, $||f||$ $\geq$ 0 where $||f||$ = 0 only if f = 0 almost everywhere.
    \end{wtheorem}

    \begin{proof}
        $||cf||$
        = $(\int c^2f^2 d\mu)^{\frac{1}{2}}$
        = $|c| (\int f^2 d\mu)^{\frac{1}{2}}$
        = $|c| ||f||$

        Since $\int f^2 d\mu$ $\geq$ 0, then $||f||$ $\geq$ 0.
        If $||f||$ = 0, then $\int f^2 d\mu$ = 0
        so by {\color{orange} corollary 18.2.4}, $f^2$ = 0 almost
        everywhere so f = 0 almost everywhere.
    \end{proof}

    \vspace{0.5cm}



    \begin{wtheorem}{If f,g $\in$ $L^2[a,b]$, then fg is integrable}{14cm}
        If f,g $\in$ $L^2[a,b]$, then fg is integrable where:

        \hspace{0.5cm}
        2 $\int$ $|fg|$ $d\mu$ $\leq$ $||f||^2 + ||g||^2$

        Also, 2 $\int$ $|fg|$ $d\mu$ = $||f||^2 + ||g||^2$
        if and only if $|f|$ = $|g|$ almost everywhere
    \end{wtheorem}

    \begin{proof}
        0 $\leq$ $(|f| - |g|)^2$
        = $f^2 - 2|fg| + g^2$
        \hspace{0.5cm}
        $\Rightarrow$
        \hspace{0.5cm}
        $2|fg|$ $\leq$ $f^2 + g^2$

        By {\color{red} theorem 18.2.3}, $|fg|$ is integrable so fg is integrable
        where:
        
        \hspace{0.5cm}
        $\int 2|fg| d\mu$
        $\leq$ $\int f^2 + g^2 d\mu$
        = $||f||^2 + ||g||^2$

        Since equality holds if and only if $\int (|f| - |g|)^2 d\mu$ = 0,
        then by {\color{orange} corollary 18.2.4}, $(|f| - |g|)^2$ = 0
        almost everywhere so $|f|$ = $|g|$ almost everywhere.
    \end{proof}

    \vspace{0.5cm}



    \begin{wtheorem}{$L^2[a,b]$ is a vector space}{14cm}
        $L^2[a,b]$ is a vector space
    \end{wtheorem}

    \begin{proof}
        If f,g $\in$ $L^2[a,b]$, then $f^2,g^2$ is integrable.
        Since $(c_1f+c_2g)^2$ = $c_1^2f^2 + 2c_1c_2fg + c_2^2g^2$
        where $c_1^2f^2,c_2^2g^2$ are integrable and $2c_1c_2fg$
        is integrable by {\color{red} theorem 19.1.3}, then
        $(c_1f+c_2g)^2$ is integrable and thus,
        $c_1f+c_2g$ $\in$ $L^2[a,b]$.
    \end{proof}

    \vspace{0.5cm}



    \begin{wtheorem}{Holder's Inequality in $L^2$}{14cm}
        If f,g $\in$ $L^2[a,b]$, then:

        \hspace{0.5cm}
        $\int$ $|fg|$ $d\mu$ $\leq$ $||f||$ $||g||$

        Equality if and only if
        $|f|$ = $c|g|$ almost everywhere for some c $\in$ $\mathbb{R}$
    \end{wtheorem}

    \begin{proof}
        If either $||f||,||g||$ = 0, then the inequality holds true.
        Let $f_0$ = $\frac{f}{||f||}$ and $g_0$ = $\frac{g}{||g||}$.

        Then by {\color{red} theorem 19.1.3}:

        \hspace{0.5cm}
        $2 \int |f_0g_0| d\mu$
        $\leq$ $||f_0||^2 + ||g_0||^2$
        = $||\frac{f}{||f||}||^2 + ||\frac{g}{||g||}||^2$
        = $\frac{||f||}{||f||}^2 + \frac{||g||}{||g||}^2$
        = 2

        \hspace{0.5cm}
        $\int |f_0g_0| d\mu$ $\leq$ 1
        \hspace{0.5cm}
        $\Rightarrow$
        \hspace{0.5cm}
        $\int |fg| d\mu$ $\leq$ $||f||$ $||g||$

        where $\int |f_0g_0| d\mu$ = 1 if and only if
        $\frac{1}{||f||}|f|$ = $|f_0|$ = $|g_0|$ = $\frac{1}{||g||}|g|$
        almost everywhere.
    \end{proof}

    \newpage



    \begin{corollary}{Cauchy-Schwarz Inequality in $L^2$}{14cm}
        If f,g $\in$ $L^2[a,b]$, then:
        
        \hspace{0.5cm}
        $|$ $\int$ fg $d\mu$ $|$ $\leq$ $||f||$ $||g||$

        Equality if and only if
        f = cg almost everywhere for some c $\in$ $\mathbb{R}$ 
    \end{corollary}

    \begin{proof}
        $|$ $\int$ fg $d\mu$ $|$
        $\leq$ $\int$ $|fg|$ $d\mu$
        $\leq$ $||f||$ $||g||$

        Suppose $|$ $\int$ fg $d\mu$ $|$ = $||f||$ $||g||$
        so $\int$ $|fg|$ $d\mu$ = $||f||$ $||g||$.

        If $\int fg d\mu$ $\geq$ 0, then $\int |fg| d\mu$ = $\int fg d\mu$
        so $|fg|$ = fg almost everywhere.
        Since $|f|$ = $c|g|$ almost everywhere, then
        f = cg almost everywhere.

        If $\int fg d\mu$ $\leq$ 0, then $\int |-fg| d\mu$ = $\int -fg d\mu$
        so $|fg|$ = -fg almost everywhere.
        Since $|f|$ = $c|g|$ almost everywhere, then
        f = -cg almost everywhere.
    \end{proof}

    \vspace{0.5cm}



    \begin{wtheorem}{Minkowski's Inequality in $L^2$}{14cm}
        If f,g $\in$ $L^2[a,b]$, then:

        \hspace{0.5cm}
        $||f+g||$ $\leq$ $||f||$ + $||g||$
    \end{wtheorem}

    \begin{proof}
        $||f+g||^2$
        = $\int$ $(f+g)^2$ $d\mu$
        = $\int$ $f^2 + 2fg + g^2$ $d\mu$
        $\leq$ $\int$ $f^2 + 2|fg| + g^2$ $d\mu$

        \hspace{1.6cm}
        $\leq$ $||f||^2$ + $2||f||$ $||g||$ + $||g||^2$
        = ($||f||$ + $||g||$)$^2$
        
        Thus, $||f+g||$ $\leq$ $||f||$ + $||g||$.
    \end{proof}

    \vspace{0.5cm}



    \begin{definition}{Inner Product on $L^2$}{14cm}
        If f,g $\in$ $L^2[a,b]$, then the {\color{red} inner product}
        of f and g:

        \hspace{0.5cm}
        $\langle f,g \rangle$ = $\int$ fg $d\mu$
    \end{definition}

    \vspace{0.5cm}



    \begin{wtheorem}{Properties of the Inner Product on $L^2$}{14cm}
        For $f_1,f_2,g$ $\in$ $L^2[a,b]$ and $c_1,c_2$ $\in$ $\mathbb{R}$:
    \end{wtheorem}

    \begin{enumerate}[label=(\alph*), leftmargin=2cm, itemsep=0.1cm]
        \item {\color{lblue} Commutativity}:
            $\langle f_1,f_2 \rangle$ = $\langle f_2,f_1 \rangle$

        \item {\color{lblue} Bilinearity}:
            $\langle c_1f_1 + c_2f_2 , g \rangle$
            = $c_1 \langle f_1,g \rangle + c_2 \langle f_2,g \rangle$

        \item {\color{lblue} Positive Definiteness}:
            $\langle f_1,f_1 \rangle$ = $||f_1||^2$ $\geq$ 0

            $\langle f_1,f_1 \rangle$ = 0 if and only if $f_1$ = 0 almost everywhere
    \end{enumerate}

    \begin{proof}
        $\langle f_1,f_2 \rangle$
        = $\int f_1f_2 d\mu$
        = $\int f_2f_1 d\mu$
        = $\langle f_2,f_1 \rangle$

        \rule[0.1cm]{15.2cm}{0.01cm}

        $\langle c_1f_1 + c_2f_2 , g \rangle$
        = $\int (c_1f_1 + c_2f_2)g d\mu$
        = $c_1\int f_1g d\mu + c_2\int f_2g d\mu$
        = $c_1 \langle f_1,g \rangle + c_2 \langle f_2,g \rangle$

        \rule[0.1cm]{15.2cm}{0.01cm}

        $\langle f_1,f_1 \rangle$
        = $\int f_1^2 d\mu$
        = $||f_1||^2$ $\geq$ 0
        where $||f_1||^2$ = $\langle f_1,f_1 \rangle$ = 0
        if and only if $f_1$ = 0 almost everywhere
        by {\color{red} theorem 19.1.2}
    \end{proof}

    \newpage





\subsection{ Convergence in $L^2$ }

    \begin{definition}{Convergence in $L^2$}{14cm}
        \{$f_n$\} $\in$ $L^2[a,b]$ converges to f $\in$ $L^2[a,b]$ if:

        \hspace{0.5cm}
        $\lim_{n \rightarrow \infty}$ $||f - f_n||$ = 0
    \end{definition}

    \vspace{0.5cm}



    \begin{wtheorem}{Approximating f $\in$ $L^2[a,b]$ with bounded $f_n$}{14cm}
        For f $\in$ $L^2[a,b]$, let:

        \hspace{0.5cm}
        $f_n(x)$ =
        $\begin{cases}
            -n & f(x) < -n \\
            f(x) & f(x) \in [-n,n] \\
            n & f(x) > n
        \end{cases}$

        Then, $\lim_{n \rightarrow \infty}$ $||f - f_n||$ = 0.
    \end{wtheorem}

    \begin{proof}
        Since $|f_n|$ $\leq$ $|f|$, then:

        \hspace{0.5cm}
        $|f - f_n|^2$
        $\leq$ $|f|^2 + 2|f||f_n| + |f_n|^2$
        $\leq$ $4|f|^2$

        Let set $E_n$ = \{ x $|$ $|f(x)|$ $>$ n \}
        = \{ x $|$ $|f(x)|^2$ $>$ $n^2$ \} and let C = $\int |f|^2 d\mu$.

        \hspace{0.5cm}
        C = $\int |f|^2 d\mu$
        $\geq$ $\int_{E_n} |f|^2 d\mu$
        $\geq$ $\int_{E_n} n^2 d\mu$
        = $n^2 \mu(E_n)$
        \hspace{0.5cm}
        $\Rightarrow$
        \hspace{0.5cm}
        $\mu(E_n)$ $\leq$ $\frac{C}{n^2}$

        Thus, $E_n$ is a null set and thus, measurable.
        Since f $\in$ $L^2[a,b]$, then $|f|^2$ is integrable
        so by {\color{red} theorem 18.2.5}, there is a $\delta > 0$
        where for $\mu(A)$ $<$ $\delta$, then
        $\int_A |f|^2 d\mu$ $<$ $\frac{\epsilon^2}{4}$.
        
        Since $|f(x) - f_n(x)|$ = 0 for x $\not \in$ $E_n$,
        then for n where $\mu(E_n)$ $\leq$ $\frac{C}{n^2}$ $<$ $\delta$:

        \hspace{0.5cm}
        $||f - f_n||^2$
        = $\int |f - f_n|^2 d\mu$
        = $\int_{E_n} |f - f_n|^2 d\mu$
            + $\int_{E_n^c} |f - f_n|^2 d\mu$

        \hspace{2.4cm}
        $\leq$ $\int_{E_n} 4|f|^2 d\mu$ + 0
        $<$ $4\frac{\epsilon^2}{4}$
        = $\epsilon^2$
    \end{proof}

    \vspace{0.5cm}



    \begin{wtheorem}{Approximating f $\in$ $L^2[a,b]$ with step or continuous
    functions}{14cm}
        For $\epsilon > 0$ and f $\in$ $L^2[a,b]$, there is a step function
        g such that $||f-g||$ $<$ $\epsilon$.

        Also, there is a continuous function h such that h(a) = h(b) and
        $||f-h||$ $<$ $\epsilon$.
    \end{wtheorem}

    \begin{proof}
        By {\color{red} theorem 19.2.2}, there is a n where
        $||f-f_n||$ $<$ $\frac{\epsilon}{2}$.
        Note $|f_n(x)|$ $\leq$ n for all x.

        Since $f_n$ is integrable, then by {\color{red} theorem 18.4.5},
        for $\delta > 0$, there is a step function g with $|g|$ $\leq$ n
        and measurable set A where $\mu(A) < \delta$ such that for x $\not \in$ A:

        \hspace{0.5cm}
        $|f_n(x) - g(x)|$ $<$ $\delta$

        Thus, for $\delta$ where
        $4n^2\delta +(b-a)\delta^2$ $<$ $\frac{\epsilon^2}{4}$:

        \hspace{0.5cm}
        $||f_n - g||^2$
        = $\int |f_n - g|^2 d\mu$
        = $\int_A |f_n - g|^2 d\mu$ + $\int_{A^c} |f_n - g|^2 d\mu$

        \hspace{2.4cm}
        $\leq$ $\int_A (2n)^2 d\mu$ + $\int_{A^c} \delta^2 d\mu$
        = $4n^2 \mu(A)$ + $\delta^2 \mu(A^c)$
        = $4n^2 \delta + (b-a)\delta^2$
        $<$ $\frac{\epsilon^2}{4}$

        \hspace{0.5cm}
        $||f_n - g||$ $<$ $\frac{\epsilon}{2}$
        \hspace{0.5cm}
        $\Rightarrow$
        \hspace{0.5cm}
        $||f-g||$
        $\leq$ $||f-f_n|| + ||f_n-g||$
        $<$ $\frac{\epsilon}{2} + \frac{\epsilon}{2}$
        = $\epsilon$

        Since if f is integrable, there is a continuous h where h(a) = h(b)
        and a measurable set A where $\mu(A) < \epsilon$
        such that $|f(x)-h(x)| < \epsilon$ for all x $\not \in$ A,
        then the proof for continuous function h is similar.
    \end{proof}

    \vspace{0.5cm}



    \begin{definition}{Hilbert Space}{14cm}
        A {\color{lblue} Hilbert Space}
        is a vector space with an inner product whose associated norm
        is complete
        (i.e. Cauchy sequences converge in the norm of the vector space).
    \end{definition}

    \newpage



    \begin{wtheorem}{$L^2[a,b]$ is complete}{14cm}
        $L^2[a,b]$ is a Hilbert Space
    \end{wtheorem}

    \begin{proof}
        By {\color{red} theorem 19.1.9}, $L^2[a,b]$ is an inner product space.

        Let \{$f_n$\} be a Cauchy sequence.
        Then there are $n_i$ such that for m,n $\geq$ $n_i$:

        \hspace{0.5cm}
        $||f_m - f_n||$ $<$ $\frac{1}{2^i}$

        Let $g_0$ = 0 and $g_i$ = $f_{n_i}$.
        Then $||g_{i+1} - g_i||$ $<$ $\frac{1}{2^i}$
        so $\sum_{i=0}^{\infty}$ $||g_{i+1} - g_i||$ converges to S.

        Let $h_n(x)$ = $\sum_{i=0}^{n-1}$ $|g_{i+1}(x) - g_i(x)|$
        and h(x) = $\lim_{n \rightarrow \infty}$ $h_n(x)$.

        \hspace{0.5cm}
        $||h_n||$
        $\leq$ $\sum_{i=0}^{n-1}$ $||g_{i+1} - g_i||$
        $\leq$ $\sum_{i=0}^{\infty}$ $||g_{i+1} - g_i||$
        = S

        \hspace{0.5cm}
        $\int h_n^2$ = $||h_n||^2$ $\leq$ $S^2$

        Since $h_n(x)$ is monotonically increasing
        so $h_n(x)^2$ is monotonically increasing converging to $h(x)^2$,
        then by {\color{red} theorem 18.3.3}:

        \hspace{0.5cm}
        $\int h^2 d\mu$ = $\lim_{n \rightarrow \infty}$ $\int h_n(x) d\mu$
        $\leq$ $S^2$

        Thus, $h^2$ is integrable and thus, finite almost everywhere.
        For x where h(x) is finite, $\sum_{i=0}^{\infty} (g_{i+1}(x) - g_i(x))$
        converges absolutely and thus, converges.
        
        Let g(x) =
        $\begin{cases}
            \sum_{i=0}^{\infty} (g_{i+1}(x) - g_i(x))
                = \lim_{n \rightarrow \infty} g_n(x) & \text{h(x) is finite} \\
            0 & \text{h(x) is infinite}
        \end{cases}$.

        Thus, for almost all x:

        \hspace{0.5cm}
        $|g(x)|$
        = $\lim_{n \rightarrow \infty}$ $|g_n(x)|$
        $\leq$ $\lim_{n \rightarrow \infty}$
            $\sum_{i=0}^{n-1}$ $|g_{i+1}(x) - g_i(x)|$
        = $\lim_{n \rightarrow \infty}$ $h_n(x)$
        = h(x)

        Thus, $|g(x)|^2$ $\leq$ $h(x)^2$
        so $|g(x)|^2$ is integrable where g(x) $\in$ $L^2[a,b]$.

        Since $\lim_{n \rightarrow \infty}$ $|g(x) - g_n(x)|^2$ = 0
        for almost all x and

        \hspace{0.5cm}
        $|g(x) - g_n(x)|^2$
        $\leq$ $(|g(x)| + |g_n(x)|)^2$
        $\leq$ $(2h(x))^2$
        
        then by {\color{red} theorem 18.4.4}:

        \hspace{0.5cm}
        $\lim_{n \rightarrow \infty}$ $\int |g(x) - g_n(x)|^2$ $d\mu$ = 0

        Thus, $\lim_{n \rightarrow \infty}$ $||g-g_n||$ = 0
        so there is an i such that $||g-g_i||$ $<$ $\frac{1}{2^i}$
        $<$ $\frac{\epsilon}{2}$.

        Thus, for any m $\geq$ $n_i$:

        \hspace{0.5cm}
        $||g - f_m||$
        $\leq$ $||g - g_i|| + ||g_i - f_m||$
        = $||g - g_i|| + ||f_{n_i} - f_m||$
        $<$ $\frac{\epsilon}{2} + \frac{1}{2^i}$
        $<$ $\frac{\epsilon}{2} + \frac{\epsilon}{2}$
        = $\epsilon$

        Thus, $\lim_{m \rightarrow \infty}$ $||g - f_m||$ = 0
        where g $\in$ $L^2[a,b]$
        so every Cauchy sequence converges in the $L^2$ norm.
    \end{proof}

    \vspace{0.5cm}



    \begin{corollary}{Convergent \{$f_n(x)$\} in $L^2[a,b]$
    implies convergent \{$f_{n_i}(x)$\} }{14cm}
        If \{$f_n$\} converges to f in $L^2[a,b]$, then
        there is a subsequence \{$f_{n_i}$\} such that:

        \hspace{0.5cm}
        $\lim_{i \rightarrow \infty}$ $f_{n_i}(x)$ = f(x)

        for almost all x $\in$ [a,b]
    \end{corollary}

    \begin{proof}
        Since \{$f_n$\} converges to f in $L^2[a,b]$,
        then \{$f_n$\} is Cauchy in $L^2[a,b]$.

        For {\color{red} theorem 19.2.5}'s proof,
        there is g(x) = $\lim_{i \rightarrow \infty} g_i(x)$
        where $\lim_{i \rightarrow \infty}$ $||g-g_i||$ = 0 and
        $g_i$ = $f_{n_i}$ for almost all x.
        
        Since \{$g_i$\} converges to g and f in $L^2[a,b]$,
        then g(x) = f(x) for almost all x.

        \hspace{0.5cm}
        $\lim_{i \rightarrow \infty}$ $f_{n_i}(x)$
        = $\lim_{i \rightarrow \infty}$ $g_i(x)$
        = g(x) = f(x)
    \end{proof}

    \newpage





\subsection{ Hilbert Space }

    \begin{definition}{Absolute Convergence}{14cm}
        If \{$u_m$\} is a sequence in Hilbert space $\mathcal{H}$, then
        $\sum_{m=1}^{\infty}$ $u_m$ {\color{lblue} converges absolutely}
        if $\sum_{m=1}^{\infty}$ $||u_m||$ converges
    \end{definition}

    \vspace{0.5cm}



    \begin{wtheorem}{Absolute convergence implies convergence}{14cm}
        If $\sum_{m=1}^{\infty}$ $u_m$ in $\mathcal{H}$ converges absolutely, then
        it converges
    \end{wtheorem}

    \begin{proof}
        Since $\sum_{m=1}^{\infty}$ $u_m$ converges absolutely,
        then there is a N such that for n $>$ m $\geq$ N:

        \hspace{0.5cm}
        $\sum_{i=m}^n$ $||u_i||$
        $\leq$ $\sum_{i=m}^{\infty}$ $||u_i||$
        $<$ $\epsilon$

        Let $s_n$ = $\sum_{i=1}^n$ $u_i$. Then:

        \hspace{0.5cm}
        $||s_n - s_m||$
        $\leq$ $\sum_{i=m}^n$ $||u_i||$
        $<$ $\epsilon$

        Thus, $s_n$ is Cauchy so \{$s_n$\} = $\sum_{i=1}^n$ $u_i$ converges.
    \end{proof}

    \vspace{0.5cm}



    \begin{wtheorem}{Pythagorean Theorem}{14cm}
        x,y $\in$ $\mathcal{H}$ are {\color{lblue} perpendicular}, x $\perp$ y,
        if $\langle x , y \rangle$ = 0

        If $x_1,...,x_n$ $\in$ $\mathcal{H}$ are mutually perpendicular, then:

        \hspace{0.5cm}
        $|| \sum_{i=1}^n x_i ||^2$
        = $\sum_{i=1}^n$ $||x_i||^2$
    \end{wtheorem}

    \begin{proof}
        Since $\langle x_i , x_j \rangle$ = 0 for any i $\not =$ j, then:

        \hspace{0.5cm}
        $|| \sum_{i=1}^n x_i ||^2$
        = $\langle \sum_{i=1}^n x_i , \sum_{i=1}^n x_i \rangle$
        = $\sum_{i=1}^n$ $\langle x_i , x_i \rangle$
            + $2 \sum_{i \not = j}$ $\langle x_i , x_j \rangle$
        = $\sum_{i=1}^n$ $||x_i||^2$
    \end{proof}

    \vspace{0.5cm}



    \begin{definition}{Bounded Linear Functional}{14cm}
        A {\color{lblue} bounded linear functional}
        L: $\mathcal{H}$ $\rightarrow$ $\mathbb{R}$
        where for all v,w $\in$ $\mathcal{H}$ and $c_1,c_2$ $\in$ $\mathbb{R}$:

        \hspace{0.5cm}
        $L(c_1v+c_2w)$ = $c_1L(v) + c_2L(w)$
        \hspace{1cm}
        $|L(v)|$ $\leq$ $M||v||$
    \end{definition}

    \vspace{0.5cm}



    \begin{wtheorem}{Cauchy-Schwarz Inequality for $\mathcal{H}$}{14cm}
        For Hilbert space $(H, \langle \ \rangle)$ where v,w $\in$ $\mathcal{H}$:

        \hspace{0.5cm}
        $|\langle v , w \rangle|$
        $\leq$ $||v||$ $||w||$

        with equality if and only if w and w are multiples of a vector
    \end{wtheorem}

    \begin{proof}
        For fixed x $\in$ $\mathcal{H}$, define
        L: $\mathcal{H}$ $\rightarrow$ $\mathbb{R}$
        by L(v) = $\langle v , x \rangle$.
        h
        Then L is linear by {\color{red} theorem 19.1.9}
        and bounded by {\color{orange} corollary 19.1.6}
        since $|L(v)|$ $\leq$ $||v||$ $||x||$
        where $|L(v)|$ = $||v||$ $||x||$ if v = cx almost everywhere for some c.
    \end{proof}

    \newpage



    \begin{wtheorem}{inf($L^{-1}(1)$) is unique
    and perpendicular to $L^{-1}(0)$}{14cm}
        For bounded linear functional L: $\mathcal{H}$ $\rightarrow$ $\mathbb{R}$
        not identically 0, let $\mathcal{V}$ = $L^{-1}(1)$.
        Then there is a unique x $\in$ $\mathcal{V}$ such that:

        \hspace{0.5cm}
        $||x||$ = $\underset{v \in \mathcal{V}}{\text{inf}} (||v||)$

        Also, x is perpendicular to every v $\in$ $L^{-1}(0)$.
    \end{wtheorem}

    \begin{proof}
        For $x_n$ $\in$ $\mathcal{V}$, let $\lim_{n \rightarrow \infty}$ $x_n$ = x.

        \hspace{0.5cm}
        $|L(x) - L(x_n)|$
        = $|L(x - x_n)|$
        $\leq$ $M || x-x_n ||$

        \hspace{0.5cm}
        $|L(x) - 1|$
        $\leq$ $\lim_{n \rightarrow \infty}$ $M || x-x_n ||$
        = 0

        Thus, L(x) = 1 so x $\in$ $\mathcal{V}$
        and thus, $\mathcal{V}$ is closed.

        Let d = $\underset{v \in \mathcal{V}}{\text{inf}} (||v||)$
        and \{$x_n$\} $\in$ $\mathcal{V}$ such that
        $\lim_{n \rightarrow \infty}$ $||x_n||$ = d.

        Since $\frac{x_n + x_m}{2}$ $\in$ $\mathcal{V}$, then
        $|| \frac{x_n + x_m}{2} ||$ $\geq$ d.

        Since $||x_n - x_m||^2 + ||x_n + x_m||^2$
        = $2||x_n||^2 + 2||x_m||^2$, then:

        \hspace{0.5cm}
        $||x_n - x_m||^2$
        = $2||x_n||^2 + 2||x_m||^2 - ||x_n + x_m||^2$
        $\leq$ $2||x_n||^2 + 2||x_m||^2 - 4d^2$

        Thus, as n,m $\rightarrow$ $\infty$, then
        $2||x_n||^2 + 2||x_m||^2 - 4d^2$ $\rightarrow$ 0
        so $||x_n - x_m||$ $\rightarrow$ 0.
        Thus, \{$x_n$\} is Cauchy and thus, converges.
        Let $\lim_{n \rightarrow \infty}$ $x_n$ = x.

        \hspace{0.5cm}
        $||x||$
        $\leq$ $\lim_{n \rightarrow \infty} || x-x_n ||$
                + $\lim_{n \rightarrow \infty} || x_n ||$
        = 0+d = d

        Since $\mathcal{V}$ is closed, then x $\in$ $\mathcal{V}$
        so $||x||$ $\geq$ d and since $||x||$ $\leq$ d, then $||x||$ = d.
        
        Suppose there is a y $\in$ $\mathcal{V}$ where $||y||$ = d.
        Then $\frac{x+y}{2}$ $\in$ $\mathcal{V}$ so
        $|| \frac{x+y}{2} ||$ $\geq$ d.
        
        \hspace{0.5cm}
        $||x-y||^2$ = $2||x||^2 + 2||y||^2 - ||x+y||^2$
        $\leq$ $4d^2 - 4d^2$ = 0

        Thus, x = y.
        Suppose v $\in$ $L^{-1}(0)$.
        For any t $\in$ $\mathbb{R}$, then x+tv $\in$ $L^{-1}(1)$ where:
        
        \hspace{0.5cm}
        $||x+tv||^2$ $\geq$ $||x||^2$

        \hspace{0.5cm}
        $||x||^2 + 2t \langle x , v \rangle + t^2 ||v||^2$ $\geq$ $||x||^2$

        \hspace{0.5cm}
        $2t \langle x , v \rangle + t^2 ||v||^2$ $\geq$ 0

        Suppose $\langle x , v \rangle$ $>$ 0. Choose t $<$ 0
        such that $2 \langle x , v \rangle + t ||v||^2$ $>$ 0.
        Thus, $2t \langle x , v \rangle + t^2 ||v||^2$ $<$ 0
        Suppose $\langle x , v \rangle$ $<$ 0. Choose t $>$ 0
        such that $2 \langle x , v \rangle + t ||v||^2$ $<$ 0.
        Thus, $2t \langle x , v \rangle + t^2 ||v||^2$ $<$ 0.
        Thus by contradiction, $\langle x , v \rangle$ = 0.
    \end{proof}

    \vspace{0.5cm}



    \begin{wtheorem}{The bounded linear functionals of $\mathcal{H}$ are unique}{14cm}
        For bounded linear functional L: $\mathcal{H}$ $\rightarrow$ $\mathbb{R}$,
        there is a unique x $\in$ $\mathcal{H}$ such that:

        \hspace{0.5cm}
        L(v) = $\langle v , x \rangle$
    \end{wtheorem}

    \begin{proof}
        If L(v) = 0 for all v, then x = 0 satisfy the condition.
        Suppose L(v) $\not =$ 0, then by {\color{red} theorem 19.3.6},
        there is a unique $x_0$ $\in$ $L^{-1}(1)$ with the smallest norm.

        Suppose v $\in$ $L^{-1}(1)$.
        Then, L($v - x_0$) = L(v) - L($x_0$) = 1-1 = 0
        so by {\color{red} theorem 19.3.6},
        then $\langle v - x_0, x_0 \rangle$ = 0.
        Thus, x = $\frac{x_0}{||x_0||^2}$ is perpendicular to $v-x_0$.

        \hspace{0.5cm}
        $\langle v , x \rangle$
        = $\langle v-x_0 , \frac{x_0}{||x_0||^2} \rangle$
            + $\langle x_0 , \frac{x_0}{||x_0||^2} \rangle$
        = 0+1 = 1 = L(v)

        Also, by {\color{red} theorem 19.3.6}, for v $\in$ $L^{-1}(0)$,
        then L(v) = 0 = $\langle v , x \rangle$.

        Then for w $\in$ $L^{-1}(c)$ $\not =$ 0, let v = $\frac{w}{c}$
        so L(v) = $\frac{1}{c} L(w)$ = $\frac{1}{c} c$ = 1.

        \hspace{0.5cm}
        L(w) = L(cv) = cL(v) = c$\langle v , x \rangle$
        = $\langle cv , x \rangle$ = $\langle w , x \rangle$

        Suppose y $\in$ $\mathcal{H}$ satisfy L(v) = $\langle v , y \rangle$
        for all v $\in$ $\mathcal{H}$.
        Then for every v $\in$ $\mathcal{H}$:

        \hspace{0.5cm}
        $\langle v , x \rangle$
        = L(v) = $\langle v , y \rangle$
        \hspace{1cm}
        $\Rightarrow$
        \hspace{1cm}
        $\langle v , x-y \rangle$ = 0

        Take v = x-y so $||x-y||^2$ = $\langle x-y , x-y \rangle$ = 0
        so x = y.
    \end{proof}

    \newpage





\subsection{ Fourier Series }

    \begin{definition}{Orthonormal Family}{14cm}
        \{$u_n$\} $\in$ $\mathcal{H}$ are {\color{lblue} orthonormal}
        if $||u_n||$ = 1 and $\langle u_n , u_m \rangle$ = 0 for n $\not =$ m
    \end{definition}

    \vspace{0.5cm}



    \begin{wtheorem}{Minimal Distance of w $\in$ $\mathcal{H}$
    to orthonormal basis}{14cm}
        If $u_0,...,n_N$ $\in$ $\mathcal{H}$ are orthonormal
        and w $\in$ $\mathcal{H}$, then the $c_n$ to minimize

        \hspace{0.5cm}
        $|| w - \sum_{n=0}^N c_nu_n ||$

        are $c_n$ = $\langle w , u_n \rangle$
    \end{wtheorem}

    \begin{proof}
        Let v = $\sum_{n=0}^N c_nu_n$ and u = $\sum_{n=0}^N a_nu_n$
        where $a_n$ = $\langle w , u_n \rangle$. Since:

        \hspace{0.5cm}
        $\langle v , v \rangle$
        = $\sum_{n=0}^N$ $|c_n|^2$
        \hspace{1cm}
        $\langle u , u \rangle$
        = $\sum_{n=0}^N$ $|a_n|^2$

        \hspace{0.5cm}
        $\langle w , v \rangle$
        = $\sum_{n=0}^N$ $c_n \langle w , u_n \rangle$
        = $\sum_{n=0}^N$ $a_nc_n$

        then:

        \hspace{0.5cm}
        $||w-v||^2$
        = $\langle w-v , w-v \rangle$
        = $||w||^2 - 2 \langle w , v \rangle + ||v||^2$

        \hspace{2.3cm}
        = $||w||^2 - 2 \sum_{n=0}^N a_nc_n + \sum_{n=0}^N |c_n|^2$
        
        \hspace{2.3cm}
        = $||w||^2 - \sum_{n=0}^N |a_n|^2 + \sum_{n=0}^N (a_n-c_n)^2$
        = $||w||^2 - ||u||^2 + \sum_{n=0}^N |a_n-c_n|^2$

        Thus, for any $c_n$,
        $||w-v||^2$
        $\geq$ $||w||^2 - ||u||^2$
        where equality holds if $c_n$ = $a_n$.
    \end{proof}

    \vspace{0.5cm}



    \begin{definition}{Complete Orthonormal Family and Fourier Series}{14cm}
        Orthonormal \{$u_n$\} $\in$ $\mathcal{H}$ is {\color{lblue} complete}
        if for every w $\in$ $\mathcal{H}$:

        \hspace{0.5cm}
        w = $\sum_{n=0}^{\infty}$ $c_n u_n$

        The n-th Fourier coefficient of w with respect to \{$u_n$\}
        is $\langle w , u_n \rangle$.

        Then, $\sum_{n=0}^{\infty}$ $\langle w , u_n \rangle u_n$
        is called the {\color{lblue} Fourier series of w}.
    \end{definition}

    \vspace{0.5cm}



    \begin{wtheorem}{Bessel's Inequality}{14cm}
        For orthonormal \{$u_i$\} $\in$ $\mathcal{H}$
        where w $\in$ $\mathcal{H}$:

        \hspace{0.5cm}
        $\sum_{i=0}^{\infty}$ $|\langle w , u_i \rangle|^2$
        $\leq$ $||w||^2$

        converges
    \end{wtheorem}

    \begin{proof}
        Let $s_n$ = $\sum_{i=0}^{n}$ $\langle w , u_i \rangle u_i$.
        Since $||s_n||^2$ = $\sum_{i=0}^{n}$ $|\langle w , u_i \rangle|^2$, then:

        \hspace{0.5cm}
        $\langle w-s_n , s_n \rangle$
        = $\langle w , s_n \rangle$ - $\langle s_n , s_n \rangle$
        = $\sum_{i=0}^{n}$ $|\langle w , u_i \rangle|^2$ - $||s_n||^2$
        = 0

        Thus, $w-s_n$ and $s_n$ are perpendicular so
        $||w||^2$ = $||s_n||^2$ + $||w-s_n||^2$. Thus:

        \hspace{0.5cm}
        $\sum_{i=0}^{n}$ $|\langle w , u_i \rangle|^2$
        = $||s_n||^2$ $\leq$ $||w||^2$

        Since $||s_n||^2$ is increasing and bounded by $||w||^2$, then:

        \hspace{0.5cm}
        $\sum_{i=0}^{\infty}$ $|\langle w , u_i \rangle|^2$
        = $\lim_{n \rightarrow \infty}$ $||s_n||^2$ $\leq$ $||w||^2$
    \end{proof}

    \vspace{0.5cm}



    \begin{wtheorem}{Fourier Series Converge}{14cm}
        For orthonormal \{$u_n$\} $\in$ $\mathcal{H}$ where w $\in$ $\mathcal{H}$,
        then $\sum_{i=0}^{\infty}$ $\langle w , u_i \rangle u_i$ converges.

        If \{$u_n$\} is complete, then
        $\sum_{i=0}^{\infty}$ $c_i u_i$ converges to w
        must have $c_i$ = $\langle w , u_i \rangle$.
    \end{wtheorem}

    \begin{proof}
        Let $s_n$ = $\sum_{i=0}^{n}$ $\langle w , u_i \rangle u_i$.
        For n $>$ m, then $s_n - s_m$
        = $\sum_{i=m+1}^{n}$ $\langle w , u_i \rangle u_i$
        where $||s_n - s_m||^2$
        = $\sum_{i=m+1}^{n}$ $|\langle w , u_i \rangle|^2$
        which converges so \{$s_n$\} is Cauchy and thus, converges.

        If \{$u_n$\} is complete, then there are $c_i$ such that
        $S_n$ = $\sum_{i=0}^n$ $c_i u_i$ $\rightarrow$ w.

        Since bounded linear L(x) = $\langle x , u_i \rangle$
        has $|L(x)|$ $\leq$ $M||x||$, then L(x) is continuous.

        \hspace{0.5cm}
        $\langle w , u_i \rangle$
        = $\langle \lim_{n \rightarrow \infty} S_n , u_i \rangle$
        = $\lim_{n \rightarrow \infty} \langle S_n , u_i \rangle$
        = $c_i$

        \hspace{0.5cm}
        $\lim_{n \rightarrow \infty}$ $\sum_{i=0}^{n}$ $\langle w , u_i \rangle u_i$
        = $\lim_{n \rightarrow \infty}$ $\sum_{i=0}^n$ $c_i u_i$ $\rightarrow$ w.
    \end{proof}

    \newpage



    \begin{wtheorem}{Parseval's Theorem}{14cm}
        For orthonormal \{$u_n$\} $\in$ $\mathcal{H}$ where w $\in$ $\mathcal{H}$,
        then:
        
        \hspace{0.5cm}
        $\sum_{i=0}^{\infty}$ $|\langle w , u_i \rangle|^2$ = $||w||^2$
        if and only if $\sum_{i=0}^{\infty}$ $\langle w , u_i \rangle u_i$ = w
    \end{wtheorem}

    \begin{proof}
        Let $s_n$ = $\sum_{i=0}^{n}$ $\langle w , u_i \rangle u_i$.
        Note $||w||^2$ = $||s_n||^2$ + $||w-s_n||^2$.

        If $\lim_{n \rightarrow \infty}$ $||s_n||^2$
        = $\sum_{i=0}^{\infty}$ $|\langle w , u_i \rangle|^2$ = $||w||^2$,
        then $\lim_{n \rightarrow \infty}$ $||w-s_n||^2$ = 0 so
        
        $\lim_{n \rightarrow \infty}$ $||w-s_n||$ = 0.
        Thus, $\sum_{i=0}^{\infty}$ $\langle w , u_i \rangle u_i$ = w.

        If $\sum_{i=0}^{\infty}$ $\langle w , u_i \rangle u_i$ = w,
        then $\lim_{n \rightarrow \infty}$ $||w-s_n||$ = 0
        so $\lim_{n \rightarrow \infty}$ $||w-s_n||^2$ = 0.
        Thus, $\sum_{i=0}^{\infty}$ $|\langle w , u_i \rangle|^2$
        = $\lim_{n \rightarrow \infty}$ $||s_n||^2$ = $||w||^2$.
    \end{proof}





































