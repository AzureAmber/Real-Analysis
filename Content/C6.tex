\newpage
\section[Day 6: Existence of $\mathbb{R}$]
{Construction of $\mathbb{R}$: {\normalsize \color{red} Theorem 2.3.3}}

	\begin{adjustbox}{minipage=14cm, right, vspace=0.1cm 0cm}
		There exists an ordered field $\mathbb{R}$ which has the
		least upper bound property.

		Also, $\mathbb{R}$ contains $\mathbb{Q}$ as a subfield. \\
	\end{adjustbox}
	
{ \color{blue} Definition 6.1: Cuts }

	\begin{adjustbox}{minipage=14cm, right, vspace=0.1cm 0cm}
		Define a cut as any set $\alpha$ $\subset$ $\mathbb{Q}$ with the properties:
	\end{adjustbox}

	\begin{itemize}[leftmargin=1cm, itemsep=0.4em]
		\item $\alpha$ is not empty and $\alpha$ $\not =$ $\mathbb{Q}$
		
		\item If p $\in$ $\alpha$ and q $\in$ $\mathbb{Q}$ $<$ p,
			then q $\in$ $\alpha$
		
			\item If p $\in$ $\alpha$, then p $<$ r $\in$ $\mathbb{Q}$ for
			some r $\in$ $\alpha$ \\
	\end{itemize}

{ \color{blue} Proposition 6.2: Order of $\mathbb{R}$ $\rightarrow$ ordered set $\mathbb{R}$ }

	\begin{adjustbox}{minipage=14cm, right, vspace=0.1cm 0cm}
		Define $\alpha$ $<$ $\beta$ if $\alpha$ is a proper subset of $\beta$.

		\begin{itemize}[leftmargin=1cm, itemsep=0.4em]
			\item If $\alpha$ $\not \geq$ $\beta$, then $\beta$ is not a subset of $\alpha$.

				Then there is a p $\in$ $\beta$ such that p $\not \in$ $\alpha$.

				Then for any q $\in$ $\alpha$, q $<$ p and thus, q $\in$ $\beta$.

				Thus, $\alpha$ $\subset$ $\beta$ and since $\alpha$ $\neq$ $\beta$,
				then $\alpha$ $<$ $\beta$.

			\item If $\alpha$ $\not <$ $\beta$ and $\alpha$ $\not >$ $\beta$,
				then either $\alpha$ = $\beta$ or $\alpha$ $\neq$ $\beta$.

				If $\alpha$ $\neq$ $\beta$, there are p,q such that
				p $\in$ $\alpha$, but p $\not \in$ $\beta$ and q $\in$ $\beta$,
				but q $\not \in$ $\alpha$.

				But if p $\not \in$ $\beta$, then for any b $\in$ $\beta$, b $<$ p.
				Thus, q $<$ p.

				Similarly, if q $\not \in$ $\alpha$, then for any a $\in$ $\alpha$,
				a $<$ q. Thus, p $<$ q.

				Thus, there is a contradiction since p $>$ q and p $<$ q so
				$\alpha$ = $\beta$.

			\item If $\alpha$ $\not \leq$ $\beta$, then $\alpha$ is not a subset of $\beta$.

				Then there is a p $\in$ $\alpha$ such that p $\not \in$ $\beta$.

				Then for any q $\in$ $\beta$, q $<$ p and thus, q $\in$ $\alpha$.

				Thus, $\beta$ $\subset$ $\alpha$ and since $\alpha$ $\neq$ $\beta$,
				then $\beta$ $<$ $\alpha$.

			\item If $\alpha$ $<$ $\beta$ and $\beta$ $<$ $\gamma$, then since
				$\alpha$ is a proper subset of $\beta$ and $\beta$ is a proper subset
				of $\gamma$, then $\alpha$ is a proper subset of $\gamma$.
				Thus, $\alpha$ $<$ $\gamma$.
		\end{itemize}

		Thus, $\mathbb{R}$ is an ordered set with such an order $<$. \\
	\end{adjustbox}

{ \color{blue} Proposition 6.3: Least Upper Bound of $\mathbb{R}$
$\rightarrow$ Least Upper Bound Property } 

	\begin{adjustbox}{minipage=14cm, right, vspace=0.1cm 0cm}
		Let A $\subset$ $\mathbb{R}$ and $\beta$ be an upper bound for A.
		Let $\gamma$ be the union of all $\alpha$ $\in$ A.

		Thus, p $\in$ $\gamma$ if and only if p $\in$ $\alpha$ for some $\alpha$ $\in$ A.

		$\gamma$ defines a cut since:

		\begin{itemize}[leftmargin=1cm, itemsep=0.4em]
			\item Since A is nonempty, there exists a $\alpha_0$ $\in$ A where $\alpha_0$ is nonempty.

				Since $\alpha_0$ is nonempty, then $\gamma$ is nonempty.

				Since every $\alpha$ $\in$ A is $\alpha$ $<$ $\beta$, then $\gamma$ $<$ $\beta$
				so $\gamma$ $\subset$ $\beta$ and thus, $\gamma$ $\neq$ $\mathbb{Q}$.

			\item If p $\in$ $\gamma$, then p $\in$ $\alpha_1$ for some $\alpha_1$ $\in$ A.
				If q $<$ p, then q $\in$ $\alpha_1$ so q $\in$ A.

			\item If p $\in$ $\gamma$, then p $\in$ $\alpha_1$ for some $\alpha_1$ $\in$ A.
				Thus, there is a r $\in$ $\alpha_1$ such that r $>$ p so r $\in$ $\gamma$.
				Thus, there is a r $\in$ $\gamma$ where r $>$ p.
		\end{itemize}

		Since $\gamma$ defines a cut, then $\gamma$ $\in$ $\mathbb{R}$.
		Since every $\alpha$ $\in$ A $\subset$ $\gamma$, then $\alpha$ $\leq$ $\gamma$
		so $\gamma$ is an upper bound for A.

		Suppose $\delta$ $<$ $\gamma$. Then there is a s $\in$ $\gamma$
		such that s $\not \in$ $\delta$.
		Since s $\in$ $\gamma$, then there is a $\alpha$ $\in$ A such that s $\in$ $\alpha$.
		Since $\delta$ $<$ $\alpha$, then $\delta$ is not an upper bound of A.

		Thus, $\gamma$ = sup(A).
	\end{adjustbox}

\newpage

{ \color{blue} Proposition 6.4: $\mathbb{R}$ is a field} 

	\begin{adjustbox}{minipage=14cm, right, vspace=0.1cm 0cm}
		If $\alpha$, $\beta$ $\in$ $\mathbb{R}$, define $\alpha$ + $\beta$ as the set of all
		sums r + s where r $\in$ $\alpha$ and s $\in$ $\beta$.

		Also, let 0* be the set of all negative rational numbers which is a cut since:
		
			\begin{itemize}[leftmargin=1cm, itemsep=0.4em]
				\item 0* is nonempty and 0* $\not =$ $\mathbb{Q}$

				\item If p $\in$ 0*, then any q $\in$ $\mathbb{Q}$ $<$ p is a negative rational
					and thus, q $\in$ 0*.

				\item Since $\mathbb{Q}$ is dense in $\mathbb{R}$, then for any p $\in$ 0*,
					there is a r $\in$ $\mathbb{Q}$ where p $<$ r $<$ 0 so r is a
					negative rational so r $\in$ 0*.
			\end{itemize}

		$\alpha$ + $\beta$ $\in$ $\mathbb{R}$ since $\alpha$ + $\beta$ is a cut:

			\begin{itemize}[leftmargin=1cm, itemsep=0.4em]
				\item $\alpha$ + $\beta$ is non-empty since $\alpha$, $\beta$ are non-empty.
					Take r' $\not \in$ $\alpha$, s' $\not \in$ $\beta$, then
					r' + s' $>$ r + s for r $\in$ $\alpha$, s $\in$ $\beta$.
					Thus, r' + s' $\not \in$ $\alpha$ + $\beta$ so
					$\alpha$ + $\beta$ $\neq$ $\mathbb{Q}$.

				\item If p $\in$ $\alpha$ + $\beta$, then p = r + s where r $\in$ $\alpha$
					and s $\in$ $\beta$.

					If q $<$ p, then q - s $<$ p - s = (r + s) - s = r so q - s $\in$ $\alpha$.

					Since q - s $\in$ $\alpha$ and s $\in$ $\beta$, then
					(q - s) + s = q $\in$ $\alpha$ + $\beta$.

				\item If r $\in$ $\alpha$, then there is a t $\in$ $\alpha$ such that t $>$ r.
					Let s $\in$ $\beta$.

					Thus, for any p = r + s $\in$ $\alpha$ + $\beta$, there is a
					q = t + s $\in$ $\alpha$ + $\beta$ such that p = r + s $<$ t + s = q.
			\end{itemize}

		$\alpha$ + $\beta$ = $\beta$ + $\alpha$

			\qquad If p = r + s $\in$ $\alpha$ + $\beta$ where
			r $\in$ $\alpha$, s $\in$ $\beta$, then s + r = r + s = p $\in$ $\beta$ + $\alpha$.

		($\alpha$ + $\beta$) + $\gamma$  = $\alpha$ + ($\beta$ + $\gamma$)

			\qquad If r $\in$ $\alpha$, s $\in$ $\beta$, t $\in$ $\gamma$, then
			r + s + t = (r + s) + t $\in$ ($\alpha$ + $\beta$) + $\gamma$ and

			\qquad r + s + t = r + (s + t) $\in$ $\alpha$ + ($\beta$ + $\gamma$).

		$\alpha$ + 0* = $\alpha$

			\qquad If r $\in$ $\alpha$, s $\in$ 0*, then r + s $<$ r. Thus, r + s $\in$ $\alpha$.
			Thus, $\alpha$ + 0* $\subset$ $\alpha$.

			\qquad If p $\in$ $\alpha$, there is a r $\in$ $\alpha$ where r $>$ p.
			Thus, p - r $\in$ 0*.

			\qquad Since p = r + (p - r) $\in$ $\alpha$ + 0*, then $\alpha$ $\subset$ $\alpha$ + 0*.
			Thus, $\alpha$ + 0* = $\alpha$.

		There is a -$\alpha$ such that $\alpha$ + -$\alpha$ = 0*

			\qquad Fix $\alpha$ $\in$ $\mathbb{R}$. Let set $\beta$ be all p where
			there is r $>$ 0 such that -p - r $\not \in$ $\alpha$.

			\qquad $\beta$ $\in$ $\mathbb{R}$ since $\beta$ is a cut:

			\begin{itemize}[leftmargin=2cm, itemsep=0.4em]
				\item If s $\not \in$ $\alpha$ and p = -s - 1, then -p - 1 $\not \in$ $\alpha$.
					Thus, p $\in$ $\beta$ so $\beta$ is nonempty.
					If q $\in$ $\alpha$, then -q $\not \in$ $\beta$ so $\beta$ $\neq$ $\mathbb{R}$.

				\item If p $\in$ $\beta$, let r $>$ 0 so -p - r $\not \in$ $\alpha$.
					If q $<$ p, then -q - r $>$ -p - r and thus, -q - r $\not \in$ $\alpha$
					so q $\in$ $\beta$.

				\item If p $\in$ $\beta$, let t = p + (r/2). Then
					-t - (r/2) = -p - r $\not \in$ $\alpha$ and thus, t $\in$ $\beta$
					where p $<$ t.
			\end{itemize}
			
			\qquad If r $\in$ $\alpha$, s $\in$ $\beta$, then s $\not \in$ $\alpha$. Thus,
			r $<$ -s so r + s $<$ 0. Thus, $\alpha$ + $\beta$ $\subset$ 0*.

			\qquad Let v $\in$ 0* and let w = -v/2 so w $>$ 0.

			\qquad Thus, by the Achimedean property, there is an integer n such that
			nw $\in$

			\qquad $\alpha$, but (n+1)w $\not \in$ $\alpha$.
			Let p = -(n+2)w so -p - w = (n+1)w $\not \in$ $\alpha$ so p $\in$ $\beta$.

			\qquad Then, v = -2w = nw + -nw - 2w = nw + -(n+2)w = nw + p $\in$ $\alpha$ + $\beta$.

			\qquad Since v $\in$ 0*, then 0* $\subset$ $\alpha$ + $\beta$.
			Thus, $\alpha$ + $\beta$ = 0*. Then, let -$\alpha$ = $\beta$.

		Thus, if $\alpha$, $\beta$, $\gamma$ $\in$ $\mathbb{R}$ and $\beta$ $<$ $\gamma$, then
		$\alpha$ + $\beta$ $<$ $\alpha$ + $\gamma$.

		Thus, if $\alpha$ $>$ 0*, then -$\alpha$ = -$\alpha$ + 0* $<$ -$\alpha$ + $\alpha$ = 0*
		so -$\alpha$ $<$ 0*.

		If $\alpha$, $\beta$ $\in$ $\mathbb{R}_+$, define $\alpha$$\beta$ as the set of all p
		such that p $\leq$ rs for r $\in$ $\alpha$, s $\in$ $\beta$.

		Define 1* as the set of all q $<$ 1.
		Then all multiplication axioms holds with similar proofs as addition.
		Also, note since $\alpha$, $\beta$ $>$ 0*, then $\alpha$$\beta$ $>$ 0*.

		Also, $\alpha$($\beta$ + $\gamma$) = $\alpha$$\beta$ + $\alpha$$\gamma$ holds
		through cases were $\alpha$, $\beta$, $\gamma$ $>$,$<$ 0*.
	\end{adjustbox}
































