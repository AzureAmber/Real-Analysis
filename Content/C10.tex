\newpage

\section[Day 10: Sequences and Series]{Sequences and Series}

\subsection{Convergent Sequences}

{ \color{blue} Definition 10.1.1: Convergent Sequence } 

    \begin{adjustbox}{minipage=14cm, right, vspace=0.1cm 0cm}
        A sequence \{$x_n$\} in metric space X converge if
        there is a x $\in$ X such that:

        \hspace{1cm}
        For every $\epsilon$ $>$ 0, there is a N $\in$ $\mathbb{Z}$ such that
        for all n $\geq$ N, d($x_n$,x) $<$ $\epsilon$
        
        Then, \{$x_n$\} converges to x: \hspace{1cm}
        $\lim_{n \rightarrow \infty}$ $x_n$ = x

        If \{$x_n$\} does not converge, then it diverges. \\
	\end{adjustbox}

{ \color{purple} Example 10.1.2 }

    \begin{enumerate}[label=(\alph*), leftmargin=2cm, itemsep=0.4em]
        \item Let $x_n$ = $\frac{1}{n}$ in $\mathbb{R}^2$.
        Then, $\lim_{n \rightarrow \infty}$ $x_n$ = 0

            { \color{magenta} \underline{Proof} }

                For $\epsilon$ $>$ 0, there is a $\frac{1}{N}$ $<$ $\epsilon$.
                Then:

                \hspace{1cm}
                d($x_n$,0) = $|x_n - 0|$ = $\frac{1}{n}$
                $<$ $\frac{1}{N}$ $<$ $\epsilon$

        \item Let $x_n$ = $(-1)^n$ + $\frac{1}{n}$ in $\mathbb{R}^2$.
        Then, \{$x_n$\} diverges.

            { \color{magenta} \underline{Proof} }

                $\lim_{n \rightarrow \infty}$ $x_n$
                = $\lim_{n \rightarrow \infty}$ $(-1)^n$
                + $\lim_{n \rightarrow \infty}$ $\frac{1}{n}$
                = $\lim_{n \rightarrow \infty}$ $(-1)^n$

                Since $(-1)^n$ alternates between -1 and 1, then
                \{$x_n$\} diverges. \\
    \end{enumerate}

{ \color{red} Theorem 10.1.3: }

    \begin{enumerate}[label=(\alph*), leftmargin=2cm, itemsep=0.4em]
        \item \{$p_n$\} converges to p $\in$ X if and only if
        every N$_r(p)$ contains $p_n$ for all, but finitely many n.

            { \color{magenta} \underline{Proof} }

                Suppose $p_n$ $\rightarrow$ p.

                Then for N$_{\epsilon}(p)$, any q $\in$ X such that
                d(q,p) $<$ $\epsilon$ is q $\in$ N$_{\epsilon}(p)$.
                Since $p_n$ $\rightarrow$ p, there is a N such that for
                n $\geq$ N, d($p_n$,p) $<$ $\epsilon$.

                Thus, for n $\geq$ N, $p_n$ $\in$ N$_{\epsilon}(p)$.

                Suppose every N$_r(p)$ contains $p_n$ for all, but finitely
                many n.

                For $\epsilon$ $>$ 0, let N$_{\epsilon}(p)$ be the set of
                all q $\in$ X such that d(p,q) $<$ $\epsilon$.
                Thus, there exists an N such that $p_n$ $\in$ N$_{\epsilon}(p)$
                if n $\geq$ N.
                
                Thus, d($p_n$,p) $<$ $\epsilon$ so $p_n$ $\rightarrow$ p.


        \item If p,p' $\in$ X and \{$p_n$\} converges to p and p', then
        p = p'.

            { \color{magenta} \underline{Proof} }

                For $\epsilon$ $>$ 0, there exists N,N' such that:

                \hspace{1cm}
                d($p_n$,p) $<$ $\frac{\epsilon}{2}$ for n $\geq$ N
                \hspace{1cm}
                d($p_n$,p') $<$ $\frac{\epsilon}{2}$ for n $\geq$ N'

                Then for n $\geq$ max(N,N'):

                \hspace{1cm}
                d(p,p') $\leq$ d(p,$p_n$) + d($p_n$,p') $<$ $\epsilon$

                Thus, p = p'.

        \item If \{$p_n$\} converges, then \{$p_n$\} is bounded.

            { \color{magenta} \underline{Proof} }

                If \{$p_n$\} $\rightarrow$ p,
                there is a N such that for n $>$ N, d($p_n$,p) $<$ 1.

                Let r = max( 1, d($p_1$,p), ... , d($p_N$,p) ).
                Thus for all n, d($p_n$,p) $\leq$ r.

        \item If E $\subset$ X and p $\in$ E', there is a \{$p_n$\}
        in E such that p = $\lim_{n \rightarrow \infty}$ $p_n$.

            { \color{magenta} \underline{Proof} }

                Since p $\in$ E', then for each n $\in$ $\mathbb{Z}_+$,
                there is a $p_n$ $\in$ E such that d($p_n$,p) $<$ $\frac{1}{n}$.
                For $\epsilon$ $>$ 0, there is a $\frac{1}{N}$ $<$ $\epsilon$
                so for n $\geq$ N,
                d($p_n$,p) $<$ $\frac{1}{n}$ $<$ $\frac{1}{N}$ $<$ $\epsilon$.

                Thus, p = $\lim_{n \rightarrow \infty}$ $p_n$. \\
    \end{enumerate}

\newpage

    























































