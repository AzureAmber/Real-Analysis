\newpage

\section[Day 10: Sequences and Series]{Sequences and Series}

\subsection{Convergent Sequences}

{ \color{blue} Definition 10.1.1: Convergent Sequence } 

    \begin{adjustbox}{minipage=14cm, right, vspace=0.1cm 0cm}
        A sequence \{$x_n$\} in metric space X converge if
        there is a x $\in$ X such that:

        \hspace{1cm}
        For every $\epsilon$ $>$ 0, there is a N $\in$ $\mathbb{Z}$ such that
        for all n $\geq$ N, d($x_n$,x) $<$ $\epsilon$
        
        Then, \{$x_n$\} converges to x: \hspace{1cm}
        $\lim_{n \rightarrow \infty}$ $x_n$ = x

        If \{$x_n$\} does not converge, then it diverges. \\
	\end{adjustbox}

{ \color{purple} Example 10.1.2 }

    \begin{enumerate}[label=(\alph*), leftmargin=2cm, itemsep=0.4em]
        \item Let $x_n$ = $\frac{1}{n}$ in $\mathbb{R}^2$.
        Then, $\lim_{n \rightarrow \infty}$ $x_n$ = 0

            { \color{magenta} \underline{Proof} }

                For $\epsilon$ $>$ 0, there is a $\frac{1}{N}$ $<$ $\epsilon$.
                Then:

                \hspace{1cm}
                d($x_n$,0) = $|x_n - 0|$ = $\frac{1}{n}$
                $<$ $\frac{1}{N}$ $<$ $\epsilon$

        \item Let $x_n$ = $(-1)^n$ + $\frac{1}{n}$ in $\mathbb{R}^2$.
        Then, \{$x_n$\} diverges.

            { \color{magenta} \underline{Proof} }

                $\lim_{n \rightarrow \infty}$ $x_n$
                = $\lim_{n \rightarrow \infty}$ $(-1)^n$
                + $\lim_{n \rightarrow \infty}$ $\frac{1}{n}$
                = $\lim_{n \rightarrow \infty}$ $(-1)^n$

                Since $(-1)^n$ alternates between -1 and 1, then
                \{$x_n$\} diverges. \\
    \end{enumerate}

{ \color{red} Theorem 10.1.3: A convergent sequence is unique }

    \begin{enumerate}[label=(\alph*), leftmargin=2cm, itemsep=0.4em]
        \item \{$p_n$\} converges to p $\in$ X if and only if
        every N$_r(p)$ contains $p_n$ for all, but finitely many n.

            { \color{magenta} \underline{Proof} }

                Suppose $p_n$ $\rightarrow$ p.
                Then for N$_{\epsilon}(p)$, any q $\in$ X such that
                d(q,p) $<$ $\epsilon$ is q $\in$ N$_{\epsilon}(p)$.
                Since $p_n$ $\rightarrow$ p, there is a N such that for
                n $\geq$ N, d($p_n$,p) $<$ $\epsilon$.

                Thus, for n $\geq$ N, $p_n$ $\in$ N$_{\epsilon}(p)$.

                Suppose every N$_r(p)$ contains $p_n$ for all, but finitely
                many n.

                For $\epsilon$ $>$ 0, let N$_{\epsilon}(p)$ be the set of
                all q $\in$ X such that d(p,q) $<$ $\epsilon$.
                Thus, there exists an N such that $p_n$ $\in$ N$_{\epsilon}(p)$
                if n $\geq$ N.

                Thus, d($p_n$,p) $<$ $\epsilon$ so $p_n$ $\rightarrow$ p.


        \item If p,p' $\in$ X and \{$p_n$\} converges to p and p', then
        p = p'.

            { \color{magenta} \underline{Proof} }

                For $\epsilon$ $>$ 0, there exists N,N' such that:

                \hspace{1cm}
                d($p_n$,p) $<$ $\frac{\epsilon}{2}$ for n $\geq$ N
                \hspace{1cm}
                d($p_n$,p') $<$ $\frac{\epsilon}{2}$ for n $\geq$ N'

                Then for n $\geq$ max(N,N'):

                \hspace{1cm}
                d(p,p') $\leq$ d(p,$p_n$) + d($p_n$,p') $<$ $\epsilon$

                Thus, p = p'.

        \item If \{$p_n$\} converges, then \{$p_n$\} is bounded.

            { \color{magenta} \underline{Proof} }

                If \{$p_n$\} $\rightarrow$ p,
                there is a N such that for n $>$ N, d($p_n$,p) $<$ 1.

                Let r = max( 1, d($p_1$,p), ... , d($p_N$,p) ).
                Thus for all n, d($p_n$,p) $\leq$ r.

        \item If E $\subset$ X and p $\in$ E', there is a \{$p_n$\}
        in E such that p = $\lim_{n \rightarrow \infty}$ $p_n$.

            { \color{magenta} \underline{Proof} }

                Since p $\in$ E', then for each n $\in$ $\mathbb{Z}_+$,
                there is a $p_n$ $\in$ E such that d($p_n$,p) $<$ $\frac{1}{n}$.
                For $\epsilon$ $>$ 0, there is a $\frac{1}{N}$ $<$ $\epsilon$
                so for n $\geq$ N,
                d($p_n$,p) $<$ $\frac{1}{n}$ $<$ $\frac{1}{N}$ $<$ $\epsilon$.

                Thus, p = $\lim_{n \rightarrow \infty}$ $p_n$. \\
    \end{enumerate}

\newpage

{ \color{red} Theorem 10.1.4: Arithmetic Operations for Sequences } 

    \begin{adjustbox}{minipage=14cm, right, vspace=0.1cm 0cm}
        Suppose \{$s_n$\},\{$t_n$\} $\in$ $\mathbb{C}$ where
        $\lim_{n \rightarrow \infty}$ $s_n$ = s and
        $\lim_{n \rightarrow \infty}$ $t_n$ = t.
    \end{adjustbox}

	\begin{enumerate}[label=(\alph*), leftmargin=2cm, itemsep=0.1cm]
        \item $\lim_{n \rightarrow \infty}$ $s_n + t_n$ = s + t
        
            { \color{magenta} \underline{Proof} }

                For $\epsilon$ $>$ 0, there exists $N_1$, $N_2$ such that

                \hspace{1cm}
                $|s_n - s|$ $<$ $\frac{\epsilon}{2}$ for n $\geq$ $N_1$
                \hspace{1cm}
                $|t_n - t|$ $<$ $\frac{\epsilon}{2}$ for n $\geq$ $N_2$

                If N = max($N_1$,$N_2$), then for n $\geq$ N:

                \hspace{1cm}
                $|s_n+t_n - s+t|$ $\leq$ $|s_n - s|$ + $|t_n - t|$ $<$ $\epsilon$
        
        \item $\lim_{n \rightarrow \infty}$ $cs_n$ = cs and 
        $\lim_{n \rightarrow \infty}$ $c + s_n$ = c + s

            { \color{magenta} \underline{Proof} }

                For $\epsilon$ $>$ 0, there exists a N such that

                \hspace{1cm}
                $|s_n - s|$ $<$ $\frac{\epsilon}{c}$ for n $\geq$ N

                \hspace{1cm}
                $|cs_n - cs|$ $\leq$ c $\cdot$ $|s_n - s|$ $<$ $\epsilon$
        
        \item $\lim_{n \rightarrow \infty}$ $s_n t_n$ = st
        
            { \color{magenta} \underline{Proof} }

                Note $s_n t_n$ - st
                = ($s_n - s$)($t_n - t$) + t($s_n$ - s) + s($t_n$ - t).

                For $\epsilon$ $>$ 0, there exists $N_1$,$N_2$ such that

                \hspace{1cm}
                $|s_n - s|$ $<$ $\sqrt{\epsilon}$ for n $\geq$ $N_1$
                \hspace{1cm}
                $|t_n - t|$ $<$ $\sqrt{\epsilon}$ for n $\geq$ $N_2$

                If N = max($N_1$,$N_2$), then for n $\geq$ N,
                $|(s_n - s)(t_n - t)|$ $<$ $\epsilon$.

                Thus, $\lim_{n \rightarrow \infty}$ 
                $(s_n - s)(t_n - t)$ = 0.

                \hspace{1cm}
                $\lim_{n \rightarrow \infty}$ $(s_n t_n - st)$
                = $\lim_{n \rightarrow \infty}$
                $(s_n - s)(t_n - t) + t(s_n - s) + s(t_n - t)$

                \hspace{4.4cm}
                = 0 + t $\cdot$ 0 + s $\cdot$ 0 = 0

        \item $\lim_{n \rightarrow \infty}$ $\frac{1}{s_n}$ = $\frac{1}{s}$
        where $s_n, s$ $\not =$ 0

            { \color{magenta} \underline{Proof} }

                Choose m such that $|s_n - s|$ $<$ $\frac{1}{2} |s|$ if n $\geq$ m
                so $|s_n|$ $>$ $\frac{1}{2} |s|$ for n $\geq$ m.

                For $\epsilon$ $>$ 0, there is a N $>$ m such that for n $\geq$ N, 
                $|s_n - s|$ $<$ $\frac{1}{2} |s|^2 \epsilon$.


                Thus, for n $\geq$ N, 
                $|\frac{1}{s_n} - \frac{1}{s}|$
                = $\frac{s_n - s}{s_n s}$
                $<$ $\frac{2}{|s|^2} |s_n - s|$
                $<$ $\epsilon$. \\
    \end{enumerate}

{ \color{red} Theorem 10.1.5: Extension to $\mathbb{R}^k$ }

    \begin{enumerate}[label=(\alph*), leftmargin=2cm, itemsep=0.1cm]
        \item Suppose $x_n$ $\in$ $\mathbb{R}^k$ and $x_n$
        = ($\alpha_{n_1}$, ... , $\alpha_{n_k}$).
        Then \{$x_n$\} converges to x = ($\alpha_1$, ... , $\alpha_k$)
        if and only if $\lim_{n \rightarrow \infty}$ $\alpha_{n_i}$ = $\alpha_i$
        for i $\in$ [1,k].

            { \color{magenta} \underline{Proof} }

                Suppose \{$x_n$\} converges to x = ($\alpha_1$, ... , $\alpha_k$).
                
                Since for any i $\in$ [1,k],
                $|\alpha_{n_i} - \alpha_i|$ $\leq$ $|x_n - x|$ $<$ $\epsilon$.
                Then, $\lim_{n \rightarrow \infty}$ $\alpha_{n_i}$ = $\alpha_i$.

                Suppose $\lim_{n \rightarrow \infty}$ $\alpha_{n_i}$ = $\alpha_i$
                for i $\in$ [1,k].
                
                Then for $\epsilon$ $>$ 0, there is an N such that for n $\geq$ N:

                \hspace{1cm}
                $|\alpha_{n_i} - \alpha_i|$ $<$ $\frac{\epsilon}{\sqrt{k}}$
                for i $\in$ [1,k]

                \hspace{1cm}
                $|x_n - x|$
                = $\sqrt{\sum_{i=1}^k |\alpha_{n_i} - \alpha_i|^2}$
                $<$ $\sqrt{k \cdot (\frac{\epsilon}{\sqrt{k}})^2}$
                = $\epsilon$

        \item Suppose \{$x_n$\},\{$y_n$\} $\in$ $\mathbb{R}^k$ and
        \{$\beta_n$\} $\in$ $\mathbb{R}$ and $x_n$ $\rightarrow$ x,
        $y_n$ $\rightarrow$ y, $\beta_n$ $\rightarrow$ $\beta$.
        
        $\lim_{n \rightarrow \infty}$ $x_n + y_n$ = x+y
        \hspace{0.5cm}
        $\lim_{n \rightarrow \infty}$ $x_n \cdot y_n$ = x$\cdot$y
        \hspace{0.5cm}
        $\lim_{n \rightarrow \infty}$ $\beta_n x_n$ = $\beta$x

            { \color{magenta} \underline{Proof} }

                By part a, then
                $\lim_{n \rightarrow \infty}$ $x_{n_i} + y_{n_i}$
                = $x_i + y_i$ so
                \{$x_n + y_n$\} $\rightarrow$ x+y.

                Also,
                $\lim_{n \rightarrow \infty}$ $\sum_{i=1}^k x_{n_i} y_{n_i}$
                = $\sum_{i=1}^k x_i y_i$ so
                \{$x_n \cdot y_n$\} $\rightarrow$ x$\cdot$y.

                Also,
                $\lim_{n \rightarrow \infty}$ $\beta_i x_{n_i}$
                = $\beta_i x_i$ so
                \{$\beta_n x_n$\} $\rightarrow$ $\beta x$. \\
    \end{enumerate}

\newpage

\subsection{Subsequences}






















































