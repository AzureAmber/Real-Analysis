\newpage

\section[Day 13: Power Series]
{Power Series}

\subsection{ Power Series }

{ \color{blue} Definition 13.1.1: Power Series }

    \begin{adjustbox}{minipage=14cm, right, vspace=0.1cm 0cm}
        For a sequence \{$c_n$\} $\in$ $\mathbb{C}$, the series
        $\sum_{n=0}^{\infty}$ $c_n z^n$
        is a power series.

        $c_n$ are the coefficients and z $\in$ $\mathbb{C}$.
    \end{adjustbox}

    \vspace{0.5cm}

{ \color{red} Theorem 13.1.2: Radius of Convergence }

    \begin{adjustbox}{minipage=14cm, right, vspace=0.1cm 0cm}
        For power series $\sum$ $c_n z^n$, let
        $\alpha$ = $\lim_{n \rightarrow \infty}$ sup($\sqrt[n]{|c_n|}$)
        and R = $\frac{1}{\alpha}$.

        Then $\sum$ $c_n z^n$ converges if $|z|$ $<$ R and
        diverges if $|z|$ $>$ R.
    \end{adjustbox}

{ \color{magenta} \underline{Proof} }

    \fbox{
    \begin{minipage}{15cm}
        Let $a_n$ = $c_n z^n$. Using the {\color{red} root test},

        \hspace{1cm}
        $\lim_{n \rightarrow \infty}$ sup($\sqrt[n]{|a_n|}$)
        = $\lim_{n \rightarrow \infty}$ sup($\sqrt[n]{|c_n z^n|}$)

        \hspace{4.7cm}
        = $|z|$ $\lim_{n \rightarrow \infty}$ sup($\sqrt[n]{|c_n|}$)
        = $\frac{|z|}{R}$

        Thus, $\sum$ $c_n z^n$ converges if $\frac{|z|}{R}$ $<$ 1
        and diverges if $\frac{|z|}{R}$ $>$ 1
    \end{minipage} }





\subsection{ Summation By Parts }

{ \color{red} Theorem 13.2.1: Summation By Parts }

    \begin{adjustbox}{minipage=14cm, right, vspace=0.1cm 0cm}
        For sequences \{$a_n$\}, \{$b_n$\}, let $A_n$ = $\sum_{k=0}^n$ $a_k$.
        Then for 0 $\leq$ p $\leq$ q:

        \hspace{1cm}
        $\sum_{n = p}^q$ $a_n b_n$
        = ($\sum_{n = p}^{q-1}$ $A_n (b_n - b_{n+1})$)
        + $A_q b_q$ - $A_{p-1} b_p$
    \end{adjustbox}

{ \color{magenta} \underline{Proof} }

    \fbox{
    \begin{minipage}{15cm}
        $\sum_{n = p}^q$ $a_n b_n$
        = $\sum_{n = p}^q$ $(A_n - A_{n-1}) b_n$

        \hspace{2cm}
        = $\sum_{n = p}^q$ $A_n b_n$ - $\sum_{n = p}^q$ $A_{n-1} b_n$
        = $\sum_{n = p}^q$ $A_n b_n$ - $\sum_{n = p-1}^{q-1}$ $A_{n} b_{n+1}$

        \hspace{2cm}
        = $\sum_{n = p}^{q-1}$ $A_n b_n$ - $\sum_{n = p}^{q-1}$ $A_{n} b_{n+1}$
        + $A_q b_q$ - $A_{p-1} b_p$

        \hspace{2cm}
        = ($\sum_{n = p}^{q-1}$ $A_n (b_n - b_{n+1})$)
        + $A_q b_q$ - $A_{p-1} b_p$
    \end{minipage} }

    \vspace{0.5cm}

{ \color{red} Theorem 13.2.2: Conditions for convergent $\sum$ $a_n b_n$ }

    \begin{adjustbox}{minipage=14cm, right, vspace=0.1cm 0cm}
        Suppose for \{$a_n$\}, \{$b_n$\}:

        \begin{itemize}[leftmargin=1cm, itemsep=0.1cm]
            \item partial sums $A_n$ of $\sum$ $a_n$ form a bounded sequence
            
            \item $b_i$ $\geq$ $b_{i+1}$
            
            \item $\lim_{n \rightarrow \infty}$ $b_n$ = 0
        \end{itemize}

        Then $\sum$ $a_n b_n$ converges.
    \end{adjustbox}

{ \color{magenta} \underline{Proof} }

    \fbox{
    \begin{minipage}{15cm}
        Since \{$A_n$\} is bounded, $|A_n|$ $\leq$ M for all n.

        Since \{$b_n$\} is monotonically decreasing and
        $\lim_{n \rightarrow \infty}$ $b_n$ = 0, then
        for $\epsilon$ $>$ 0, there is a N such that
        $b_N$ $\leq$ $\frac{\epsilon}{2M}$.
        Then for N $\leq$ p $\leq$ q:

        \hspace{1cm}
        $| \sum_{n=p}^q a_n b_n |$
        = ($| \sum_{n = p}^{q-1}$ $A_n (b_n - b_{n+1})$)
        + $A_q b_q$ - $A_{p-1} b_p |$

        \hspace{3.3cm}
        $\leq$ M $| \sum_{n = p}^{q-1} (b_n - b_{n+1}) + b_q + b_p |$
        = 2M$b_p$
        $\leq$ 2M$b_N$
        $\leq$ $\epsilon$
    \end{minipage} }

\newpage

{ \color{orange} Corollary 13.2.3: Convergent Series of Alternating Sequences }

    \begin{adjustbox}{minipage=14cm, right, vspace=0.1cm 0cm}
        Suppose for \{$c_n$\}:

        \begin{itemize}[leftmargin=1cm, itemsep=0.1cm]
            \item $|c_i|$ $\geq$ $|c_{i+1}|$
            
            \item $c_{2i-1}$ $\geq$ 0 and $c_{2i}$ $\leq$ 0
            
            \item $\lim_{n \rightarrow \infty}$ $c_n$ = 0
        \end{itemize}

        Then $\sum$ $c_n$ converges.
    \end{adjustbox}

{ \color{magenta} \underline{Proof} }

    \fbox{
    \begin{minipage}{15cm}
        From {\color{red} theorem 13.2.2}, let
        $a_n$ = $(-1)^{n+1}$ and $b_n$ = $|c_n|$.
    \end{minipage} }

    \vspace{0.5cm}

{ \color{orange} Corollary 13.2.4: Convergent Power Series }

    \begin{adjustbox}{minipage=14cm, right, vspace=0.1cm 0cm}
        Suppose for \{$c_n$\}:

        \begin{itemize}[leftmargin=1cm, itemsep=0.1cm]
            \item Radius of convergence of $\sum$ $c_n z^n$ is 1
            
            \item $c_i$ $\geq$ $c_{i+1}$
            
            \item $\lim_{n \rightarrow \infty}$ $c_n$ = 0
        \end{itemize}

        Then $\sum$ $c_n z^n$ converges at every point where $|z|$ = 1
        except possibly z = 1.
    \end{adjustbox}

{ \color{magenta} \underline{Proof} }

    \fbox{
    \begin{minipage}{15cm}
        From {\color{red} theorem 13.2.2}, let
        $a_n$ = $z^n$ and $b_n$ = $c_n$.

        $A_n$ of $\sum$ $a_n$ form a bounded sequence since
        $| A_n |$
        = $| \sum_0^n z^n |$
        = $| \frac{1 - z^{n+1}}{1 - z} |$
        $\leq$ $\frac{2}{|1 - z|}$.
    \end{minipage} }





\subsection{ Absolute Convergence }

{ \color{blue} Definition 13.3.1: Absolute Convergence }

    \begin{adjustbox}{minipage=14cm, right, vspace=0.1cm 0cm}
        $\sum$ $a_n$ converges absolutely if $\sum$ $|a_n|$ converges.

        If $\sum$ $a_n$ converges, but $\sum$ $|a_n|$ diverges,
        then $\sum$ $a_n$ converges non-absolutely.
    \end{adjustbox}

    \vspace{0.5cm}

{ \color{red} Theorem 13.3.2: Absolute convergence $\rightarrow$ convergence }

    \begin{adjustbox}{minipage=14cm, right, vspace=0.1cm 0cm}
        If $\sum$ $a_n$ converges absolutely, then $\sum$ $a_n$ converges.
    \end{adjustbox}

{ \color{magenta} \underline{Proof} }

    \fbox{
    \begin{minipage}{15cm}
        Since $\sum$ $a_n$ converges absolutely, then for every
        $\epsilon$ $>$ 0, there is an integer N such that for m $\geq$ n $\geq$ N,
        $|\sum_{k=n}^m |a_k||$ = $\sum_{k=n}^m |a_k|$ $\leq$ $\epsilon$.

        Thus, 
        $|\sum_{k=n}^m a_k|$
        $\leq$ $\sum_{k=n}^m |a_k|$
        $\leq$ $\epsilon$
        so $\sum$ $a_n$ converges.
    \end{minipage} }





\subsection{ Addition \& Multiplication of Series }

{ \color{red} Theorem 13.4.1: Addition and Scalar Multiplication }

    \begin{adjustbox}{minipage=14cm, right, vspace=0.1cm 0cm}
        If $\sum a_n$ = A and $\sum b_n$ = B, then
        $\sum (a_n + b_n)$ = A + B and $\sum ca_n$ = cA.
    \end{adjustbox}

{ \color{magenta} \underline{Proof} }

    \fbox{
    \begin{minipage}{15cm}
        Let $A_n$ = $\sum_{k=0}^n a_k$ and $B_n$ = $\sum_{k=0}^n b_k$.

        Then $A_n + B_n$ = $\sum_{k=0}^n a_k + b_k$
        so $\lim_{n \rightarrow \infty}$ $A_n + B_n$
        = A + B.

        Then $\lim_{n \rightarrow \infty}$ $cA_n$
        = $\underbrace{A + ... + A}_c$ = cA
    \end{minipage} }

    \vspace{0.5cm}

{ \color{blue} Definition 13.4.2: Cauchy Product }

    \begin{adjustbox}{minipage=14cm, right, vspace=0.1cm 0cm}
        For $\sum a_n$ and $\sum b_n$, let
        $c_n$ = $\sum_{k=0}^n a_k b_{n-k}$
        and the product as $\sum c_n$.

        \vspace{0.2cm}

        $\sum_{n=0}^{\infty} a_n z^n$ $\sum_{n=0}^{\infty} b_n z^n$
        = $(a_0 + a_1z + a_2z^2 + ... + a_nz^n)$
        $(b_0 + b_1z + b_2z^2 + ... + b_nz^n)$

        \hspace{3.9cm}
        = $a_0b_0 + (a_0b_1 + a_1b_0)z + (a_0b_2 + a_1b_1 + a_2b_0)z^2 + ... $
    \end{adjustbox}

\newpage

{ \color{red} Theorem 13.4.3: Conditions $\sum$ $c_n$ = AB }

    \hspace{1cm}
    Suppose

    \begin{enumerate}[label=(\alph*), leftmargin=3cm, itemsep=0.1cm]
        \item $\sum_{n=0}^{\infty}$ $a_n$ converges absolutely
        
        \item $\sum_{n=0}^{\infty}$ $a_n$ = A
        
        \item $\sum_{n=0}^{\infty}$ $b_n$ = B
        
        \item $c_n$ = $\sum_{k=0}^{\infty}$ $a_k b_{n-k}$ 
    \end{enumerate}

    \hspace{1cm}
    Then $\sum_{n=0}^{\infty}$ $c_n$ = AB.

{ \color{magenta} \underline{Proof} }

    \fbox{
    \begin{minipage}{15cm}
        Let $A_n$ = $\sum_{k=0}^{n}$ $a_k$, $B_n$ = $\sum_{k=0}^{n}$ $b_k$,
        $C_n$ = $\sum_{k=0}^{n}$ $c_k$, and $\beta_n$ = $B_n$ - B.

        \hspace{1cm}
        $C_n$
        = $a_0b_0 + (a_0b_1 + a_1b_0) + ... + (a_0b_n + ... + a_nb_0)$
        
        \hspace{1.6cm}
        = $a_0B_n + a_1B_{n-1} + ... + a_nB_0$

        \hspace{1.6cm}
        = $a_0(B + \beta_n) + a_1(B + \beta_{n-1}) + ... + a_n(B + \beta_{0})$

        \hspace{1.6cm}
        = $A_nB + a_0\beta_n + a_1\beta_{n-1} + ... + a_n\beta_0$

        Let $\gamma_n$ = $a_0\beta_n + a_1\beta_{n-1} + ... + a_n\beta_0$
        so $C_n$ = $A_nB + \gamma_n$.

        Since $a_n$ converges absolutely, then
        $\sum_{n=0}^{\infty}$ $|a_n|$ = $\alpha$.

        Since $\sum_{n=0}^{\infty}$ $b_n$ = B, then $\beta_n$ $\rightarrow$ 0.
        
        Then for $\epsilon$ $>$ 0, there is a N such that
        $|\beta_n|$ $\leq$ $\frac{\epsilon}{\alpha}$ for n $\geq$ N.

        \hspace{1cm}
        $|\gamma_n|$
        $\leq$ $|\beta_0a_n + ... + \beta_Na_{n-N}|$
        + $|\beta_{N+1}a_{n-N-1} + ... + \beta_na_{0}|$

        \hspace{1.75cm}
        $\leq$ $|\beta_0a_n + ... + \beta_Na_{n-N}|$
        + $|a_{n-N-1} + ... + a_{0}|$$\frac{\epsilon}{\alpha}$

        \hspace{1.75cm}
        $\leq$ $|\beta_0a_n + ... + \beta_Na_{n-N}|$
        + $\alpha \frac{\epsilon}{\alpha}$

        Thus, with a fixed N, since $a_n$ $\rightarrow$ 0, then
        $\lim_{n \rightarrow \infty}$ $|\gamma_n|$ $\leq$ $\epsilon$
        so $\lim_{n \rightarrow \infty}$ $\gamma_n$ = 0.

        Thus, $\lim_{n \rightarrow \infty}$ $C_n$
        = $\lim_{n \rightarrow \infty}$ $A_nB + \gamma_n$ = AB.
    \end{minipage} }

    \vspace{0.5cm}

{ \color{red} Theorem 13.4.4: By Cauchy Product, $\sum c_n$ = C implies C = AB }

    \begin{adjustbox}{minipage=14cm, right, vspace=0.1cm 0cm}
        If $\sum a_n$ = A, $\sum b_n$ = B, $\sum c_n$ = C
        where $c_n$ = $a_0b_n + ... + a_nb_0$, then C = AB.
    \end{adjustbox}





\subsection{ Rearrangements }

{ \color{blue} Definition 13.5.1:  Rearrangements }

    \begin{adjustbox}{minipage=14cm, right, vspace=0.1cm 0cm}
        Let $a_n'$ = $a_{k_n}$.
        Then $\sum a_n'$ is a rearrangement of $\sum a_n$.
    \end{adjustbox}

    \vspace{0.5cm}

{ \color{red} Theorem 13.5.2: Rearrangements can converge or diverge }

    \begin{adjustbox}{minipage=14cm, right, vspace=0.1cm 0cm}
        Let $\sum a_n$ $\in$ $\mathbb{R}$ converge non-absolutely.
        Suppose $-\infty \leq \alpha \leq \beta \leq \infty$.

        Then there exists a rearrangement $\sum a_n'$ with partial
        sums $s_n'$ such that:

        \hspace{1cm}
        $\lim_{n \rightarrow \infty}$ inf($s_n'$) = $\alpha$
        \hspace{1cm}
        $\lim_{n \rightarrow \infty}$ sup($s_n'$) = $\beta$
    \end{adjustbox}

{ \color{magenta} \underline{Proof} }

    \fbox{
    \begin{minipage}{15cm}
        Let $p_n$ = $\frac{|a_n| + a_n}{2}$ and
        $q_n$ = $\frac{|a_n| - a_n}{2}$.
        Since $\sum |a_n|$ diverge, then $\sum p_n$ and $\sum q_n$ diverges.

        Let $P_1, P_2, P_3, ...$ be the nonnegative terms of $\sum a_n$
        in order and $Q_1, Q_2, Q_3, ...$ be the absolute values of the
        negative terms of $\sum a_n$ in order.
        Thus, $\sum P_n$ and $\sum Q_n$ differ from $\sum p_n$ and $\sum q_n$
        only by the zero terms and thus, are divergent.

        Choose real-valued sequences \{$\alpha_n$\} $\rightarrow$ $\alpha$,
        \{$\beta_n$\} $\rightarrow$ $\beta$ such that
        $\alpha_n < \beta_n$ and $\beta_1 > 0$.

        Let $m_1$,$k_1$ be the smallest integers such that:

        \hspace{1cm}
        $P_1 + ... + P_{m_1} > \beta_1$
        \hspace{1cm}
        $P_1 + ... + P_{m_1} - Q_1 - ... - Q_{k_1} < \alpha_1$

        Let $m_2$,$k_2$ be the smallest integers such that:

        \hspace{1cm}
        $P_1 + ... + P_{m_1} - Q_1 - ... - Q_{k_1}
        + P_{m_1 + 1} + ... + P_{m_2} < \beta_2$

        \hspace{1cm}
        $P_1 + ... + P_{m_1} - Q_1 - ... - Q_{k_1}
        + P_{m_1 + 1} + ... + P_{m_2}
        - Q_{k_1 + 1} - ... - Q_{k_2} < \alpha_2$

        Continuing such a process, then
        $\lim_{n \rightarrow \infty}$ inf($s_n'$) = $\alpha$
        and
        $\lim_{n \rightarrow \infty}$ sup($s_n'$) = $\beta$.
    \end{minipage} }

\newpage

{ \color{red} Theorem 13.5.3: Absolute rearrangements converges uniquely }

    \begin{adjustbox}{minipage=14cm, right, vspace=0.1cm 0cm}
        If $\sum a_n$ $\in$ $\mathbb{C}$ converges absolutely,
        then every rearrangement of $\sum a_n$ converges to the same sum.
    \end{adjustbox}

{ \color{magenta} \underline{Proof} }

    \fbox{
    \begin{minipage}{15cm}
        Let $\sum a_n'$ be a rearrangement with partial sums $s_n'$.

        For $\epsilon$ $>$ 0, there is a N such that for m $\geq$ n $\geq$ N,
        $\sum_{i=n}^m$ $|a_i|$ $\leq$ $\epsilon$.

        Let p be the maximum index of \{$a_1, a_2, ... , a_N$\} in $a_n'$
        and $a_n$.

        Since if n $>$ p, then $a_1, a_2, ... , a_N$ will cancel in
        $s_n - s_n'$ and thus,
        $|s_n - s_n'|$ $\leq$ $\epsilon$.

        Thus, every \{$s_n'$\} converges to \{$s_n$\}.
    \end{minipage} }
    



