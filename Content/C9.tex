\newpage
\section[Day 9: Perfect \& Connected Sets]{Perfect and Connected Sets}





\subsection{ Perfect Sets }

{ \color{blue} Definition 9.1.1: Perfect Set } 

	\begin{adjustbox}{minipage=14cm, right, vspace=0.1cm 0cm}
		E $\subset$ X is perfect if E is closed and if every p $\in$ E
		is p $\in$ E'. \\
	\end{adjustbox}

{ \color{red} Theorem 9.1.2: Perfect sets are uncountable } 

	\begin{adjustbox}{minipage=14cm, right, vspace=0.1cm 0cm}
		Let P be a nonempty perfect set in $\mathbb{R}^k$.
		Then, P is uncountable.
	\end{adjustbox}

{ \color{magenta} \underline{Proof} } 

	Since P has limit points, then by {\color{red} theorem 7.1.4},
	P is infinite.

	Suppose P is countable. Then let $x_1, x_2, ...$ $\in$ P.

	Let $V_i$ be a neighborhood of $x_i$ where y $\in$ $V_i$ for any
	y $\in$ $\mathbb{R}^k$ such that $|y-x_i|$ $<$ r.

	Thus, the $\overline{V_i}$ is the set of all y $\in$ $\mathbb{R}^k$
	such that $|y-x_i|$ $\leq$ r.

	Since every $x_i$ are limit points, then any $V_i$ $\cap_{}^{}$ P
	is not empty where there is a $V_{i+1}$

	\begin{enumerate}[label=(\alph*), leftmargin=1cm, itemsep=0.1cm]
		\item $\overline{V_{i+1}}$ $\subset$ $V_i$
		\item $x_i$ $\not \in$ $\overline{V_{i+1}}$
		\item $V_{i+1}$ $\cap_{}^{}$ P is nonempty
	\end{enumerate}

	Let $K_i$ = $\overline{V_i}$ $\cap_{}^{}$ P.
	Since $\overline{V_i}$ is closed and bounded, then by
	{\color{red} theorem 8.3.11}, $\overline{V_i}$ is compact.
	Since $x_i$ $\not \in$ $K_{i+1}$, then no $x_i$ $\in$ P is
	$x_i$ $\in$ $\cap_{}^{}$ $K_i$.
	Since $K_n$ $\subset$ P, then $\cap_{}^{}$ $K_i$ is nonempty
	which contradicts {\color{orange} corollary 8.3.8 } since
	each $K_i$ is empty and $K_{i+1}$ $\subset$ $K_i$. \\

{ \color{orange} Corollary 9.1.3: $\mathbb{R}$ is not countable } 

	\begin{adjustbox}{minipage=14cm, right, vspace=0.1cm 0cm}
		Every interval [a,b] and (a,b) is uncountable.
		Thus, $\mathbb{R}$ is countable. \\
	\end{adjustbox}

{ \color{blue} Definition 9.1.4: Cantor Sets } 

	\begin{adjustbox}{minipage=14cm, right, vspace=0.1cm 0cm}
		There exists perfect segments in $\mathbb{R}^1$ which contain no segement.

		Let $E_0$ = [0,1].

		For $E_1$, remove ($\frac{1}{3}$,$\frac{2}{3}$).
		Thus, $E_1$ = [0,$\frac{1}{3}$] $\cup_{}^{}$ [$\frac{2}{3}$,1].

		For $E_2$, remove ($\frac{1}{9}$,$\frac{2}{9}$) and ($\frac{7}{9}$,$\frac{8}{9}$).
		Thus, $E_2$ = [0,$\frac{1}{9}$] $\cup_{}^{}$ [$\frac{2}{9}$,$\frac{3}{9}$] $\cup_{}^{}$
		[$\frac{6}{9}$,$\frac{7}{9}$] $\cup_{}^{}$ [$\frac{8}{9}$,1].

		Continuing such a sequence, the set of compact sets $E_n$ are such that:

		\begin{enumerate}[label=(\alph*), leftmargin=2cm, itemsep=0.1cm]
			\item $E_{n+1}$ $\subset$ $E_n$
			\item $E_n$ is the union of $2^n$ intervals each of length $3^{-n}$.
		\end{enumerate}

		P = $\cap_{}^{}$ $E_n$ is called the Cantor set.
		P is compact and nonempty.
	\end{adjustbox}


\subsection{ Connected Sets }
























































