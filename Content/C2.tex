\newpage
\section[Day 2:]{Day 2}

\subsection{Greatest Upper Bound Property} 
{ \color{blue} Theorem }

\begin{adjustbox}{minipage=14cm, right}
	

If If an ordered set has the least upper bound property, then it has the greatest upper bound property. \\

Let S be a ordered set with the least upper bound property. Let non-empty B $\subset$ Sand bounded below.
Let L be the set of all lower bounds of B. Then $\alpha$ = sup(L) exists and $\alpha \in$ B. \\
\end{adjustbox}

{ \color{magenta} Proof }

L is non-empty since B is bounded from below and $\gamma \in$ L.

Thus, by the least upper bound property of S, $\alpha$ = sup(L) exists.

We claim that $\alpha$ = inf(B).

For any x $< \alpha$, then x is not an upper bound for L so x $\in$ L.

For any x $> \alpha$, then x is an upper bound for L so x not in L and thus, x $\in$ B.

\subsection{Fields}

Addition Axioms
\begin{itemize}[leftmargin=2cm]
	\item x,y $\in$ F, then x+y $\in$ F
	\item Addition is commutative
	\item Addition is associative
	\item x+0 = x
\end{itemize}


Multiplicative xioms
\begin{itemize}[leftmargin=2cm]
	\item xy
	\item xy = yx
	\item (xy)z = x(yx)
	\item 1/x
\end{itemize}

Distributive Law

\qquad x(y+z) = xy + xz hold for all x,y,z $\in$ F. \\

*** Remember to insert all the propostitions


\subsection{Ordered Fields}

\qquad An ordered field F is a field F which is also an ordered set for all x,y,z $\in$ F.

\begin{itemize}[leftmargin=2cm]
	\item If y $<$ z, then y+x $<$ z+x
	\item If x,y $>$ 0, then xy $>$ 0
\end{itemize}
\qquad *** If x $>$ 0, then x is positive *** \\

{ \color{blue} Definition 2.3.1} 

\qquad $ \mathbb{Q} $,$ \mathbb{R} $ are ordered fields. $ \mathbb{C} $ is not a ordered field. \\

{ \color{blue} Definition 2.3.2} 

\qquad Let F be an ordered field. For all x,y,z $\in$ F.
\begin{itemize}[leftmargin=2cm]
	\item If x $>$ 0, -x $<$ 0 and vice versa
	\item If x $>$ 0 and y $<$ z, then xy < xz
	\item If x $ < $ 0 and y $ < $ z, then xy $ > $ xz
	\item If x $\neq$ 0, $x^2 > $ 0
	\item If 0 $< x < y$, then 0 $< 1/y < 1/x$
	
	\qquad Proof for A \\
	If x $>$ 0, then x+(-x) $>$ 0+(-x) so 0 > (-x)
\end{itemize}

{\color{blue} Theorem 2.3.3: \mathbb{R} is a ordered field with $<$ }

\qquad There exists a ordered field $ \mathbb{R} $ with the least upper bound property.

\qquad Also, $ \mathbb{Q} $  $\subset$ $ \mathbb{R} $.

\qquad $ \mathbb{R} $ is unique ordered field with least upper bound property. \\

{\color{blue} Theorem 2.3.4}

For all x,y $\in$ $ \mathbb{R} $:
\begin{itemize}[leftmargin=2cm]
	\item Archimedean Property: If x $>$ 0, there is n $\in$ $ \mathbb{Z} $ such that nx $>$ y.
	
		Proof

		\qquad Fix x $>$ 0. Suppose there is a y such that the property fails.

		\qquad Let A = \{ nx: n = 1, 2, 3... \}.

		\qquad Then, A is nonempty and bounded from above by y.

		\qquad Then by the least upper bound property by $ \mathbb{R} $ , then $\alpha$ = sup(A) exists in $ \mathbb{R} $ .

		\qquad Since x $>$ 0, then -x $<$ 0 so $\alpha - x < \alpha-0 = \alpha$.

		\qquad So $\alpha-x$ is an upper bound of A. So there is a mx $\in$ A such that mx $> \alpha-x$

		\qquad But then $\alpha < (m+1)x$ so (m+1)x $\in$ A which contradicts $\alpha$ is an upper bound for A.

	\item $ \mathbb{Q} $  is dense in $ \mathbb{R} $: If x $<$ y, there is a p $\in$ $ \mathbb{Q} $
		such that x $<$ p $<$ y.

		Proof

		n(y-x) $>$ 1
		
		ny $>$ nx+1 $>$ nx
		
		1 $>$ nx

		By the well-ordering principle, there is a smallest m of positive integers such that m $>$ nx

		Then, m $>$ nx $\geq$ m-1 and nx+1 $\geq$ m $>$ nx

		By ny $>$ nx+1 $\geq$ m $>$ nx.

		SO y $>$ m/n $>$ x.

\end{itemize}




