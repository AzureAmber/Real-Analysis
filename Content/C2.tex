\newpage

\section[Day 2: Complex Field \& Euclidean Spaces]
{ Roots, Complex Field, \& Euclidean Spaces }

\subsection{ nth Root }

	\begin{ltheorem}{nth Root}{1.5cm}
		\item If 0 $<$ t $\leq$ 1, then t$^\text{n}$ $ \leq $ t
		
			\begin{proof}[15.5cm]
				Since t $>$ 0 and t $\leq$ 1, then t$^\text{2}$ $\leq$ t.
				Since t$^\text2{}$ $\leq$ t, then t$^\text{3}$ $\leq$ t$^\text{2}$
				so t$^\text{3}$ $\leq$ t$^\text{2}$ $\leq$ t.

				Applying the process n times, then t$^\text{n}$ $\leq$ t.
			\end{proof}

		\item If t $\geq$ 1, then t$^\text{n}$ $ \geq $ t

			\begin{proof}[15.5cm]
				Since 0 $<$ 1 $\leq$ t, then t $\leq$ t$^\text{2}$.
				Since t $\leq$ t$^\text{2}$, then t$^\text{2}$ $\leq$ t$^\text{3}$
				so t $\leq$ t$^\text{2}$ $\leq$ t$^\text{3}$.

				Applying the process n times, t $\leq$ t$^\text{n}$.
			\end{proof}

		\item If 0 $<$ s $<$ t, then s$^\text{n}$ $<$ t$^\text{n}$

			\begin{proof}[15.5cm]
				$\underbrace{\text{s} \cdot \text{s} \cdot
								... \cdot \text{s}}_\text{n}$
				$<$ t $\cdot$ s $\cdot$ ... $\cdot$ s
				$<$ t $\cdot$ t $\cdot$ ... $\cdot$ s $<$ ... $<$
				$\underbrace{\text{t} \cdot ... \cdot \text{t}}_\text{n}$
			\end{proof}
	\end{ltheorem}

	\vspace{0.5cm}



	\begin{wtheorem}{y$^\text{n}$ = x has a unique y}{16cm}
		Fix n $\in$ $\mathbb{Z}_+$. For every x $>$ 0, there exists a unique y
		$\in$ $ \mathbb{R} $ such that y$^\text{n}$ = x.

		Also, such a y is written as y = $\sqrt[n]{\text{x}}$
		= x$^{\frac{1}{\text{n}}}$.		
	\end{wtheorem}
	
	\begin{proof}
		{ \color{lgreen} Uniqueness: }

		y is unique since if y$_{1}$ $<$ y$_{2}$, then
		x = y$_{1}^\text{n}$ $<$ y$_{2}^\text{n}$ $\neq$ x.

		{ \color{lgreen} Existence: }

		Let set A = \{ t $>$ 0 : t$^\text{n}$ $<$ x \}.

		A $\not =$ $\emptyset$ since let t$_{1}$ = $\frac{x}{x+1}$ $<$ 1 so t$_1$ $<$ x
		and thus, 0 $<$ t$_{1}$$^\text{n}$ $<$ t$_{1}$ $<$ x so t$_{1}$ $\in$ A.

		A is bounded above since if t $ \geq $ x+1, then
		t $>$ 1 so t$^\text{n}$ $ \geq $ t $ \geq $ x+1 $>$ x so t $\not \in$ A.

		So x+1 is an upper bound of A.

		Thus by the least upper bound property, y = sup(A) exists.

		For y$^\text{n}$ = x, show y$^\text{n}$ $<$ x and
		y$^\text{n}$ $>$ x cannot hold true.

		***(Not an upper bound of A if $<$ and not a least upper bound of A if $>$)***

		For 0 $<$ $\alpha$ $<$ $\beta$:

		\hspace{1cm}
		$\beta ^\text{n}$ - $\alpha^\text{n}$
		= ($\beta$ - $\alpha$) ($\underbrace{\beta^{n-1}
		+ \beta^{n-2}\alpha^1 + ... + \alpha^{n-1}}
		_{\beta^{n-1} \hspace{0.7cm} <\beta^{n-1} \hspace{1.3cm} <\beta^{n-1}}$)
		$<$ ($\beta - \alpha$)n$\beta$$^\text{n-1}$ 

		Suppose y$^\text{n}$ $<$ x. Pick 0 $<$ h $<$ 1 and
		h $<$ $\frac{x-y^n}{n(y+1)^{n-1}}$.

		\hspace{0.5cm}
		From inequality, let $\beta$ = y+h and $\alpha$ = y.

		\hspace{1cm}
		(y+h)$^\text{n}$ - y$^\text{n}$ $<$ hn(y+h)$^\text{n-1}$
		$<$ hn(y+1)$^\text{n-1}$ $<$ x - y$^\text{n}$

		\begin{adjustbox}{minipage=16cm, right, vspace=0.1cm 0cm}
			Thus, (y+h)$^\text{n}$ $<$ x so y+h $\in$ A and thus, not an upper bound of A
			which is a contradiction since y = sup(A).
		\end{adjustbox}

		Suppose y$^\text{n}$ $>$ x.
		Pick 0 $<$ k = $\frac{y^n - x}{ny^{n-1}}$ $<$ $\frac{y^n}{ny^{n-1}}$
		= $\frac{1}{n}y$ $<$ y.

		\hspace{0.5cm}
		Consider t $ \geq $  y-k, then:
		y$^\text{n}$ - t$^\text{n}$ $ \leq $  y$^\text{n}$ - (y-k)$^\text{n}$ $<$
		kny$^\text{n-1}$ = y$^\text{n}$ - x

		\hspace{0.5cm}
		Thus, t$^\text{n}$ $>$ x so t $\not \in$ A.
		Then, y-k is an upper bound of A which contradicts y = sup(A).

		Since y$^\text{n}$ $<$ x and y$^\text{n}$ $>$ x, then y$^\text{n}$ = x.
	\end{proof}

	\newpage



	\begin{corollary}{n-th root of product = product of n-th root}{16cm}
		If a,b $>$ 0 and n $\in$ $\mathbb{Z}_+$, then
		(ab)$^{\frac{1}{n}}$ = a$^{\frac{1}{n}}$b$^{\frac{1}{n}}$
	\end{corollary}
	
	\begin{proof}
		Let A = a$^{\frac{1}{n}}$ , B = b$^{\frac{1}{n}}$.
		By {\color{red} theorem 2.1.2}, since A is a root for
		$y_1^n$ = a, then A$^n$ = a.

		Similarly, B is a solution of $y_2^n$ = b so B$^n$ = b. Thus:

		\hspace{1cm} ab = $A^n$$B^n$ = $A_1A_2...A_nB_1B_2...B_n$

		\hspace{1.6cm} = $A_1A_2...B_1A_nB_2...B_n$
		= ... = $A_1B_1A_2...A_{n-1}A_nB_2...B_n$

		\hspace{1.6cm} = ... = $A_1B_1A_2B_2...A_nB_n$ = $(AB)^n$

		Then again by {\color{red} theorem 2.1.2}, there is a unique
		(ab)$^{\frac{1}{n}}$ = AB = a$^{\frac{1}{n}}$b$^{\frac{1}{n}}$.		
	\end{proof}

	\vspace{0.5cm}




	
\subsection{ Decimals }

	\begin{definition}{Decimals}{16cm}
		Let n$_0$ be the largest integer such that n$_0$ $\leq$ x for
		x $>$ 0 $\in$ $\mathbb{R}$.

		Then let n$_k$ be the largest integer such that
		d$_k$ = n$_0$ + $\frac{n_1}{10}$ + ... + $\frac{n_k}{10^k}$ $\leq$ x

		Let E be the set of d$_k$ for k = 0, 1, ... $\infty$.
		Then, {\color{lblue} decimal} x = sup(E).	
	\end{definition}

	\vspace{0.5cm}





\subsection{ Extended Reals }

	\begin{definition}{Extended Reals}{16cm}
		The {\color{lblue} extended real number system} consist of $\mathbb{R}$
		and $\pm$$\infty$ such that:

		\hspace{0.5cm}
		-$\infty$ $<$ x $<$ $\infty$ \hspace{1cm} for every x $\in$ $\mathbb{R}$

		with the properties:
		\begin{itemize}[leftmargin=1cm, itemsep=0.1cm]
			\item x $\pm$ $\infty$ = $\pm$$\infty$
		
			\item x / $\pm$$\infty$ = 0

			\item If x $>$ 0, then x($\pm$$\infty$) = $\pm$$\infty$.
				If x $<$ 0, then x($\pm$$\infty$) = $\mp$$\infty$
		\end{itemize}	
	\end{definition}

	\vspace{0.5cm}





\subsection{ Complex Numbers }

	\begin{definition}{Complex Number}{16cm}
		A complex number is an ordered pair (a,b) where a,b $\in$ $ \mathbb{R} $.
		For x,y $\in$ $\mathbb{C}$

		\begin{itemize}[leftmargin=1cm, itemsep=0.1cm]
			\item x + y = (a,b) + (c,d) = (a + c , b + d)
			\item xy = (a,b) (c,d) = (ac - bd , ad + bc)
			\item $\frac{1}{\text{x}}$  = (a$^2$ + b$^2$)(a,-b)
		\end{itemize}

		Thus, the axioms form a field where (0,0) = 0 and (1,0) = 1 and (0,1) = i.
	\end{definition}

	\vspace{0.5cm}



	\begin{wtheorem}{Imaginary i and Form a + bi}{16cm}
		Let i = (0,1). Then:
		
		\hspace{0.5cm}
		i$^2$ = -1
		\hspace{1cm}
		(a,b) = a + bi
	\end{wtheorem}

	\begin{proof}
		i$^\text{2}$ = (0,1)(0,1) = (0-1,0+0) = (-1,0) = -1

		(a,b) = (a,0) +(0,b) = (a,0) + (b,0)(0,1) = a + bi
	\end{proof}
	
	\newpage



	\begin{definition}{Conjugate}{16cm}
		Let conjugate: $\bar{z}$ = a - bi where Re(z) = a , Im(z) = b.

		Let z = (a,b) and w = (c,d):
	\end{definition}

	\begin{enumerate}[label=(\alph*), leftmargin=2cm, itemsep=0.1cm]
		\item $\overline{z+w}$ = $\overline{z}$ + $\overline{w}$
		
			\begin{proof}[15cm]
				$\overline{z+w}$ = $\overline{(a+c,b+d)}$ = (a+c,-b-d)
				= (a,-b) + (c,-d) = $\overline{z}$ + $\overline{w}$
			\end{proof}

		\item $\overline{zw}$ = $\overline{z}$ $\overline{w}$
		
			\begin{proof}[15cm]
				$\overline{zw}$ = $\overline{(ac-bd,ad+bc)}$ = (ac-bd,-ad-bc)
				= (a,-b) (c,-d) = $\overline{z}$ $\overline{w}$
			\end{proof}

		\item z + $\overline{z}$ = 2 Re(z)
			\hspace{1cm} z - $\overline{z}$ = 2i Im(z)

			\begin{proof}[15cm]
				z + $\overline{z}$ = (a,b) + (a,-b) = (2a,0) = 2 Re(z)

				z - $\overline{z}$ = (a,b) - (a,-b) = (0,2b) = (0,2) b = 2i Im(z)
			\end{proof}

		\item z$\overline{z}$ $\geq$ 0

			\begin{proof}[15cm]
				z$\overline{z}$ = (a,b)(a,-b) = (a$^2$ + b$^2$ , -ab+ab)
				= a$^2$ + b$^2$ $\geq$ 0
			\end{proof}
	\end{enumerate}

	\vspace{0.5cm}



	\begin{definition}{Absolute Value}{16cm}
		Let absolute value: $|$ z $|$ = $\sqrt{z \overline{z}}$

		Let z = (a,b) and w = (c,d):
	\end{definition}
	
	\begin{enumerate}[label=(\alph*), leftmargin=2cm, itemsep=0.1cm]
		\item If z $\neq$ 0, then $|$ z $|$ $>$ 0.

			\begin{proof}[15cm]
				$\sqrt{z\overline{z}}$ = $\sqrt{a^2 + b^2}$ $\geq$ 0
				where $|$ z $|$ = 0 only if a,b = 0 so only if z = (0,0).
			\end{proof}

		\item $|$ $\overline{z}$ $|$ = $|$ z  $|$
			
			\begin{proof}[15cm]
				$|$ $\overline{z}$ $|$ = $\sqrt{a^2 + (-b)^2}$
				= $\sqrt{a^2 + b^2}$ = $|$ z $|$
			\end{proof}

		\item $|$ zw $|$ = $|$ z $|$ $|$ w $|$
		
			\begin{proof}[15cm]
				$|$ zw $|$ = $|$ (ac-bd,ad+bc) $|$ = $\sqrt{(ac-bd)^2 + (ad+bc)^2}$
			
				= $\sqrt{a^2c^2 + b^2d^2 + a^2d^2 + b^2c^2}$
				= $\sqrt{(a^2+b^2)(c^2+d^2)}$

				= $\sqrt{a^2+b^2}$ $\sqrt{c^2+d^2}$ = $|$ z $|$ $|$ w $|$
			\end{proof}

		\item $|$ Re(z) $|$ $\leq$ $|$ z $|$

			\begin{proof}[15cm]
				$|$ Re(z) $|$ = $|$ a $|$ = $\sqrt{a^2}$
				$\leq$ $\sqrt{a^2+b^2}$ = $|$ z $|$
			\end{proof}

		\item $|$ z+w $|$ $ \leq $  $|$ z $|$ + $|$ w $|$
		
			\begin{proof}[15cm]
				$| z+w |^2$ = (z+w)$\overline{(z+w)}$
				= (z+w)($\overline{z} + \overline{w}$)
				= z$\overline{z}$ + z$\overline{w}$
				+ w$\overline{z}$ + w$\overline{w}$
			
				= $|z|^2$ + $|w|^2$ + 2 Re(z$\overline{w}$)
				$\leq$ $|z|^2$ + $|w|^2$ + 2 $|z\overline{w}|$

				= $|z|^2$ + $|w|^2$ + 2$|z||w|$
				= ($|z|$ + $|w|$)$^2$				
			\end{proof}
	\end{enumerate}

	\newpage





\subsection{ Euclidean Spaces }

	\begin{definition}{Euclidean Spaces}{16cm}
		For each positive integer k, let $\mathbb{R}$$^\text{k}$ be the set of all ordered k-tuples:

		\hspace{0.5cm}
		x = $(x_1,...,x_k)$
		\hspace{1cm}
		for each $x_i$ $\in$ $\mathbb{R}$

		with the properties:
		\begin{itemize}[leftmargin=1cm, itemsep=0.1cm]
			\item x+y = $(x_1+y_1,...,x_k+y_k)$ $\in$ $\mathbb{R}^{\text{k}}$
		
			\item cx = $(cx_1,...,cx_k)$ $\in$ $\mathbb{R}$$^\text{k}$
		\end{itemize}

		So, $\mathbb{R}$$^\text{n}$ has a vector space structure.
		Similarly, for $\mathbb{C}^{\text{n}}$.
	\end{definition}

	\vspace{0.5cm}



	\begin{definition}{Inner Product for $\mathbb{R}^k$ (Dot Product)}{16cm}
		$x \cdot y = x_1y_1 + ... + x_ky_k$ $\in$ $\mathbb{R}$
	\end{definition}

	\vspace{0.5cm}



	\begin{definition}{Norm}{16cm}
		$|x| = \sqrt{x \cdot x}$ = $\sqrt{\sum_{i=1}^k x_i^2}$
	\end{definition}

	\vspace{0.5cm}



	\begin{definition}{Extension to $\mathbb{C}^k$}{16cm}
		For z,w $\in$ $\mathbb{C}^n$:
		
		\begin{itemize}[leftmargin=1cm, itemsep=0.1cm]
			\item $z \cdot w = z_1\overline{w_1} + ... + z_k\overline{w_k}$
		
			\item $z \cdot z = z_1\overline{z_1} + ... + z_k\overline{z_k}
				= |z_1|^2 + ... + |z_k|^2$ = $|z|^2$
		\end{itemize}
	\end{definition}

	\vspace{0.5cm}





\subsection{ Cauchy-Schwarz } 

	\begin{wtheorem}{Cauchy-Schwarz}{16cm}
		If $\alpha_1 , ... , \alpha_n$ $\in$ $\mathbb{C}$ and
		$b_1 , ... , b_n$ $\in$ $\mathbb{C}$, then:

		\hspace{0.5cm}
		$| \sum_{\text{j=1}}^{n} a_j(\overline{b_j})  |^2 \leq
		\sum_{\text{j=1}}^{n} |a_j|^2 \sum_{\text{j=1}}^{n} |b_j|^2 $
	\end{wtheorem}

	\begin{proof}
		Let $A = \sum$ $|a_j|^2$ and $B = \sum$ $|b_j|^2$ and
		$C = \sum$ $a_j(\overline{b_j})$.

		If $B = 0$, then $b_1$ = ... = $b_n$ = 0. Thus, $0 \leq A(0)$ holds true.

		Suppose $B > 0$. Then:

		\hspace{1cm}
		$\sum$ $|Ba_j - Cb_j|^2$
		= $\sum$ $(Ba_j - Cb_j)\overline{(Ba_j - Cb_j)}$
		= $\sum$ $(Ba_j - Cb_j)(\overline{B} \ \overline{a_j}
		- \overline{C} \ \overline{b_j})$

		\hspace{1cm}
		= $\sum$ $(Ba_j-Cb_j)(B\overline{a_j} - \overline{C} \ \overline{b_j})$
		= $\sum$ $B^2a_j\overline{a_j} - B\overline{C}a_j\overline{b_j}
		- BC\overline{a_j}b_j + C\overline{C}b_j\overline{b_j}$

		\hspace{1cm}
		= $B^2 \sum |a_j|^2 - B\overline{C}\sum a_j\overline{b_j}
		- BC \sum \overline{a_j}b_j+ |C|^2 \sum |b_j|^2$

		\hspace{1cm}
		= $B^2A - B\overline{C}C - BC\overline{C} + |C|^2B$
		= $B^2A - 2|C|^2B + |C|^2B$ = $B^2A -|C|^2B$

		\hspace{1cm}
		= $B(AB - |C|^2)$

		Since $| Ba_j - Cb_j |$ $\geq$ 0, then $B(AB - |C|^2)$ $\geq$ 0.

		Since $B$ $>$ 0, then $AB - |C|^2$ $\geq$ 0 so $AB$ $\geq$ $|C|^2$.
	\end{proof}
	
	\vspace{0.5cm}
	


	\begin{corollary}{$|z \cdot w|$ $\leq$ $|z| |w|$}{16cm}
		For z,w $\in$ $\mathbb{C}$:

		\hspace{0.5cm}
		$|z \cdot w|$ $\leq$ $|z| |w|$
	\end{corollary}

	\begin{proof}
		Since $|z_i|^2 = z_i\overline{z_i}$, then
		$\sum z_i\overline{z_i} = \sum |z_i|^2$ = $|z|^2$. Thus:

		\hspace{1cm}
		$|z \cdot w|^2$ = $|\sum z_i\overline{w_i} \ |^2$
		$\leq$ $\sum |z_i|^2 \sum |w_i|^2$ = $|z|^2$ $|w|^2$

		\hspace{0.5cm}
		$|z \cdot w|$ $\leq$ $|z| |w|$
	\end{proof}

	\newpage



	\begin{wtheorem}{Properties of $\mathbb{R}^k$}{16cm}
		Let $x,y,z$ $\in$ $\mathbb{R}^k$ where $\alpha$ $\in$ $\mathbb{R}$:
	\end{wtheorem}
	
	\begin{enumerate}[label=(\alph*), leftmargin=2cm, itemsep=0.1cm]
		\item $|x|$ $\geq$ 0 where $|x| = 0$ only if x = 0

			\begin{proof}[15cm]
				$|x| = \sqrt{\sum_{i=1}^{k} x_i^2} \geq 0$ where $|x| = 0$
				only if $x_1 = ... = x_k = 0$
			\end{proof}

		\item $|\alpha x|$ = $|\alpha| |x|$
		
			\begin{proof}[15cm]
				$|\alpha x|$ = $\sqrt{\sum_{i=1}^k (\alpha x_i)^2}$
				= $\sqrt{\alpha^2} \sqrt{\sum_{i=1}^k x_i^2}$ = $|\alpha| |x|$
			\end{proof}
	
		\item $|x+y|$ $\leq$ $|x| + |y|$
		
			\begin{proof}[15cm]
				$|x+y|^2$ = $(x+y) \cdot (x+y)$ = $|x|^2 + 2(x \cdot y) + |y|^2$

				$\leq$ $|x|^2 + 2|x||y| + |y|^2$ = $(|x|+|y|)^2$
			\end{proof}

		\item $|x-y|$ $\leq$ $|x-z| + |y-z|$
		
			\begin{proof}[15cm]
				$|x-y|$ = $|x-z + z-y|$ $\leq$ $|x-z| + |z-y|$ = $|x-z| + |y-z|$
			\end{proof}
	\end{enumerate}




	