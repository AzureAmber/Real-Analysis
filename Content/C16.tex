\newpage

\section[Day 16: Discontinuities]{Discontinuities}

\subsection{ Discontinuities }

{ \color{blue} Definition 16.1.1: Right and left Limits }

    \begin{adjustbox}{minipage=14cm, right, vspace=0.1cm 0cm}
        Let f be defined on (a,b).
        
        \vspace{0.2cm}

        Then for any x where x $\in$ [a,b), f(x+) = q if:
        
        \hspace{1cm}
        f($t_n$) $\rightarrow$ q as n $\rightarrow$ $\infty$
        for all sequences \{$t_n$\} in (x,b) such that $t_n$ $\rightarrow$ x.

        \vspace{0.2cm}

        Then for any x where x $\in$ (a,b], f(x-) = q if:
        
        \hspace{1cm}
        f($t_n$) $\rightarrow$ q as n $\rightarrow$ $\infty$
        for all sequences \{$t_n$\} in (a,x) such that $t_n$ $\rightarrow$ x.

        \vspace{0.2cm}

        Then $\lim_{t \rightarrow x}$ f(t) exists if and only if
        f(x-) = f(x+) = $\lim_{t \rightarrow x}$ f(t).
    \end{adjustbox}

    \vspace{0.5cm}

{ \color{blue} Definition 16.1.2: Types of discontinuities }

    \begin{adjustbox}{minipage=14cm, right, vspace=0.1cm 0cm}
        Let f be defined on (a,b).

        If f is discontinuous at x, but f(x+) and f(x-) exists,
        then f have a simple discontinuity of the first kind else
        it is a discontinuity of the second kind.

        Thus, a simple discontinuity is either:

        \begin{itemize}[leftmargin=1cm, itemsep=0.1cm]
            \item f(x-) $\not = $ f(x+)
            
            \item f(x-) = f(x+) $\not =$ f(x) 
        \end{itemize}
    \end{adjustbox}





\subsection{ Monotonic Functions }

{ \color{blue} Definition 16.2.1: Monotonic }

    \begin{adjustbox}{minipage=14cm, right, vspace=0.1cm 0cm}
        Let f be real on (a,b).

        f is monotonically increasing if f(x) $\leq$ f(y) for $a < x < y < b$.

        f is monotonically decreasing if f(x) $\geq$ f(y) for $a < x < y < b$.
    \end{adjustbox}

    \vspace{0.5cm}

{ \color{red} Theorem 16.2.2: Right and left limits of monotonics on (a,b) }

    \begin{adjustbox}{minipage=14cm, right, vspace=0.1cm 0cm}
        Let f be monotonically increasing on (a,b).

        Then f(x+) and f(x-) exists at every x $\in$ (a,b) where:

        \hspace{1cm}
        sup$_{t \in (a,x)}$ f(t)
        = f(x-)
        $\leq$ f(x)
        $\leq$ f(x+)
        = inf$_{t \in (x,b)}$ f(t)

        Furthermore, for $a < x < y < b$, f(x+) $\leq$ f(y-).

        Properties analogous for monotonically decreasing functions.
    \end{adjustbox}

{ \color{magenta} \underline{Proof} }

    \fbox{
    \begin{minipage}{15cm}
        Since f is monotonically increasing, then for t $\in$ (a,x),
        f(t) is bounded above by f(x) and thus, by the least upper bounded
        property, sup$_{t \in (a,x)}$ f(t) exists.

        For $\epsilon > 0$, there exists a $\delta > 0$ such that
        sup$_{t \in (a,x)}$ f(t) - $\epsilon$
        $<$ f(x - $\delta$)
        $\leq$ sup$_{t \in (a,x)}$ f(t)
        for a $<$ x - $\delta$ $<$ x.
        Since f(x - $\delta$) $\leq$ f(t) $\leq$ sup$_{t \in (a,x)}$ f(t)
        for t $\in$ (x-$\delta$,x), then
        $|f(t) - \text{sup}_{t \in (a,x)} f(t)| < \epsilon$ for
        t $\in$ (x-$\delta$,x) so f(x-) = sup$_{t \in (a,x)}$ f(t).

        For $\epsilon > 0$, there exists a $\delta > 0$ such that
        inf$_{t \in (x,b)}$ f(t)
        $<$ f(x + $\delta$)
        $\leq$ inf$_{t \in (x,b)}$ f(t) + $\epsilon$
        for x $<$ x + $\delta$ $<$ b.
        Since f(x + $\delta$) $\geq$ f(t) $\geq$ inf$_{t \in (x,b)}$ f(t)
        for t $\in$ (x,x+$\delta$), then
        $|f(t) - \text{inf}_{t \in (x,b)} f(t)| < \epsilon$ for
        t $\in$ (x,x+$\delta$) so f(x+) = inf$_{t \in (x,b)}$ f(t).

        Thus,
        sup$_{t \in (a,x)}$ f(t) = f(x-)
        $\leq$ f(x)
        $\leq$ f(x+) = inf$_{t \in (x,b)}$ f(t).

        \hspace{0.2cm}

        If $a < x < y < b$, then:

        \hspace{1cm}
        f(x+) = inf$_{t \in (x,b)}$ f(t)
        = inf$_{t \in (x,y)}$ f(t)
        $\leq$ sup$_{t \in (x,y)}$ f(t)
        = sup$_{t \in (a,y)}$ f(t)
        = f(y-)
    \end{minipage} }

\newpage

{ \color{orange} Corollary 16.2.3: Monotonics can only have simple discontinuities }

    \begin{adjustbox}{minipage=14cm, right, vspace=0.1cm 0cm}
        Monotonic functions have no discontinuities of the second kind.
    \end{adjustbox}

{ \color{magenta} \underline{Proof} }

    \fbox{
    \begin{minipage}{15cm}
        By {\color{red} theorem 16.2.2}, f(x-) and f(x+) exists and thus,
        f can only have simple discontinuities and not discontinuities
        of the second kind.
    \end{minipage} }

    \vspace{0.5cm}

{ \color{red} Theorem 16.2.4: Discontinuities of monotonics is at most countable }

    \begin{adjustbox}{minipage=14cm, right, vspace=0.1cm 0cm}
        Let f be monotonic on (a,b).

        Then the set of points of (a,b) where f is discontinuous is
        at most countable.
    \end{adjustbox}

{ \color{magenta} \underline{Proof} }

    \fbox{
    \begin{minipage}{15cm}
        Suppose f is increasing.
        Let E be the set of points where f is discontinuous.
        Then for x $\in$ E, there is a rational r(x) where
        f(x-) $<$ r(x) $<$ f(x+).
        
        Then for $x_1$ $<$ $x_2$, by {\color{red} theorem 16.2.2},
        f($x_1$+) $\leq$ f($x_2$-). Then:

        \hspace{1cm}
        f($x_1$-) $<$ r($x_1$) $<$ f($x_1$+)
        $\leq$ f($x_2$-) $<$ r($x_2$) $<$ f($x_2$+)

        Thus, r($x_1$) $\not =$ r($x_2$) if $x_1$ $\not =$ $x_2$.

        Since there is a 1-1 correspondence between E and a subset
        of rational numbers which is countable, then E is at most countable.

        If f is decreasing, proof is analogous.
    \end{minipage} }





\subsection{ Infinite Limits $\backslash$ Limits at Infinity }

{ \color{blue} Definition 16.3.1: Neighborhoods in extended reals }

    \begin{adjustbox}{minipage=14cm, right, vspace=0.1cm 0cm}
        For any real c, a neighborhood of $+\infty$ = (c,$+\infty$).

        For any real c, a neighborhood of $-\infty$ = ($-\infty$,c).
    \end{adjustbox}

    \vspace{0.5cm}

{ \color{blue} Definition 16.3.2: Infinite Limits }

    \begin{adjustbox}{minipage=14cm, right, vspace=0.1cm 0cm}
        Let real function f be defined on E $\subset$ $\mathbb{R}$.

        Then f(t) $\rightarrow$ A as t $\rightarrow$ x where A and x
        are extended reals if:
        
        \begin{adjustbox}{minipage=13cm, right, vspace=0.1cm 0cm}
            For every neighborhood U of A,
            there is a neighborhood V of x such that V $\cap$ E $\not =$ $\emptyset$
            and f(t) $\in$ U for all t $\in$ V $\cap$ E where t $\not =$ x.
        \end{adjustbox}
    \end{adjustbox}

    \vspace{0.5cm}

{ \color{red} Theorem 16.3.2: Arithmetric operations on functions of infinite limits }

    \begin{adjustbox}{minipage=14cm, right, vspace=0.1cm 0cm}
        Let f,g be defined on E $\subset$ $\mathbb{R}$ where
        f(t) $\rightarrow$ A and g(t) $\rightarrow$ B as t $\rightarrow$ x.

        \begin{enumerate}[label=(\alph*), leftmargin=2cm, itemsep=0.1cm]
            \item If f(t) $\rightarrow$ A', then A' = A.
            
            \item (f+g)(t) $\rightarrow$ A + B
            
            \item (fg)(t) $\rightarrow$ AB
            
            \item $\frac{f}{g}$(t) $\rightarrow$ $\frac{A}{B}$
        \end{enumerate}
    \end{adjustbox}