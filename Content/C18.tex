\newpage

\section[Day 18: Convergence Theorems]{ Lebesgue Convergence Theorems }

\subsection{ Bounded Convergence Theorem }

    \begin{wtheorem}{Bounded Convergence Theorem}{14cm}
        Suppose measurable \{$f_n$\} on [0,1] converge pointwise to f where
        $|f_n(x)|$ $\leq$ M.
        Then, f is a bounded measurable function where:
        
        \hspace{0.5cm}
        $\lim_{n \rightarrow \infty}$ $\int$ $f_n$ $d\mu$ = $\int$ f $d\mu$
    \end{wtheorem}

    \begin{proof}
        Since $\lim_{n \rightarrow \infty}$ $f_n$ = f pointwise, then
        for any x $\in$ [0,1], then is a $N_x$ where for n $\geq$ $N_x$:

        \hspace{0.5cm}
        $|f(x) - f_n(x)|$ $<$ $\epsilon$

        \hspace{0.5cm}
        $|f(x)|$ $\leq$ $|f(x) - f_n(x)| + |f_n(x)|$ $<$ $\epsilon + M$
        \hspace{1cm}
        $\Rightarrow$
        \hspace{1cm}
        $|f(x)|$ $\leq$ M

        Thus, f is bounded.
        Since $\lim_{n \rightarrow \infty}$ $f_n$ = f, then by
        {\color{red} theorem 17.4.7}, f is measurable.

        Let set $E_n$ = \{ x $\in$ [0,1] $|$
                            $|f_n(x) - f(x)|$ $<$ $\frac{\epsilon}{2}$ \}.
        Since $\lim_{n \rightarrow \infty}$ $f_n$ = f pointwise, then
        $\cup_{n=1}^{\infty}$ $E_n$ = [0,1].
        Since $E_n$ $\subset$ $E_{n+1}$, then
        $\lim_{n \rightarrow \infty} \mu(E_n)$ = $\mu([0,1])$ = 1.

        Then, there is a N where $\mu(E_N)$ $>$ $1 - \frac{\epsilon}{4M}$
        so $\mu(E_N^c)$ $<$ $\frac{\epsilon}{4M}$. Thus:

        \hspace{0.5cm}
        $\lim_{n \rightarrow \infty}$ $|\int f_n d\mu - \int f d\mu|$
        = $\lim_{n \rightarrow \infty}$ $|\int f_n - f d\mu|$
        $\leq$ $\lim_{n \rightarrow \infty}$ $\int |f_n - f| d\mu$

        \hspace{5.25cm}
        = $\int_{E_N} |f_n - f| d\mu$ + $\int_{E_N^c} |f_n - f| d\mu$

        \hspace{5.25cm}
        $<$ $\frac{\epsilon}{2} \mu(E_N)$ + $2M \mu(E_N^c)$
        $<$ $\frac{\epsilon}{2}$ + $2M \frac{\epsilon}{4M}$
        = $\epsilon$
    \end{proof}

    \vspace{0.5cm}



    \begin{definition}{Almost Everywhere}{14cm}
        If a property holds for all x except for a null set,
        then it holds {\color{lblue} almost everywhere}
    \end{definition}

    \vspace{0.5cm}



    \begin{wtheorem}{Bounded Convergence Theorem for Almost Everywhere}{14cm}
        Suppose bounded \{$f_n$\} on [0,1] are measurable
        and f is bounded such that $\lim_{n \rightarrow \infty}$ $f_n$ = f
        for almost all x.
        If $|f_n(x)|$ $\leq$ M almost everywhere, then f is measurable where:

        \hspace{0.5cm}
        $\lim_{n \rightarrow \infty}$ $\int$ $f_n$ $d\mu$ = $\int$ f $d\mu$
    \end{wtheorem}

    \begin{proof}
        Let A = \{ x $|$ $\lim_{n \rightarrow \infty}$ $f_n(x)$ $\not =$ f(x) \}
        so $\mu(A)$ = 0.
        Let $D_n$ = \{ x $|$ $|f_n(x)|$ $>$ M \} so $\mu(D_n)$ = 0.
        Let E = A $\cup_{n=1}^{\infty} D_n$. Thus:

        \hspace{0.5cm}
        $\mu(E)$ $\leq$ $\mu(A) + \sum_{i=1}^{\infty} \mu(D_n)$ = 0
        \hspace{1cm}
        $\Rightarrow$
        \hspace{1cm}
        $\mu(E)$ = 0

        Let $g_n(x)$ = $f_n(x) \mathfrak{X}_{E^c}(x)$
        which is measurable since $f_n(x)$,$\mathfrak{X}_{E^c}(x)$
        are measurable. Then, $|g_n(x)|$ $\leq$ M.
        Let g(x) = $f(x) \mathfrak{X}_{E^c}(x)$
        so $\lim_{n \rightarrow \infty}$ $g_n(x)$ = g(x)
        and g(x) $\leq$ M.

        Since $\lim_{n \rightarrow \infty}$ $g_n(x)$ = g(x),
        then by {\color{red} theorem 17.4.7},  g(x) is measurable.

        Since g(x) = f(x) almost everywhere, then
        by {\color{red} theorem 17.4.6b}, f(x) is measurable.

        \hspace{0.5cm}
        $\int g d\mu$ = $\int f d\mu$
        \hspace{1cm}
        $\int g_n d\mu$ = $\int f_n d\mu$

        By {\color{red} theorem 18.1.1},
        $\lim_{n \rightarrow \infty}$ $\int$ $g_n$ $d\mu$ = $\int$ g $d\mu$.
        Thus:

        \hspace{0.5cm}
        $\lim_{n \rightarrow \infty}$ $\int$ $f_n$ $d\mu$
        = $\lim_{n \rightarrow \infty}$ $\int$ $g_n$ $d\mu$
        = $\int$ g $d\mu$
        = $\int$ f $d\mu$
    \end{proof}

    \newpage



    
\subsection{ Integral of Unbounded Functions }

    \begin{definition}{Integrable Function}{14cm}
        If f: [0,1] $\rightarrow$ [0,$\infty$] is Lebesgue measurable,
        let $f_n(x)$ = min(f(x),n).

        Then $f_n$ is a bounded measurable function and let:

        \hspace{0.5cm}
        $\int$ f $d\mu$ = $\lim_{n \rightarrow \infty}$ $f_n(x)$ $d\mu$

        If $\int$ f $d\mu$ $<$ $\infty$, then f is {\color{lblue} integrable}.
    \end{definition}

    \vspace{0.5cm}



    \begin{wtheorem}{Unbounded sets of Integrable functions have measure 0}{14cm}
        If f is a non-negative integrable function and
        A = \{ x $|$ f(x) = $\infty$ \}, then:
        
        \hspace{0.5cm}
        $\mu(A)$ = 0 
    \end{wtheorem}

    \begin{proof}
        If x $\in$ A, then $f_n(x)$ = n $\geq$ $n \mathfrak{X}_A(x)$.
        Thus, $\int f_n d\mu$ $\geq$ $\int n \mathfrak{X}_A d\mu$ = $n\mu(A)$.

        If $\mu(A)$ $>$ 0, then:

        \hspace{0.5cm}
        $\int f d\mu$
        = $\lim_{n \rightarrow \infty}$ $\int f_n d\mu$
        $\geq$ $\lim_{n \rightarrow \infty}$ $\int n \mathfrak{X}_A d\mu$
        = $\lim_{n \rightarrow \infty}$ $n\mu(A)$
        = $\infty$

        Thus, if f is integrable, then $\mu(A)$ = 0.
    \end{proof}

    \vspace{0.5cm}



    \begin{wtheorem}{Integrable functions for Almost Everywhere}{14cm}
        Suppose f,g are non-negative measurable functions with g(x) $\leq$ f(x)
        for almost all x. If f is integrable, then g is integrable where:

        \hspace{0.5cm}
        $\int$ g $d\mu$
        $\leq$ $\int$ f $d\mu$

        If g = 0 almost everywhere, then $\int$ g $d\mu$ = 0.
    \end{wtheorem}

    \begin{proof}
        If $f_n(x)$ = min(f(x),n) and $g_n(x)$ = min(g(x),n),
        then $f_n,g_n$ are bounded measurable functions
        where $g_n(x)$ $\leq$ $f_n(x)$ almost everywhere.
        If f is integrable, then:

        \hspace{0.5cm}
        $\int g_n d\mu$
        $\leq$ $\int f_n d\mu$
        $\leq$ $\int f d\mu$

        Since \{$g_n$\} is increasing and bounded above by $\int f d\mu$,
        then $\int g d\mu$ is finite and thus, exist.
        If 0 $\leq$ g(x) $\leq$ 0 almost everywhere,
        for almost all x so $\int g d\mu$ = $\int 0 d\mu$ = 0.
    \end{proof}

    \vspace{0.5cm}



    \begin{corollary}{If integrable f $\geq$ 0, then
    $\int$ f $d\mu$ 0 $\rightleftharpoons$ f(x) = 0 almost everywhere}{14cm}
        If f: [0,1] $\rightarrow$ [0,$\infty$] is a non-negative integrable
        function and $\int f d\mu$ = 0, then f(x) = 0 almost everywhere
    \end{corollary}

    \begin{proof}
        Let $E_n$ = \{ x $|$ f(x) $\geq$ $\frac{1}{n}$ \}.
        Then, f(x) $\geq$ $\frac{1}{n} \mathfrak{X}_{E_n}(x)$ where:

        \hspace{0.5cm}
        $\frac{1}{n} \mu(E_n)$
        = $\int \frac{1}{n} \mathfrak{X}_{E_n} d\mu$
        $\leq$ $\int f d\mu$
        = 0

        Thus, $\mu(E_n)$ = 0.
        Let E = \{ x $|$ f(x) $>$ 0 \}
        so E = $\cup_{n=1}^{\infty} E_n$ where $E_n$ $\subset$ $E_{n+1}$ so:

        \hspace{0.5cm}
        $\mu(E)$
        = $\mu(\cup_{n=1}^{\infty} E_n)$
        = $\lim_{n \rightarrow \infty} \mu(E_n)$
        = 0.
    \end{proof}

    \newpage



    \begin{wtheorem}{Absolute Continuity}{14cm}
        If f is a non-negative integrable function, then for $\epsilon > 0$,
        there is a $\delta > 0$ where for every measurable A $\subset$ [0,1]
        with $\mu(A)$ $<$ $\delta$, then $\int_A$ f $d\mu$ $<$ $\epsilon$
    \end{wtheorem}

    \begin{proof}
        Let $E_n$ = \{ x $|$ f(x) $\geq$ n \}
        so $f_n(x)$
        = $\begin{cases}
            f(x) & x \in E_n^c \\
            n & x \in E_n
        \end{cases}$.
        Thus:
        
        \hspace{0.5cm}
        f(x) - $f_n(x)$
        = $\begin{cases}
            0 & x \in E_n^c \\
            f(x) - n & x \in E_n
        \end{cases}$

        \hspace{0.5cm}
        $\int f d\mu$ - $\int f_n d\mu$
        = $\int f-f_n d\mu$
        = $\int_{E_n} f(x)-n d\mu$

        Since f is integrable, then
        $\lim_{n \rightarrow \infty}$ $\int f d\mu$ - $\int f_n d\mu$ = 0. Thus:

        \hspace{0.5cm}
        $\lim_{n \rightarrow \infty}$ $\int_{E_n} f(x)-n d\mu$ = 0

        Thus, there is a N where
        $\int_{E_n} f(x)-n d\mu$ $<$ $\frac{\epsilon}{2}$.
        Then for $\delta$ $<$ $\frac{\epsilon}{2N}$, if $\mu(A)$ $<$ $\delta$:

        \hspace{0.5cm}
        $\int_A f d\mu$
        = $\int_{A \cap E_N} f d\mu$ + $\int_{A \cap E_N^c} f d\mu$
        $\leq$ $\int_{A \cap E_N} (f-N) d\mu$
                + $\int_{A \cap E_N} N d\mu$
                + $\int_{A \cap E_N^c} N d\mu$

        \hspace{1.9cm}
        $\leq$ $\int_{A \cap E_N} (f-N) d\mu$
                + $\int_A N d\mu$
        $<$ $\frac{\epsilon}{2}$ + $N\mu(A)$
        $<$ $\frac{\epsilon}{2}$ + $N\delta$
        $<$ $\frac{\epsilon}{2}$ + $N\frac{\epsilon}{2N}$
        $<$ $\epsilon$
    \end{proof}

    \vspace{0.5cm}



    \begin{corollary}{Uniform Continuity of the Integral}{14cm}
        If f: [0,1] $\rightarrow$ [0,$\infty$] is an integrable
        function where F(x) = $\int_{[0,x]}$ f $d\mu$, then F(x) is continuous
    \end{corollary}

    \begin{proof}
        By {\color{red} theorem 17.7.5}, for $\epsilon > 0$,
        there is a $\delta > 0$ where for $\mu([x,y])$ $<$ $\delta$, then
        $\int_{[x,y]}$ f $d\mu$ $<$ $\epsilon$.

        \hspace{0.5cm}
        $|F(y) - F(x)|$
        = $|\int_{[0,y]} f d\mu - \int_{[0,x]} f d\mu|$
        = $|\int_{[x,y]} f d\mu|$
        $<$ $\epsilon$

        Thus, F(x) is uniformly continuous.
    \end{proof}

    \vspace{0.5cm}





\subsection{ Dominated Convergence Theorems }

    \begin{wtheorem}{Dominated Convergence Theorem}{14cm}
        Suppose non-negative measurable \{$f_n$\} on [0,1]
        converge pointwise to f for almost all x.
        If there is a non-negative integrable g where
        $f_n(x)$ $\leq$ g(x) for almost all x, then f is integrable where:

        \hspace{0.5cm}
        $\int$ f $d\mu$ = $\lim_{n \rightarrow \infty}$ $\int$ $f_n$ $d\mu$
    \end{wtheorem}

    \begin{proof}
        Let $h_n$ = $f_n \mathfrak{X}_E$ and h = $f \mathfrak{X}_E$
        where E = \{ x $|$ $\lim_{n \rightarrow \infty}$ $f_n(x)$ = f(x) \}
        so $\lim_{n \rightarrow \infty}$ $h_n(x)$ = h(x) for all x.
        Since $h_n(x)$ = $f_n \mathfrak{X}_E$ $\leq$ g(x) for almost all
        x and g is integrable, then h(x) $\leq$ g(x) for almost all x
        so by {\color{red} theorem 18.2.3}, h is integrable.

        For $\epsilon > 0$, let $E_n$ =
        \{ x $|$ $|h_m(x) - h(x)|$ $<$ $\frac{\epsilon}{2}$ for all m $\geq$ n \}.
        By {\color{red} theorem 18.2.5}, there is a $\delta > 0$
        where for each measurable A $\subset$ [0,1] with $\mu(A)$ $<$ $\delta$,
        then $\int_A$ g $d\mu$ $<$ $\frac{\epsilon}{4}$.

        Since $\lim_{n \rightarrow \infty}$ $h_n(x)$ = h(x) for all x $\in$ [0,1],
        then any x $\in$ $E_n$ in some n so $\cup_{n=1}^{\infty}$ $E_n$ = [0,1].
        Since $E_n$ $\subset$ $E_{n+1}$, then
        $\lim_{n \rightarrow \infty}$ $\mu(E_n)$ = $\mu([0,1])$ = 1.
        Thus, there is a n where $\mu(E_n)$ $>$ $1-\delta$
        so $\mu(E_n^c)$ $<$ $\delta$.
        Note $|h_n(x) - h(x)|$ $\leq$ $|h_n(x)| + |h(x)|$ $\leq$ $2g(x)$
        for almost all x. Thus, for any m $>$ n:

        \hspace{0.5cm}
        $|\int h_m d\mu - \int h d\mu|$
        $\leq$ $\int |h_m - h| d\mu$
        = $\int_{E_n} |h_m - h| d\mu$ + $\int_{E_n^c} |h_m - h| d\mu$
        
        \hspace{3.8cm}
        $<$ $\frac{\epsilon}{2} \mu(E_n)$ + $2 \int_{E_n^c} g d\mu$
        $<$ $\frac{\epsilon}{2}$ + $2 \frac{\epsilon}{4}$
        = $\epsilon$

        \hspace{0.5cm}
        $\lim_{n \rightarrow \infty}$ $\int$ $f_n$ $d\mu$
        = $\lim_{n \rightarrow \infty}$ $\int$ $h_n$ $d\mu$
        = $\int$ h $d\mu$
        = $\int$ f $d\mu$
    \end{proof}

    \newpage



    \begin{wtheorem}{Fatou's Lemma}{14cm}
        If non-negative measurable \{$g_n$\} on [0,1]
        converge pointwise to g(x) for almost all x, then:

        \hspace{0.5cm}
        $\int$ g $d\mu$
        $\leq$ $\lim_{n \rightarrow \infty}$ inf($\int$ $g_n$ $d\mu$)

        Thus, if $\lim_{n \rightarrow \infty}$ inf($\int$ $g_n$ $d\mu$)
        $<$ $\infty$, then g is integrable.
    \end{wtheorem}

    \begin{proof}
        Since $g_n$ is measurable and $\lim_{n \rightarrow \infty}$ $g_n$ = g
        for almost all x, then g is measurable.

        Let bounded, measurable h be h(x) $\leq$ g(x) for all x.
        Let $h_n(x)$ = min(h(x),$g_n(x)$) so $h_n$ is bounded and measurable
        where $\lim_{n \rightarrow \infty}$ $h_n$ = h.
        Then by {\color{red} theorem 18.1.1}:

        \hspace{0.5cm}
        $\int$ h $d\mu$
        = $\lim_{n \rightarrow \infty}$ $\int$ $h_n$ $d\mu$
        $\leq$ $\lim_{n \rightarrow \infty}$ inf($\int$ $g_n$ $d\mu$)

        Since the inequality holds for any bounded, measurable h
        where h(x) $\leq$ g(x), then let h(x) = $g_m(x)$ = min($g_n(x)$,m).
        Thus, for any m:

        \hspace{0.5cm}
        $\int$ $g_m$ $d\mu$
        $\leq$ $\lim_{n \rightarrow \infty}$ inf($\int$ $g_n$ $d\mu$)

        \hspace{0.5cm}
        $\int$ g $d\mu$
        = $\lim_{m \rightarrow \infty}$ $\int$ $g_m$ $d\mu$
        $\leq$ $\lim_{n \rightarrow \infty}$ inf($\int$ $g_n$ $d\mu$)
    \end{proof}

    \vspace{0.5cm}



    \begin{wtheorem}{Monotone Convergence Theorem}{14cm}
        If non-negative measurable \{$g_n$\} on [0,1]
        converge pointwise to g(x) for almost all x where
        $g_n(x)$ $\leq$ $g_{n+1}(x)$, then:

        \hspace{0.5cm}
        $\int$ g $d\mu$ = $\lim_{n \rightarrow \infty}$ $\int$ $g_n$ $d\mu$

        Thus, g is integrable if and only if
        $\lim_{n \rightarrow \infty}$ $\int$ $g_n$ $d\mu$ $<$ $\infty$.
    \end{wtheorem}

    \begin{proof}
        Since $g_n$ is measurable and $\lim_{n \rightarrow \infty}$ $g_n$ = g
        for almost all x, then g is measurable.

        If f is integrable, then by {\color{red} theorem 18.3.1}, then:
        
        \hspace{0.5cm}
        $\int$ g $d\mu$ = $\lim_{n \rightarrow \infty}$ $\int$ $g_n$ $d\mu$.

        If $\lim_{n \rightarrow \infty}$ $\int$ f $d\mu$ = $\infty$,
        then by {\color{red} theorem 18.3.2}:

        \hspace{0.5cm}
        $\lim_{n \rightarrow \infty}$ inf($\int$ $g_n$ $d\mu$) = $\infty$
        \hspace{0.5cm}
        $\Rightarrow$
        \hspace{0.5cm}
        $\lim_{n \rightarrow \infty}$ $\int$ $g_n$ $d\mu$ = $\infty$
    \end{proof}

    \vspace{0.5cm}



    \begin{corollary}{Integral of Infinite Series}{14cm}
        For non-negative measurable $u_n(x)$ and non-negative f,
        let $\sum_{n=1}^{\infty}$ $u_n(x)$ = f(x) for almost all x. Then:

        \hspace{0.5cm}
        $\int$ f $d\mu$ = $\sum_{n=1}^{\infty}$ $\int$ $u_n$ $d\mu$
    \end{corollary}

    \begin{proof}
        Let $f_N(x)$ = $\sum_{n=1}^N$ $u_n(x)$ so
        $\lim_{N \rightarrow \infty}$ $f_N(x)$
        = $\sum_{n=1}^{\infty}$ $u_n(x)$ = f(x) for almost all x.
        Since $u_n(x)$ is non-negative, then $f_N(x)$ $\leq$ $f_{N+1}(x)$.
        Then by {\color{red} theorem 18.3.3}:

        \hspace{0.5cm}
        $\int$ f $d\mu$
        = $\lim_{N \rightarrow \infty}$ $\int$ $f_N$ $d\mu$
        = $\lim_{N \rightarrow \infty}$ $\int$ $\sum_{n=1}^N$ $u_n(x)$ $d\mu$

        \hspace{1.8cm}
        = $\lim_{N \rightarrow \infty}$ $\sum_{n=1}^N$ $\int$ $u_n(x)$ $d\mu$
        = $\sum_{n=1}^{\infty}$ $\int$ $u_n$ $d\mu$
    \end{proof}

    \vspace{0.5cm}



    \begin{corollary}{Lebesgue Integral: Countable Additivity}{14cm}
        Suppose \{$E_n$\} are pairwise disjoint measurable subsets of I
        and f is a non-negative integrable function.
        If E = $\cup_{n=1}^{\infty}$, then:

        \hspace{0.5cm}
        $\int_E$ f $d\mu$
        = $\sum_{n=1}^{\infty}$ $\int_{E_n}$ f $d\mu$
    \end{corollary}

    \begin{proof}
        Let $u_n(x)$ = $f \mathfrak{X}_{E_n}$.
        Since $\mathfrak{X}_E$ = $\sum_{n=1}^{\infty} \mathfrak{X}_{E_n}$, then
        $f \mathfrak{X}_E$
        = $f \sum_{n=1}^{\infty} \mathfrak{X}_{E_n}$
        = $\sum_{n=1}^{\infty} u_n(x)$.

        Thus, by {\color{orange} corollary 18.3.4}:

        \hspace{0.5cm}
        $\int_E$ f $d\mu$
        = $\int$ $f \mathfrak{X}_E$ $d\mu$
        = $\sum_{n=1}^{\infty}$ $\int$ $u_n$ $d\mu$
        = $\sum_{n=1}^{\infty}$ $\int$ $f \mathfrak{X}_{E_n}$ $d\mu$
        = $\sum_{n=1}^{\infty}$ $\int_{E_n}$ f $d\mu$
    \end{proof}

    \newpage





\subsection{ General Lebesgue Integral }

    \begin{definition}{Measurable Function Redefined}{14cm}
        For measurable function f: [0,1] $\rightarrow$ $[-\infty,\infty]$, let:
        
        \hspace{0.5cm}
        $f^+(x)$ = max(f(x),0)
        \hspace{1cm}
        $f^-(x)$ = -min(f(x),0)

        Thus, $f^+(x)$ and $f^-(x)$ are non-negative measurable functions where:
        
        \hspace{0.5cm}
        f(x) = $f^+(x)$ - $f^-(x)$

        Then f is Lebesgue integrable if $f^+(x)$ and $f^-(x)$ are integrable.
        Thus:

        \hspace{0.5cm}
        $\int$ f $d\mu$
        = $\int$ $f^+(x)$ $d\mu$ - $\int$ $f^-(x)$ $d\mu$
    \end{definition}

    \vspace{0.5cm}



    \begin{wtheorem}{For f = g almost everywhere,
    then $\int$ f $d\mu$ = $\int$ g $d\mu$}{14cm}
        Suppose f,g are measurable functions on [0,1]
        where f = g almost everywhere. Then if f is integrable, then
        g is integrable where $\int$ f $d\mu$ = $\int$ g $d\mu$.
    \end{wtheorem}

    \begin{proof}
        If f and g are measurable functions where f = g almost everywhere,
        then $f^+$ = $g^+$ and $f^-$ = $g^-$ almost everywhere.
        Then if f is integrable, then $f^+$ and $f^-$ are integrable
        so by {\color{red} theorem 18.2.3}, $g^+$ and $g^-$ are integrable
        where:
        
        \hspace{0.5cm}
        $\int$ $f^+$ $d\mu$ = $\int$ $g^+$ $d\mu$
        \hspace{1cm}
        $\int$ $f^-$ $d\mu$ = $\int$ $g^-$ $d\mu$

        \hspace{0.5cm}
        $\int$ f $d\mu$
        = $\int$ $f^+(x)$ $d\mu$ - $\int$ $f^-(x)$ $d\mu$
        = $\int$ $g^+(x)$ $d\mu$ - $\int$ $g^-(x)$ $d\mu$
        = $\int$ g $d\mu$
    \end{proof}

    \vspace{0.5cm}



    \begin{wtheorem}{Integrable f $\rightleftharpoons$ Integrable $|f|$}{14cm}
        Measurable f: [0,1] $\rightarrow$ $[-\infty,\infty]$
        is integrable if and only if $|f|$ is integrable
    \end{wtheorem}

    \begin{proof}
        If f is integrable, then $f^+,f^-$
        are integrable. Since $|f|$ = $f^+ + f^-$, then f is integrable.

        If $|f|$ is integrable, then since $f^+,f^-$ $\leq$ $|f|$,
        by {\color{red} theorem 18.2.3}, $f^+,f^-$ are integrable
        so f is integrable.
    \end{proof}

    \vspace{0.5cm}



    \begin{wtheorem}{Lebesgue Convergence Theorem}{14cm}
        Let measurable \{$f_n$\} on [0,1] converge pointwise to f
        for almost all x. If there is a integrable g where
        $|f_n(x)|$ $\leq$ g(x) for almost all x, then f is integrable where:

        \hspace{0.5cm}
        $\int$ f $d\mu$ = $\lim_{n \rightarrow \infty}$ $\int$ $f_n$ $d\mu$
    \end{wtheorem}

    \begin{proof}
        Let $f_n^+(x)$ = max($f_n(x),0$) and $f_n^-(x)$ = -min($f_n(x),0$).
        Thus, $\lim_{n \rightarrow \infty}$ $f_n^+(x)$ = $f^+(x)$
        and $\lim_{n \rightarrow \infty}$ $f_n^-(x)$ = $f^-(x)$
        for almost all x.
        Since $|f_n(x)|$ $\leq$ g(x), then
        $f_n^+(x),f_n^-(x)$ $\leq$ g(x) for almost all x.
        Then by {\color{red} theorem 18.3.1}, $f^+,f^-$ are integrable where:

        \hspace{0.5cm}
        $\int$ $f^+$ $d\mu$
        = $\lim_{n \rightarrow \infty}$ $\int$ $f_n^+$ $d\mu$
        \hspace{1cm}
        $\int$ $f^-$ $d\mu$
        = $\lim_{n \rightarrow \infty}$ $\int$ $f_n^-$ $d\mu$

        Thus, f = $f^+ - f^-$ is integrable where:

        \hspace{0.5cm}
        $\int$ f $d\mu$
        = $\int$ $f^+ - f^-$ $d\mu$
        = $\lim_{n \rightarrow \infty}$ $\int$ $f_n^+ - f_n^-$ $d\mu$
        = $\lim_{n \rightarrow \infty}$ $\int$ $f_n$ $d\mu$
    \end{proof}

    \newpage



    \begin{wtheorem}{Integrable f can be approximated by a step function}{14cm}
        For integrable f: [0.1] $\rightarrow$ $[-\infty,\infty]$
        and $\epsilon > 0$, there is a step function g
        and measurable A $\subset$ [0,1] such that:
        
        \hspace{0.5cm}
        $\mu(A)$ $<$ $\epsilon$
        \hspace{1cm}
        $|f(x) - g(x)|$ $<$ $\epsilon$
        for all x $\not \in$ A

        If $|f(x)|$ $\leq$ M for all x, then
        there is a step function g where $|g(x)|$ $\leq$ M.
    \end{wtheorem}

    \begin{proof}
        Suppose f(x) = $\mathfrak{X}_{E}$ for some measurable set E.

        Let E $\subset$ $\cup_{i=1}^{\infty} U_i$ for open intervals
        \{$U_i$\} such that:

        \hspace{0.5cm}
        $\mu(E)$
        $\leq$ $\mu(\cup_{i=1}^{\infty} U_i)$
        $\leq$ $\sum_{i=1}^{\infty} \mu(U_i)$
        $\leq$ $\mu(E) + \frac{\epsilon}{2}$
        \hspace{0.5cm}
        $\Rightarrow$
        \hspace{0.5cm}
        $\mu((\cup_{i=1}^{\infty} U_i) \cap E^c)$ $<$ $\frac{\epsilon}{2}$

        Then choose an N such that for $V_N$ = $\cup_{i=1}^N U_i$,
        then $\mu(\cup_{i=1}^N U_i)$ $\leq$ $\sum_{i=N}^{\infty} \mu(U_i)$
        $<$ $\frac{\epsilon}{2}$.
        Let g(x) = $\mathfrak{X}_{V_N}$ so g is a step function since $V_N$
        is finite.
        Let A = \{ x $|$ f(x) $\not =$ g(x) \}.

        \hspace{0.5cm}
        A $\subset$ $(V_N \cap E^c) \cup (E \cap V_N^c)$
        $\subset$ $((\cup_{i=1}^{\infty} U_i) \cap E^c)
                    \cup (\cup_{i=N}^{\infty} U_i)$

        \hspace{0.5cm}
        $\mu(A)$ $\leq$ $\mu((\cup_{i=1}^{\infty} U_i) \cap E^c)
                        + \mu(\cup_{i=N}^{\infty} U_i)$
        $<$ $\frac{\epsilon}{2} + \frac{\epsilon}{2}$
        = $\epsilon$

        \rule[0.1cm]{15.2cm}{0.01cm}

        Suppose simple function f(x) = $\sum_{i=1}^n r_i \mathfrak{X}_{E_i}$.

        Proof is analogous to proof above except change $\frac{\epsilon}{2}$
        into $\frac{\epsilon}{2n}$.
        For j = \{1,...,n\}, let step function
        $g_j(x)$ = $\mathfrak{X}_{V_{N_j}}$
        where $V_{N_j}$ = $\cup_{i=1}^{N_j} U_{ji}$ where $E_i$ $\subset$
        $\cup_{i=1}^{\infty} U_{ji}$ open intervals.
        Thus for $A_j$ = \{ x $|$ f(x) $\not =$ $r_j g_j(x)$\},
        then $\mu(A_j) < \frac{\epsilon}{n}$
        so $\mu(\cup_{j=1}^n A_j)$
        $\leq$ $\sum_{j=1}^n \mu(A_j)$
        $<$ $\epsilon$.

        \rule[0.1cm]{15.2cm}{0.01cm}

        Suppose f(x) is a bounded measurable function.

        Then by {\color{red} theorem 17.5.1}, there is a simple function
        h(x) where $|f(x) - h(x)|$ $<$ $\frac{\epsilon}{2}$ for all x.
        As shown above, there is a step function g(x)
        such that $|h(x) - g(x)|$ $<$ $\frac{\epsilon}{2}$
        for all x $\not \in$ A for some measurable A $\subset$ [0,1]
        where $\mu(A)$ $<$ $\epsilon$. Thus, for all x $\not \in$ A:

        \hspace{0.5cm}
        $|f(x) - g(x)|$
        $\leq$ $|f(x) - h(x)| + |h(x) - g(x)|$
        $<$ $\frac{\epsilon}{2} + \frac{\epsilon}{2}$
        = $\epsilon$

        \rule[0.1cm]{15.2cm}{0.01cm}

        Suppose f is a non-negative integrable function.

        Let $A_n$ = \{ x $|$ f(x) $>$ n \}. Then:

        \hspace{0.5cm}
        n$\mu(A_n)$ = $\int n \mathfrak{X}_{A_n} d\mu$
        $\leq$ $\int f d\mu$
        $<$ $\infty$
        \hspace{0.2cm}
        $\Rightarrow$
        \hspace{0.2cm}
        $\lim_{n \rightarrow \infty}$ $\mu(A_n)$
        = $\lim_{n \rightarrow \infty}$ $\frac{1}{n} \int f d\mu$
        = 0

        Thus, there is a N where $\mu(A_N)$ $<$ $\frac{\epsilon}{2}$.
        Let $f_N$ = min(f,N) so f is a bounded measurable function.
        As shown above, there is a step function g where
        $|f_N(x) - g(x)|$ $<$ $\frac{\epsilon}{2}$
        for all x $\not \in$ B for some measurable B
        where $\mu(B)$ $<$ $\frac{\epsilon}{2}$.
        Let A = $A_N \cup B$ so $\mu(A)$ $\leq$ $\mu(A_N) + \mu(B)$
        $<$ $\epsilon$.
        Note if x $\not \in$ A, then x $\not \in$ B so f(x) = $f_N(x)$.
        Thus, for all x $\not \in$ A:

        \hspace{0.5cm}
        $|f(x) - g(x)|$
        $\leq$ $|f(x) - f_N(x)| + |f_N(x) - g(x)|$
        $<$ $\frac{\epsilon}{2}$
        $<$ $\epsilon$

        \rule[0.1cm]{15.2cm}{0.01cm}

        Suppose f is a integrable function.

        Since f = $f^+ - f^-$ where $f^+,f^-$ are non-negative integrable
        functions, then as shown above, there are step functions
        $g^+,g^-$ where $\mu(A^+),\mu(A^-)$ $<$ $\frac{\epsilon}{2}$
        and $|f^+(x) - g^+(x)|,|f^-(x) - g^-(x)|$ $<$ $\frac{\epsilon}{2}$
        for all x $\not \in$ $A^+,A^-$ respectively.
        
        Let A = $A^+ \cup A^-$ and g(x) = $g^+ + g^-$.
        Thus, for any x $\not \in$ A:

        \hspace{0.5cm}
        $|f(x) - g(x)|$
        $\leq$ $|f^+(x) - g^+(x)| + |f^-(x) - g^-(x)|$
        $<$ $\frac{\epsilon}{2} + \frac{\epsilon}{2}$
        = $\epsilon$

        \rule[0.1cm]{15.2cm}{0.01cm}

        If $|f(x)|$ $\leq$ M, take g from before and let
        $g_1(x)$ =
        $\begin{cases}
            M & g(x) > M \\
            g(x) & g(x) \in [-M,M] \\
            -M & g(x) < -M
        \end{cases}$.

        Thus, step function $g_1$ is $|g_1|$ $\leq$ M
        where $g_1(x)$ = g(x) for $|g(x)|$ $\leq$ M. For x $\not \in A$:

        \hspace{0.2cm}
        If g(x) $>$ M:
        \hspace{0.5cm}
        f(x) $\leq$ M = $g_1(x)$ $<$ g(x)
        \hspace{0.2cm}
        $\Rightarrow$
        \hspace{0.2cm}
        $|f(x) - g_1(x)| < |f(x) - g(x)| < \epsilon$

        \hspace{0.25cm}
        If g(x) $<$ -M:
        \hspace{0.4cm}
        f(x) $\geq$ -M = $g_1(x)$ $>$ g(x)
        \hspace{0.2cm}
        $\Rightarrow$
        \hspace{0.2cm}
        $|f(x) - g_1(x)| < |f(x) - g(x)| < \epsilon$
    \end{proof}

    \newpage



    \begin{wtheorem}{Properties of the Lebesgue Integral}{14cm}
        If f,g are Lebesgue integrable functions. Then:
    \end{wtheorem}

    \begin{enumerate}[label=(\alph*), leftmargin=2cm, itemsep=0.1cm]
        \item {\color{lblue} Linearity}:
            If $c_1,c_2$ $\in$ $\mathbb{R}$:

            \hspace{0.5cm}
            $\int$ $c_1f + c_2g$ $d\mu$
            = $c_1 \int$ f $d\mu$ + $c_2 \int$ g $d\mu$

        \item {\color{lblue} Monotonicity}:
            If f(x) $\leq$ g(x):

            \hspace{0.5cm}
            $\int$ f $d\mu$ $\leq$ $\int$ g $d\mu$

        \item {\color{lblue} Absolute Value}:
            $|f|$ is integrable where:

            \hspace{0.5cm}
            $|\int$ f $d\mu|$
            $\leq$ $\int$ $|f|$ $d\mu$

        \item {\color{lblue} Null Sets}:
            If f(x) = g(x) except on a set of measure zero,
            then if f is integrable, then g is integrable where:

            \hspace{0.5cm}
            $\int$ f $d\mu$ = $\int$ g $d\mu$
    \end{enumerate}