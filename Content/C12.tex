\newpage

\section[Day 12: Series \& Comparison Test]{Series and Comparison Test}

\subsection{ Series }

{ \color{blue} Definition 12.1.1: Series }

    \begin{adjustbox}{minipage=14cm, right, vspace=0.1cm 0cm}
        For sequence \{$a_n$\}, define $\sum_{n=p}^q$ $a_n$
        = $a_p + a_{p+1} + ... + a_q$.

        Then associate \{$a_n$\} with a sequence \{$s_n$\}
        such that $s_n$ = $\sum_{k=1}^n$ $a_k$.

        Then \{$s_n$\} is a series with partial sums $s_n$.

        If \{$s_n$\} $\rightarrow$ s, then
        $\sum_{n=1}^{\infty}$ $a_n$ = s
        is the sum of the convergent series.
        
        Note $a_1$ = $s_1$ and $a_n$ = $s_n$ - $s_{n-1}$. \\
    \end{adjustbox}

{ \color{red} Theorem 12.1.2: Cauchy Criterion for series }

    \begin{adjustbox}{minipage=14cm, right, vspace=0.1cm 0cm}
        $\sum$ $a_n$ converges if and only if:
        
        \hspace{0.5cm}
        For every $\epsilon$ $>$ 0, there is a N $\in$ $\mathbb{Z}$
        such that for m $\geq$ n $\geq$ N,
        $| \sum_{k=n}^m a_k |$ $\leq$ $\epsilon$
    \end{adjustbox}

{ \color{magenta} \underline{Proof} }

    \fbox{
    \begin{minipage}{15cm}
        Suppose $\sum_{k=1}^n a_k$ converges.

        Then by {\color{red} theorem 10.3.3a}, $\sum_{k=1}^n a_k$ is 
        a Cauchy sequence.

        Then for $\epsilon$ $>$ 0, there is a N such that for
        m $\geq$ n $\geq$ N:

        \hspace{1cm}
        d($\sum_{k=1}^n a_k$ , $\sum_{k=1}^m a_k$)
        = $| \sum_{k=1}^m a_k - \sum_{k=1}^n a_k |$
        = $| \sum_{k=n}^m a_k |$ $\leq$ $\epsilon$

        \vspace{0.2cm}

        Suppose for every $\epsilon$ $>$ 0, there is a N
        such that for m $\geq$ n $\geq$ N,
        $| \sum_{k=n}^m a_k |$ $\leq$ $\epsilon$.

        \hspace{1cm}
        $| \sum_{k=n}^m a_k |$
        = $| \sum_{k=1}^m a_k - \sum_{k=1}^n a_k |$
        = d($\sum_{k=1}^n a_k$ , $\sum_{k=1}^m a_k$) $\leq$ $\epsilon$

        Thus, $\sum_{k=1}^n a_k$ is a Cauchy sequence
        and thus, convergent.
    \end{minipage} }

    \vspace{0.5cm}

{ \color{red} Theorem 12.1.3:
Convergent $\sum a_n$ $\Rightarrow$ \{$a_n$\} $\rightarrow$ 0 }

    \begin{adjustbox}{minipage=14cm, right, vspace=0.1cm 0cm}
        If $\sum$ $a_n$ converges, then
        $\lim_{n \rightarrow \infty}$ $a_n$ = 0.
    \end{adjustbox}

{ \color{magenta} \underline{Proof} }

    \fbox{
    \begin{minipage}{15cm}
        Since $\sum$ $a_n$ converges, then by {\color{red} theorem 12.1.2},
        for $\epsilon$ $>$ 0, there is a N such that for m $\geq$ n $\geq$ N,
        $| \sum_{k=n}^m a_k |$ $\leq$ $\epsilon$.
        Then if m = n $\geq$ N, $| \sum_{k=n}^m a_k |$ = $| a_n |$
        $\leq$ $\epsilon$ so \{$a_n$\} $\rightarrow$ 0.
    \end{minipage} }

    \vspace{0.5cm}

{ \color{purple} Example 12.1.4:
\{$a_n$\} $\rightarrow$ 0 $\not \Rightarrow$ Convergent $\sum a_n$ }

    \begin{adjustbox}{minipage=14cm, right, vspace=0.1cm 0cm}
        $\sum_{n=1}^{\infty}$ $\frac{1}{n}$ diverges.
    \end{adjustbox}

{ \color{magenta} \underline{Proof} }

    \fbox{
    \begin{minipage}{15cm}
        $\sum_{n=1}^{\infty}$ $\frac{1}{n}$
        = 1 + $\frac{1}{2}$ + ($\frac{1}{3}$ + $\frac{1}{4}$)
        + ($\frac{1}{5}$ + ... + $\frac{1}{8}$) + ($\frac{1}{9}$ + ...)
        $\geq$ 1 + $\frac{1}{2}$ + $\frac{1}{2}$ + $\frac{1}{2}$ + ...

        Thus, $s_{2^k}$ = $\sum_{n=1}^{2^k}$ $a_n$
        $\geq$ 1 + $k \cdot \frac{1}{2}$
        which is unbounded and thus, not convergent.
    \end{minipage} }
    
    \vspace{0.5cm}

{ \color{red} Theorem 12.1.5:
Convergent series $\rightleftharpoons$ Bounded sequence }

    \begin{adjustbox}{minipage=14cm, right, vspace=0.1cm 0cm}
        A series of nonnegative terms converge if and only if
        its partial sums form a bounded sequence.
    \end{adjustbox}

{ \color{magenta} \underline{Proof} }

    \fbox{
    \begin{minipage}{15cm}
        Suppose $\sum$ $a_n$ converges where $a_n$ $\geq$ 0.

        Since $a_n$ $\geq$ 0, then \{$s_n$\} is monotonic so
        by {\color{red} theorem 10.3.6}, \{$s_n$\} is bounded above.

        \vspace{0.2cm}

        Suppose \{$s_n$\} is bounded where $a_n$ $\geq$ 0.

        Since \{$s_n$\} is monotonic and bounded, then
        by {\color{red} theorem 10.3.6}, \{$s_n$\} converges.     
    \end{minipage} }

\newpage    

{ \color{red} Theorem 12.1.6: Comparison Test }

    \begin{enumerate}[label=(\alph*), leftmargin=2cm, itemsep=0.1cm]
        \item If $|a_n|$ $\leq$ $c_n$ for n $\geq$ $N_0$ and
        $\sum$ $c_n$ converges, then $\sum$ $a_n$ converges.

            { \color{magenta} \underline{Proof} }

                \fbox{
                \begin{minipage}{13cm}
                    For $\epsilon$ $>$ 0, there exists a N $\geq$ $N_0$
                    such that for m $\geq$ n $\geq$ N,
                    $\sum_{k=n}^m$ $c_k$ $\leq$ $\epsilon$.

                    \hspace{1cm}
                    $| \sum_{k=n}^m a_k |$
                    $\leq$ $\sum_{k=n}^m |a_k|$
                    $\leq$ $\sum_{k=n}^m c_k$ $\leq$ $\epsilon$

                    Thus, $\sum$ $a_n$ converges.
                \end{minipage} }
        
        \item If $a_n$ $\geq$ $d_n$ $\geq$ 0 for n $\geq$ $N_0$
        and $\sum$ $d_n$ diverges, then $\sum$ $a_n$ diverges.

            { \color{magenta} \underline{Proof} }

                \fbox{
                \begin{minipage}{13cm}
                    Suppose $\sum$ $a_n$ converges.

                    Then from part a, $\sum$ $d_n$ converges which contradicts
                    that $\sum$ $a_n$ diverges.
                    
                    Thus, $\sum$ $a_n$ diverges. 
                \end{minipage} }
    \end{enumerate}





\subsection{ Series of Nonnegative Terms }

{ \color{red} Theorem 12.2.1: Infinite Geometric Series }

    \begin{adjustbox}{minipage=14cm, right, vspace=0.1cm 0cm}
        If x $\in$ [0,1), then:

        \hspace{1cm}
        $\sum_{n=0}^{\infty}$ $x^n$
        = {\large $\frac{1}{1-x}$ }

        If x $\geq$ 1, the series diverges.
    \end{adjustbox}

{ \color{magenta} \underline{Proof} }

    \fbox{
    \begin{minipage}{15cm}
        If x $\not =$ 1, then using the geometric series:

        \hspace{1cm}
        $s_n$ = $\sum_{k=0}^n$ $x^k$ = $\frac{1 - x^{n+1}}{1 - x}$

        Let n $\rightarrow$ $\infty$.

        If x $\in$ [0,1), then by {\color{red} theorem 11.2.1e},
        $s_n$ = $\frac{1}{1-x}$ $(1 - x^{n+1})$
        = $\frac{1}{1-x}$ (1 - 0) = $\frac{1}{1-x}$.

        Also, by {\color{red} theorem 11.2.1e}, if x $\geq$ 1,
        then the series diverges.
    \end{minipage} }

    \vspace{0.5cm}


{ \color{red} Theorem 12.2.2: Cauchy's Convergence Criterion }

    \begin{adjustbox}{minipage=14cm, right, vspace=0.1cm 0cm}
        Suppose 0 $\leq$ $a_{i+1}$ $\leq$ $a_{i}$.

        Then the series $\sum_{n=1}^{\infty}$ $a_n$ converges
        if and only if the series

        \hspace{1cm}
        $\sum_{k=0}^{\infty}$ $2^k a_{2^k}$
        = $a_1$ + $2a_2$ + $4a_4$ + $8a_8$ + ...
        converges.
    \end{adjustbox}

{ \color{magenta} \underline{Proof} }

    \fbox{
    \begin{minipage}{15cm}
        Let $s_n$ = $a_1 + a_2 + ... + a_n$ and
        $t_k$ = $a_1 + 2a_2 + ... + 2^ka_{2^k}$.

        For n $<$ $2^k$:

        \hspace{1cm}
        $s_n$
        $\leq$ $a_1 + a_2 + a_3 + a_4 + a_5 + a_6 + a_7 + ... + a_{2^k}$

        \hspace{1.55cm}
        $\leq$ $a_1 + (a_2 + a_3) + (a_4 + a_5 + a_6 + a_7)
                    + ... + (a_{2^k} + ... + a_{2^{k+1} - 1})$

        \hspace{1.55cm}
        $\leq$ $a_1 + 2a_2 + 4a_4 + ... + 2^ka_{2^k}$
        = $t_k$

        Thus, by the {\color{red} comparison test},
        if $\sum_{k=0}^{\infty}$ $2^k a_{2^k}$ converges, then
        $\sum_{n=1}^{\infty}$ $a_n$ converges.

        For n $>$ $2^k$:

        \hspace{1cm}
        $s_n$
        $\geq$ $a_1 + a_2 + a_3 + a_4 + a_5 + a_6 + a_7 + a_8 + ... + a_{2^k}$

        \hspace{1.55cm}
        = $a_1 + a_2 + (a_3 + a_4) + (a_5 + a_6 + a_7 + a_8)
                    + ... + (a_{2^{k-1} + 1} + ... + a_{2^k})$

        \hspace{1.55cm}
        $\geq$ $\frac{1}{2}a_1 + a_2 + 2a_4 + ... + 2^{k-1}a_{2^k}$
        = $\frac{1}{2}t_k$

        Thus, by the {\color{red} comparison test},
        if $\sum_{n=1}^{\infty}$ $a_n$ converges, then
        $\sum_{k=0}^{\infty}$ $2^k a_{2^k}$ converges.
    \end{minipage} }

\newpage

{ \color{red} Theorem 12.2.3: P-series }

    \begin{adjustbox}{minipage=14cm, right, vspace=0.1cm 0cm}
        $\sum$ $\frac{1}{n^p}$ converges if p $>$ 1 and diverges if p $\leq$ 1.
    \end{adjustbox}

{ \color{magenta} \underline{Proof} }

    \fbox{
    \begin{minipage}{15cm}
        If p $\leq$ 0, then by {\color{red} theorem 12.1.3},
        $\sum$ $\frac{1}{n^p}$ diverges.

        If p $>$ 0, then by {\color{red} theorem 12.2.2},
        $\sum$ $\frac{1}{n^p}$ converges only if
        $\sum_{k=0}^{\infty}$ $2^k \frac{1}{(2^k)^p}$ converges.

        Since $\sum_{k=0}^{\infty}$ $2^k \frac{1}{(2^k)^p}$
        = $\sum_{k=0}^{\infty}$ $2^{(1-p)k}$, then by {\color{red} theorem 12.2.1},
        $\sum_{k=0}^{\infty}$ $2^{k(1-p)}$ converges if
        $2^{1-p}$ $<$ 1 so if 1-p $<$ 0 so p $>$ 1.
    \end{minipage} }

    \vspace{0.5cm}

{ \color{red} Theorem 12.2.4: Log P-series }

    \begin{adjustbox}{minipage=14cm, right, vspace=0.1cm 0cm}
        $\sum_{n = 2}^{\infty}$ $\frac{1}{n(\log(n))^p}$
        converges if p $>$ 1 and diverges if p $\leq$ 1.
    \end{adjustbox}

{ \color{magenta} \underline{Proof} }

    \fbox{
    \begin{minipage}{15cm}
        Since $\frac{1}{n(\log(n))^p}$ decreases, then by {\color{red} theorem 12.2.2},

        $\sum_{n=0}^{\infty}$ $\frac{1}{n(\log(n))^p}$ converges if
        $\sum_{k=1}^{\infty}$ $2^k \frac{1}{2^k \log(2^k)}$ converges.
        
        \hspace{1cm}
        $\sum_{k=1}^{\infty}$ $2^k \frac{1}{2^k \log(2^k)}$
        = $\sum_{k=1}^{\infty}$ $\frac{1}{ k \log(2)}$
        = $\frac{1}{\log(2)}$ $\sum_{k=1}^{\infty}$ $\frac{1}{k}$

        Then by {\color{red} theorem 12.2.3},
        $\sum_{k=1}^{\infty}$ $2^k \frac{1}{2^k \log(2^k)}$ converges if
        p $>$ 1 and diverges if p $\leq$ 1.

        Thus, $\sum_{n=0}^{\infty}$ $\frac{1}{n(\log(n))^p}$ converges if
        p $>$ 1 and diverges and p $\leq$ 1.
    \end{minipage} }

    \vspace{0.5cm}

{ \color{orange} Corollary 12.2.5: Log P-series extended }

    \begin{adjustbox}{minipage=14cm, right, vspace=0.1cm 0cm}
        $\sum_{n=3}^{\infty}$ $\frac{1}{n\log(n)(\log(\log(n)))^p}$
        converges if p $>$ 1 and diverges if p $\leq$ 1.
    \end{adjustbox}

{ \color{magenta} \underline{Proof} }

    \fbox{
    \begin{minipage}{15cm}
        From {\color{red} theorem 12.2.4}, replace n = log(n) and
        multiplying by $\frac{1}{n}$ $\rightarrow$
        $\frac{1}{n\log(n)(\log(\log(n)))^p}$.

        Since $\frac{1}{n\log(n)(\log(\log(n)))^p}$ decreases,
        by {\color{red} theorem 12.2.2}
        $\sum_{k=1}^{\infty}$ $2^k \frac{1}{2^k \log(2^k)(\log(\log(2^k)))^p}$:
        
        \hspace{1cm}
        $\sum_{k=1}^{\infty}$ $\frac{1}{\log(2^k)(\log(\log(2^k)))^p}$
        = $\frac{1}{\log(2)} \sum_{k=1}^{\infty}$ $\frac{1}{k (\log(k \log(2)))^p}$
        $<$ $\frac{1}{\log(2)} \sum_{k=2}^{\infty}$ $\frac{1}{k (\log(k))^p}$

        Since $\sum_{k=2}^{\infty}$ $\frac{1}{k (\log(k))^p}$ converges
        by {\color{red} theorem 12.2.4},
        $\sum_{n=3}^{\infty}$ $\frac{1}{n\log(n)(\log(\log(n)))^p}$ converges.
    \end{minipage} }





\subsection{ The Number e}

{ \color{blue} Definition 12.3.1: Summation equivalence to e }

    \begin{adjustbox}{minipage=14cm, right, vspace=0.1cm 0cm}
        $s_m$
        = $\sum_{n=0}^{m}$ $\frac{1}{n!}$
        = 1 + $\sum_{n=1}^{m}$ $\frac{1}{n!}$
        $<$ 1 + $\sum_{n=1}^{m}$ $\frac{1}{2^{n-1}}$
        $<$ 3

        e = $\sum_{n=0}^{\infty}$ $\frac{1}{n!}$
    \end{adjustbox}

    \vspace{0.5cm}

{ \color{red} Theorem 12.3.2: Limit equivalence to e }

    \begin{adjustbox}{minipage=14cm, right, vspace=0.1cm 0cm}
        $\lim_{n \rightarrow \infty}$ $(1 + \frac{1}{n})^n$ = e
    \end{adjustbox}

{ \color{magenta} \underline{Proof} }

    \fbox{
    \begin{minipage}{15cm}
        Let $s_n$ = $\sum_{k=0}^n$ $\frac{1}{k!}$ and
        $t_n$ = $(1 + \frac{1}{n})^n$.
        Using the binomial theorem:

        \hspace{1cm}
        $t_n$ = $\sum_{k=0}^n$ $\binom{n}{k} \frac{1}{n^k}$
        = $\sum_{k=0}^n$ $\frac{n(n-1)...(n-k+1))}{k!} \frac{1}{n^k}$
        = $\sum_{k=0}^n$ $\frac{1}{k!}
                        (1) (1 - \frac{1}{n}) (1 - \frac{2}{n}) (1 - \frac{k-1}{n})$
        
        Thus, $t_n$ $\leq$ $s_n$ so
        $\lim_{n \rightarrow \infty}$ sup($t_n$) $\leq$ e.

        If n $\geq$ m, then
        $t_n$ $\geq$
        $\sum_{k=0}^m$ $\frac{1}{k!}
                        (1) (1 - \frac{1}{n}) (1 - \frac{2}{n}) (1 - \frac{k-1}{n})$.
        
        As n $\rightarrow$ $\infty$, then
        $\lim_{n \rightarrow \infty}$ inf($t_n$)
        $\geq$ $\sum_{k=0}^m$ $\frac{1}{k!}$
        = $s_m$.
        As m $\rightarrow$ $\infty$,
        $\lim_{n \rightarrow \infty}$ inf($t_n$) $\geq$ e.

    \end{minipage} }

    \vspace{0.5cm}

{ \color{blue} Definition 12.3.3: Rapidity of convergence of e }

    \begin{adjustbox}{minipage=14cm, right, vspace=0.1cm 0cm}
        0 $<$ e - $s_n$ $<$ $\frac{1}{n!n}$
    \end{adjustbox}

{ \color{magenta} \underline{Proof} }

    \fbox{
    \begin{minipage}{15cm}
        e - $s_n$
        = $\sum_{k=n+1}^{\infty}$ $\frac{1}{k!}$
        $<$ $\frac{1}{(n+1)!}$ $(1 + \frac{1}{n+1} + \frac{1}{(n+1)^2} + ...)$
        = $\frac{1}{(n+1)!}$ $\frac{1}{1 - \frac{1}{n+1}}$
        = $\frac{1}{n!n}$
    \end{minipage} }

\newpage

{ \color{red} Theorem 12.3.4: e is irrational }

    \begin{adjustbox}{minipage=14cm, right, vspace=0.1cm 0cm}
        e is irrational
    \end{adjustbox}

{ \color{magenta} \underline{Proof} }

    \fbox{
    \begin{minipage}{15cm}
        Suppose r is rational.
        Then let e = $\frac{p}{q}$ for p,q $\in$ $\mathbb{Z}_+$.

        Thus, by {\color{blue} definition 12.3.3},
        0 $<$ e - $s_q$ $<$ $\frac{1}{q!q}$ so
        0 $<$ q!(e - $s_q$) $<$ $\frac{1}{q}$.

        Since e = $\frac{p}{q}$, then q!e is an integer and
        q!$s_q$ = q!(1 + 1 + $\frac{1}{2!}$ + ... + $\frac{1}{q!}$)
        is an integer.
        
        Thus, q!(e - $s_q$) is an integer which is between 0 and $\frac{1}{q}$
        and thus, a contradiction.
    \end{minipage} }





\subsection{ Root and Ratio Tests }

{ \color{red} Theorem 12.4.1: Root Test }

    \hspace{1cm}
    For $\sum$ $a_n$, let
    $\alpha$ = $\lim_{n \rightarrow \infty}$ sup($\sqrt[n]{|a_n|}$).

    \vspace{0.1cm}

    \begin{enumerate}[label=(\alph*), leftmargin=3cm, itemsep=0.1cm]
        \item If $\alpha$ $<$ 1, $\sum$ $a_n$ converges
        
        \item If $\alpha$ $>$ 1, $\sum$ $a_n$ diverges
        
        \item If $\alpha$ = 1, unclear
    \end{enumerate}

{ \color{magenta} \underline{Proof} }

    \fbox{
    \begin{minipage}{15cm}
        If $\alpha$ $<$ 1, choose $\beta$ such that $\beta$ $\in$ ($\alpha$,1)
        and N $\in$ $\mathbb{Z}$ such that $\sqrt[n]{|a_n|}$ $<$ $\beta$ for
        n $\geq$ N.

        Since $\beta$ $\in$ (0,1), then by {\color{red} theorem 12.2.1},
        $\sum$ $\beta^n$ converges.
        Then by the {\color{red} comparison test}, $\sum$ $a_n$ converges.

        \vspace{0.2cm}

        If $\alpha$ $>$ 1, then there is a $a_{n_k}$ such that
        $\sqrt[n_k]{|a_{n_k}|}$ $\rightarrow$ $\alpha$.

        Thus, $|a_n|$ $>$ 1 for infinitely many n so 
        by {\color{red} theorem 12.1.3}, $\sum$ $a_n$ doesn't converge.

        \vspace{0.2cm}

        For $\alpha$ = 1, both $\sum$ $\frac{1}{n}$ and $\sum$ $\frac{1}{n^2}$
        have $\alpha$ = 1, but $\sum$ $\frac{1}{n}$ diverges and
        $\sum$ $\frac{1}{n^2}$ converges by {\color{red} theorem 12.2.3}.
    \end{minipage} }

    \vspace{0.5cm}

{ \color{red} Theorem 12.4.2: Ratio Test }

    \begin{enumerate}[label=(\alph*), leftmargin=2cm, itemsep=0.1cm]
        \item $\sum$ $a_n$ converges if $\lim_{n \rightarrow \infty}$
        sup($|\frac{a_{n+1}}{a_n}|$) $<$ 1
        
        \item $\sum$ $a_n$ diverges if $|\frac{a_{n+1}}{a_n}|$ $\geq$ 1
        for all n $\geq$ $n_0$ for $n_0$ $\in$ $\mathbb{Z}$
    \end{enumerate}

{ \color{magenta} \underline{Proof} }

    \fbox{
    \begin{minipage}{15cm}
        If $\lim_{n \rightarrow \infty}$ sup($|\frac{a_{n+1}}{a_n}|$) $<$ 1,
        there is a $\beta$ $<$ 1 and N such that for n $\geq$ N,
        $|\frac{a_{n+1}}{a_n}|$ $<$ $\beta$.

        Then $|a_{N+1}|$ $<$ $\beta |a_N|$ so
        $|a_{N+2}|$ $<$ $\beta$$|a_{N+1}|$ $<$ $\beta^2 |a_N|$.
        
        Thus, $|a_{N+p}|$ $<$ $\beta^p$$|a_N|$ so
        $|a_n|$ $<$ $|a_N| \beta^{-N} \beta^n$.

        Thus, by the {\color{red} comparison test}, $\sum$ $a_n$ converges.

        If $|a_{n+1}|$ $\geq$ $|a_n|$ $>$ 0 for n $\geq$ $n_0$, then by
        {\color{red} theorem 12.1.3}, $\sum$ $a_n$ diverges.
    \end{minipage} }

    \vspace{0.5cm}

{ \color{red} Theorem 12.4.3:
Ratio convergence $\rightarrow$ Root convergence }

    \begin{adjustbox}{minipage=14cm, right, vspace=0.1cm 0cm}
        $\lim_{n \rightarrow \infty}$ inf($\frac{c_{n+1}}{c_n}$)
        $\leq$ $\lim_{n \rightarrow \infty}$ inf($\sqrt[n]{c_n}$)

        $\lim_{n \rightarrow \infty}$ sup($\sqrt[n]{c_n}$)
        $\leq$ $\lim_{n \rightarrow \infty}$ sup($\frac{c_{n+1}}{c_n}$)
    \end{adjustbox}

{ \color{magenta} \underline{Proof} }

    \fbox{
    \begin{minipage}{15cm}
        Let $\alpha$ = $\lim_{n \rightarrow \infty}$ inf($\frac{c_{n+1}}{c_n}$).
        If $\alpha$ = $-\infty$, then $-\infty$ $\leq$
        $\lim_{n \rightarrow \infty}$ inf($\sqrt[n]{c_n}$) holds true.

        If $\alpha$ is finite, there is a $\beta$ $\leq$ $\alpha$ and N such that
        for n $\geq$ N, $\frac{c_{n+1}}{c_n}$ $\geq$ $\beta$
        so $c_{N+p}$ $\geq$ $\beta^p c_N$.

        Then, $c_{n}$ $\geq$ $c_N \beta^{-N} \beta^n$ so
        $\sqrt[n]{c_{n}}$ $\geq$ $\sqrt[n]{c_N \beta^{-N}} \beta$.
        Thus, $\lim_{n \rightarrow \infty}$ inf($\sqrt[n]{c_n}$)
        $\geq$ $\beta$ = $\alpha$.

        \vspace{0.2cm}

        Let $\alpha$ = $\lim_{n \rightarrow \infty}$ sup($\frac{c_{n+1}}{c_n}$).
        If $\alpha$ = $\infty$, then
        $\lim_{n \rightarrow \infty}$ sup($\sqrt[n]{c_n}$)
        $\leq$ $\infty$ holds true.

        If $\alpha$ is finite, there is a $\beta$ $\geq$ $\alpha$ and N such that
        for n $\geq$ N, $\frac{c_{n+1}}{c_n}$ $\leq$ $\beta$
        so $c_{N+p}$ $\leq$ $\beta^p c_N$.

        Then, $c_{n}$ $\leq$ $c_N \beta^{-N} \beta^n$ so
        $\sqrt[n]{c_{n}}$ $\leq$ $\sqrt[n]{c_N \beta^{-N}} \beta$.
        Thus, $\lim_{n \rightarrow \infty}$ sup($\sqrt[n]{c_n}$)
        $\leq$ $\beta$ = $\alpha$.
    \end{minipage} }




