\newpage

\section[Series \& Comparison Test]{Series and Comparison Test}

\subsection{Series}

{ \color{blue} Definition 12.1.1: Series }

    \begin{adjustbox}{minipage=14cm, right, vspace=0.1cm 0cm}
        For sequence \{$a_n$\}, define $\sum_{n=p}^q$ $a_n$
        = $a_p + a_{p+1} + ... + a_q$.

        Then associate \{$a_n$\} with a sequence \{$s_n$\}
        such that $s_n$ = $\sum_{k=1}^n$ $a_k$.

        Then \{$s_n$\} is a series with partial sums $s_n$.

        If \{$s_n$\} $\rightarrow$ s, then
        $\sum_{n=1}^{\infty}$ $a_n$ = s
        is the sum of the convergent series.
        
        Note $a_1$ = $s_1$ and $a_n$ = $s_n$ - $s_{n-1}$. \\
    \end{adjustbox}

{ \color{red} Theorem 12.1.2: Cauchy Criterion Redefined }

    \begin{adjustbox}{minipage=14cm, right, vspace=0.1cm 0cm}
        $\sum$ $a_n$ converges if and only if:
        
        \hspace{0.5cm}
        For every $\epsilon$ $>$ 0, there is a N $\in$ $\mathbb{Z}$
        such that for m $\geq$ n $\geq$ N,
        $| \sum_{k=n}^m a_k |$ $\leq$ $\epsilon$
    \end{adjustbox}

{ \color{magenta} \underline{Proof} }

    \fbox{
    \begin{minipage}{15cm}
        Suppose $\sum_{k=1}^n a_k$ converges.

        Then by {\color{red} theorem 10.3.3a}, $\sum_{k=1}^n a_k$ is 
        a Cauchy sequence.

        Then for $\epsilon$ $>$ 0, there is a N such that for
        m $\geq$ n $\geq$ N:

        \hspace{1cm}
        d($\sum_{k=1}^n a_k$ , $\sum_{k=1}^m a_k$)
        = $| \sum_{k=1}^m a_k - \sum_{k=1}^n a_k |$
        = $| \sum_{k=n}^m a_k |$ $\leq$ $\epsilon$

        Suppose for every $\epsilon$ $>$ 0, there is a N
        such that for m $\geq$ n $\geq$ N,
        $| \sum_{k=n}^m a_k |$ $\leq$ $\epsilon$.

        \hspace{1cm}
        $| \sum_{k=n}^m a_k |$
        = $| \sum_{k=1}^m a_k - \sum_{k=1}^n a_k |$
        = d($\sum_{k=1}^n a_k$ , $\sum_{k=1}^m a_k$) $\leq$ $\epsilon$

        Thus, $\sum_{k=1}^n a_k$ is a Cauchy sequence
        and thus, convergent.
    \end{minipage} }

    \vspace{0.5cm}

{ \color{red} Theorem 12.1.3:
Convergent $\sum a_n$ $\Rightarrow$ \{$a_n$\} $\rightarrow$ 0 }

    \begin{adjustbox}{minipage=14cm, right, vspace=0.1cm 0cm}
        If $\sum$ $a_n$ converges, then
        $\lim_{n \rightarrow \infty}$ $a_n$ = 0.
    \end{adjustbox}

{ \color{magenta} \underline{Proof} }

    \fbox{
    \begin{minipage}{15cm}
        Since $\sum$ $a_n$ converges, then by {\color{red} theorem 12.1.2},
        for $\epsilon$ $>$ 0, there is a N such that for m $\geq$ n $\geq$ N,
        $| \sum_{k=n}^m a_k |$ $\leq$ $\epsilon$.
        Then if m = n $\geq$ N, $| \sum_{k=n}^m a_k |$ = $| a_n |$
        $\leq$ $\epsilon$ so \{$a_n$\} $\rightarrow$ 0.
    \end{minipage} }

    \vspace{0.5cm}

{ \color{purple} Example 12.1.4:
\{$a_n$\} $\rightarrow$ 0 $\not \Rightarrow$ Convergent $\sum a_n$ }

    \begin{adjustbox}{minipage=14cm, right, vspace=0.1cm 0cm}
        $\sum_{n=1}^{\infty}$ $\frac{1}{n}$ diverges
    \end{adjustbox}

{ \color{magenta} \underline{Proof} }

    \fbox{
    \begin{minipage}{15cm}
        $\sum_{n=1}^{\infty}$ $\frac{1}{n}$
        = 1 + $\frac{1}{2}$ + ($\frac{1}{3}$ + $\frac{1}{4}$)
        + ($\frac{1}{5}$ + ... + $\frac{1}{8}$) + ($\frac{1}{9}$ + ...)
        $\geq$ 1 + $\frac{1}{2}$ + $\frac{1}{2}$ + $\frac{1}{2}$ + ...

        Thus, $s_{2^k}$ = $\sum_{n=1}^{2^k}$ $a_n$
        $\geq$ 1 + $k \cdot \frac{1}{2}$
        which is unbounded and thus, not convergent.
    \end{minipage} }
    
    \vspace{0.5cm}

{ \color{red} Theorem 12.1.5:
Convergent series $\rightleftharpoons$ Bounded sequence }

    \begin{adjustbox}{minipage=14cm, right, vspace=0.1cm 0cm}
        A series of nonnegative terms converge if and only if
        its partial sums form a bounded sequence.
    \end{adjustbox}

{ \color{magenta} \underline{Proof} }

    \fbox{
    \begin{minipage}{15cm}
        Suppose $\sum$ $a_n$ converges where $a_n$ $\geq$ 0.

        Since $a_n$ $\geq$ 0, then \{$s_n$\} is monotonic so
        by {\color{red} theorem 10.3.6}, \{$s_n$\} is bounded above.

        Suppose \{$s_n$\} is bounded where $a_n$ $\geq$ 0.

        Since \{$s_n$\} is monotonic and bounded, then
        by {\color{red} theorem 10.3.6}, \{$s_n$\} converges.     
    \end{minipage} }

\newpage    

{ \color{red} Theorem 12.1.6: Comparison Test }

    \begin{enumerate}[label=(\alph*), leftmargin=2cm, itemsep=0.1cm]
        \item If $|a_n|$ $\leq$ $c_n$ for n $\geq$ $N_0$ and
        $\sum$ $c_n$ converges, then $\sum$ $a_n$ converges.

            { \color{magenta} \underline{Proof} }

                \fbox{
                \begin{minipage}{13cm}
                    For $\epsilon$ $>$ 0, there exists a N $\geq$ $N_0$
                    such that for m $\geq$ n $\geq$ N,
                    $\sum_{k=n}^m$ $c_k$ $\leq$ $\epsilon$.

                    \hspace{1cm}
                    $| \sum_{k=n}^m a_k |$
                    $\leq$ $\sum_{k=n}^m |a_k|$
                    $\leq$ $\sum_{k=n}^m c_k$ $\leq$ $\epsilon$

                    Thus, $\sum$ $a_n$ converges.
                \end{minipage} }
        
        \item If $a_n$ $\geq$ $d_n$ $\geq$ 0 for n $\geq$ $N_0$
        and $\sum$ $d_n$ diverges, then $\sum$ $a_n$ diverges.

            { \color{magenta} \underline{Proof} }

                \fbox{
                \begin{minipage}{13cm}
                    Suppose $\sum$ $a_n$ converges.

                    Then from part a, $\sum$ $d_n$ converges which contradicts
                    that $\sum$ $a_n$ diverges.
                    
                    Thus, $\sum$ $a_n$ diverges. 
                \end{minipage} }
    \end{enumerate}





\subsection{ Series of Nonnegative Terms }

{ \color{red} Theorem 12.2.1: Infinite Geometric Series }

    \begin{adjustbox}{minipage=14cm, right, vspace=0.1cm 0cm}
        If x $\in$ [0,1), then:

        \hspace{1cm}
        $\sum_{n=0}^{\infty}$ $x^n$
        = {\large $\frac{1}{1-x}$ }

        If x $\geq$ 1, the series diverges.
    \end{adjustbox}

{ \color{magenta} \underline{Proof} }

    \fbox{
    \begin{minipage}{15cm}
        If x $\not =$ 1, then using the geometric series:

        \hspace{1cm}
        $s_n$ = $\sum_{k=0}^n$ $x^k$ = $\frac{1 - x^{n+1}}{1 - x}$

        Let n $\rightarrow$ $\infty$.

        If x $\in$ [0,1), then by {\color{red} theorem 11.2.1e},
        $s_n$ = $\frac{1}{1-x}$ $(1 - x^{n+1})$
        = $\frac{1}{1-x}$ (1 - 0) = $\frac{1}{1-x}$.

        Also, by {\color{red} theorem 11.2.1e}, if x $\geq$ 1,
        then the series diverges.
    \end{minipage} }










