\newpage
\section[Day 1: The Real Number System]{The Real Number System}

$ \mathbb{N} = 1, 2, 3, ... $ \\
$ \mathnn{Z} = -2, -1, 0, 1, 2 $ \\
$ \mathbb{Q} = p/q $ where Q is countable and fails to have the least upper bound property\\

Ex 1.1 \\
$ \alpha \in \mathbb{R} $ where $ \alpha^2 = 2 $ \\
Let $ \alpha = p/q $ where p and q cannot both be even \\

$
Let set A = \{ x \in \mathbb{Q} \ for x^2 < 2 } where A is nonempty and 2is an upper bound for A \\
$ \\

A has no least upper bound in R \\
A has a least upper bound in Q \\
\hfill \\

$ \mathbb{R} $ is the unique ordered field withthe least upper bound property. \\
Theorem: $ \mathbb{R} $ exists and unique. \\

Def 1.5 \\
Let S be a set. An order on S is a relation < satisfying 2 axioms:
\begin{itemize}
	\item Trichotomy: For all x,y \in S:
		\begin{itemize}
			\item x $<$ y
			\item x = y
			\item x $>$ y
		\end{itemize}
	\item Transitivity: If x $<$ y and y $<$ z, then x $<$ z.
\end{itemize} \\
\hfill \\

Def 1.5 \\
An ordered set is a set with an order.


\hfill \\
Def 1.7 \\
Let S be an ordered set. Let E \subset S. \\
An upper bound of E is a $ \beta \in S $ where x \<= \beta for all x \in E. \\\
If such a \beta exists, then E is bounded from above. \\

\hfill \\
Def 1.8 \\
Let S be an ordered set. Let E \subset S be bounded from above. \\
here exists an \alpha where:
\begin{itemize}
	\item \alpha is an upper bound for E
	\item if x \< \alpha, then x is not an upper bound for E
\end{itemize} \\
Then \alpha = sup(E). \\

\hfill \\
*** Lower Bound in textbook known as an inf(E) *** \\
Example 1.9 \\
Let S = $ (1,2) \cup [3,4) \cup (5,6) $ with an order $ < $ from $ \mathbb{R} $. \\
For subsets E of S \\
\begin{itemize}
	\item E = (1,2) is bound above and sup(E) = 3
	\item E = (5,6) is not bounded above so sup(E) = DNE
	\item E = [3,4) is bounded below inf(E) = 3 and sup(E) = DNE
\end{itemize} \\

\hfill \\
Observations \\
If sup E exists, it may or may not exists at E. \\
If \alpha exists, then \alpha is unique.
If $ \gamma \neq \alpha $, then $ \gamma < \alpha $ or $ \gamma > \alpha $ \\

\hfill \\
Def 1.10 \\
A is an ordered set of S. \\
A has a least upper bound property if:
\begin{itemize}
	\item For every nonempty subset E \subset S that is bounded from above, sup(E) exists in S.
\end{itemize} \\

\hill \\
Example 1.1 \\
$ \mathbb{Q} $ doesn't have a least upper bound property for example $ \sqrt(2) $. \\
Proof is at textbook.

$ z = y - \frac{y^2-2}{y+2} then take z^2-2 = 2 \frac{2(y^2-2)}{(y+2)^2} $ \\
Then compare if $y^2 < 2$ and $y^2 > 2$. \\
\begin{itemize}
	\item y^2 < 2 is not an upper bound for E
	\item y^2 > 2 is an upper bound for E, but not a sup(E).
\end{itemize} \\
ThusE has no least upper bound in \mathbb{Q}. \\
However, since E \subsetof R then \sqrt(2) is in E.










