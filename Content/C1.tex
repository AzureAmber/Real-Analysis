\newpage
\section[Day 1: The Real Number System]{The Real Number System}





\subsection{Number Systems}

\begin{align*}
	\text{Natural}:\ & \mathbb{N} = \{1, 2, 3, ... \} \\	
	\text{Integer}:\ & \mathbb{Z} = \{-2, -1, 0, 1, 2, ... \} \\
	\text{Rational}:\ & \mathbb{Q} = \frac{p}{q} \ \text{where p,q} \in \mathbb{N}
\end{align*}
*** Q is countable, but fails to have the least upper bound property *** \\

{ \color{purple} Example 1.1.1 }

\qquad Let $ \alpha \in \mathbb{R} $ where $ \alpha^2 = 2 $. Then $ \alpha $ cannot be rational.

{ \color{magenta} \underline{Proof} }

Let $ \alpha = \frac{p}{q} $ where p and q cannot both be even.

Let set A = $\{ x \in \mathbb{Q} \ \text{for} \ x^2 < 2 \} $ where A $ \neq \emptyset $
and 2 is an upper bound for A.

Then, A has no least upper bound in $ \mathbb{Q} $ , but A has a least upper bound in $ \mathbb{R} $.





\subsection{Real Number System}

$ \mathbb{R} $ is the unique ordered field with the least upper bound property. \\
Also, $ \mathbb{R} $ exists and unique. \\

{ \color{blue} Definition 1.2.1: Order }

\qquad Let S be a set. An order on S is a relation $<$ satisfying two axioms:

\begin{itemize}[leftmargin=2cm]
	\item { \color{lblue} Trichotomy}: For all x,y $ \in $ S, only one holds true:
		\begin{itemize}[leftmargin=1cm]
			\item x $<$ y
			\item x = y
			\item x $>$ y
		\end{itemize}
	
	\item { \color{lblue} Transitivity}: If x $<$ y and y $<$ z, then x $<$ z.
\end{itemize}

{ \color{blue} Definition 1.2.2: Ordered Set }

\qquad An ordered set is a set with an order. \\

{ \color{blue} Definition 1.2.3: Bounds }

\qquad Let S be an ordered set and E $ \subset $ S.

\qquad An upper bound of E is a $ \beta \in $ S if x $ \leq \beta $ for all x $ \in $ E.

\qquad \qquad If such a $ \beta $ exists, then E is bounded from above.

\qquad A lower bound of E is a $\alpha$ $\in$ S if x $ \geq $ $\alpha$ for all x $\in$ E.

\qquad \qquad If such a $\alpha$ exists, then E is bounded from below.

\newpage

{ \color{blue} Definition 1.2.4: Infimum \& Supremum}

\qquad Let S be an ordered set.

\qquad Let E $ \subset $ S be bounded from above. Least upper bound $\beta$ $\in$ S exists if:
\begin{itemize}[leftmargin=2cm]
	\item $\beta$ is an upper bound for E
	
	\item If $\gamma < \beta$, then $ \gamma $ is not an upper bound for E.
\end{itemize}

\qquad \qquad Then $\beta$ = sup(E). \\

\qquad Let E $\subset$ S be bounded from below. Greatest lower bound $\alpha$ $\in$ S exists if:
\begin{itemize}[leftmargin=2cm]
	\item $\alpha$ is a lower bound for E

	\item If $\gamma > \alpha$, then $\gamma$ is not a lower bound for E.
\end{itemize}

\qquad \qquad Then $\alpha$ = inf(E). \\

{ \color{purple} Example 1.2.5 }

\qquad Let S = $ (1,2) \cup [3,4) \cup (5,6) $ with the order $ < $ from $ \mathbb{R} $.
For subsets E of S:

\begin{itemize}[leftmargin=2cm]
	\item E = (1,2) is bounded above and sup(E) = 3
	
	\item E = (5,6) is not bounded above so sup(E) = DNE
	
	\item E = [3,4) is bounded below inf(E) = 3 and sup(E) = DNE
\end{itemize}

{ \color{green} Observations on the Least Upper Bound }

\qquad If sup(E) exists, it may or may not exists at S.

\qquad If sup(E)  exists, then sup(E) is unique.
If $ \gamma \neq \alpha $, then $ \gamma < \alpha $ or $ \gamma > \alpha $. \\





\subsection{Least Upper Bound Property}

{ \color{red} Theorem 1.3.1 }

\qquad An ordered set S has a least upper bound property if:

\qquad \qquad For every nonempty subset E $ \subset $ S that is bounded from above:

\qquad \qquad \qquad sup(E) exists in S. \\

{ \color{purple} Example 1.3.2 }

\qquad $ \mathbb{Q} $ doesn't have a least upper bound property. For example, z = $ \sqrt{2} $.

{\color{magenta} \underline{Proof}}

Let $ z = y - \frac{y^2-2}{y+2} = \frac{2y+2}{y+2} $, then take $ z^2-2 = \frac{2(y^2-2)}{(y+2)^2} $.

Let set A = $ \{ y > 0 \in \mathbb{Q} \ \text{where} \ y^2 < 2 \} $ and
set B = $ \{ y > 0 \in \mathbb{Q} \ \text{where} \ y^2 > 2 \} $

\begin{itemize}[leftmargin=1cm]
	\item If $ y^2-2 < $ 0, then z $>$ y where z $\in$ A.
		So, y is not a upper bound.

		Since for any y, there is z $>$ y where z $\in$ A, then sup(A) doesn't
		exists in $\mathbb{Q}$.
	
	\item If $ y^2-2 > $ 0, then z $<$ y where z $\in$ B.
		So, y is an upper bound, but not sup(E).

		Since for any y, there is z $<$ y where z $\in$ B, then inf(B) doesn't
		exists in $\mathbb{Q}$.
\end{itemize}

Thus, $\mathbb{Q}$ doesn't have the least upper bound.
