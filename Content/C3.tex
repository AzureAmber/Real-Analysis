\newpage

\section[Day 3]{Temp}

\subsection{nth Root}

If 0 $<$ t $<$ 1, then t$^\text{n}$ $ \leq $ t.

If t $>$ 1, t$^\text{n}$ $ \geq $ t.

If 0 $<$ s $<$t, s $^\text{n}$ $<$ t$^\text{n}$. \\

{ \color{blue} Theorem 3.1.1  }

\qquad Fix n. For every x $>$ 0, there exists a unique y $\in$ $ \mathbb{R} $ such that
	y$^\text{n}$ = x.

{ \color{magenta} \underline{Proof} } 

Uniqueness: y is unique since if y$_{1}$ $<$ y$_{2}$, then y$_{1}$$^\text{n}$ $<$ y$_{2}$$^\text{n}$.

Existence:

Let set A = \{ t $>$ 0 : t $^\text{n}$ $<$ x \}

A $\not =$ $\emptyset$ since let t$_{1}$ = $\frac{x}{x+1}$ $<$ 1 and $<$ x.

Thus, 0 $<$ t$_{1}$$^\text{n}$ $ \leq $ t$_{1}$ $<$ x so t$_{1}$ $\in$ A.

A is bounded above since t $ \geq $ x+1.

Then t $>$ 1 so t$^\text{n}$ $ \geq $ t $ \geq $ x+1 $>$ x so t $\not \in$ A.

Thus, x+1 is an upper bound of A.

By the least upper bound property, then y = sup(A).

For y$^\text{n}$ = x, show y$^\text{n}$ $<$ x and y$^\text{n}$ $>$ x cannot hold true.

***(Not an upper bound of A and y is not a least upper bound of A)

For 0 $<$ $\alpha$ $<$ $\beta$:

\qquad $\beta ^\text{n}$ - $\alpha^\text{n}$ $<$ ($\beta - \alpha$)n$\beta$$^\text{n-1}$ 
{<\color{red} Insert Proof>}

Suppose y$^\text{n}$ $<$ x. Pick 0 $<$ h $<$ 1 which contradicts the previous statement.

From inequality, let $\beta$ = y+h and $\alpha$ = y. <Insert Proof>

Thus, (y+h)$^\text{n}$ $<$ x, thus (y+h) $\in$ A and thus, not an upper bound of A.

Suppose y$^\text{n}$ $>$ x. Pick k = $\frac{y^n - x}{ny^{n-1}}$ $<$ y. <Insert Proof>

Consider t $ \geq $  y-k, then:

\qquad y$^\text{n}$ - t$^\text{n}$ $ \leq $  y$^\text{n}$ - (y-k)$^\text{n}$ $<$
	kny$^\text{n-1}$ = y$^\text{n}$ - x

So t $\not \in$ A.

Thus, y-k is not an upper bound of A contradicting y is the least upper bound of A.

Since y$^\text{n}$ $<$ x and y$^\text{n}$ $>$ x, then y$^\text{n}$ = x. 

\subsection{Decimals and Extended Reals}

\subsection{Complex Numbers}

	{ \color{blue} Definition 3.3.1 } 
	
	\qquad A complex number is an ordered pair (a,b) where a,b $\in$ $ \mathbb{R} $.
	For x,y $\in$ $\mathbb{C}$
	\begin{itemize}[leftmargin=2cm]
		\item (a,b) + (c,d) = (a + c , b + d)
		\item (a,b) * (c,d) = (ac - bd , ad + bc)
	\end{itemize}

	Thus, the axioms form a field where (0,0) = 0 and (1,0) = 1 and (0,1) = i. \\

	{ \color{blue} Definition 3.3.2 } 
	
	\qquad Let i = (0,1).

	{ \color{magenta} \underline{Proof} } 
	
	i$^\text{2}$ = -1. {\color{red} Insert Proof} \\

	{ \color{blue} Definition 3.3.3 } 
	
	\qquad (a,b) = a + bi \\

	{ \color{blue} Definition 3.3.4 }
	
	\qquad Let conjugate: $\bar{z}$ = a - bi

	\begin{itemize}[leftmargin=2cm]
		\item $\bar{(z+w)}$ = $\bar{z}$ + $\bar{w}$

			\item Product

			\item Real

			\item z$\bar{z}$ $ \geq $ 0 
	\end{itemize}

	{ \color{blue} Definition 3.3.5 } 
	
	\qquad The \|z\| = \sqrt{z \bar{z}}.

	\qquad Insert propert for real numbers conjugate and absolute value. \\

{ \color{blue} Theorem 3.36  } 

\begin{itemize}[leftmargin=2cm]
	\item If x $\not =$ 0 abd \| x \| > 0, then 
	\item conjuagte
	\item product conjugate
	\item Real part $<$ Actual for abs
	\item \| z+w \| $ \leq $  \| z \| + \| w \|
\end{itemize}


