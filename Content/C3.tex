\newpage
\section[Day 3: Roots and the Complex Field]{Roots \& Complex Field}

\subsection{nth Root}

	\begin{enumerate}[label=(\alph*), leftmargin=1.5cm, itemsep=0.1cm]
		\item If 0 $<$ t $\leq$ 1, then t$^\text{n}$ $ \leq $ t.

			{ \color{magenta} \underline{Proof} } 
		
				Since t $>$ 0 and t $\leq$ 1, then t$^\text{2}$ $\leq$ t.

				Since t$^\text2{}$ $\leq$ t, then t$^\text{3}$ $\leq$ t$^\text{2}$
				so t$^\text{3}$ $\leq$ t$^\text{2}$ $\leq$ t.

				Applying the process n times, then t$^\text{n}$ $\leq$ t.

		\item If t $\geq$ 1, t$^\text{n}$ $ \geq $ t.

			{ \color{magenta} \underline{Proof} } 
		
				Since 0 $<$ 1 $\leq$ t, then t $\leq$ t$^\text{2}$.

				Since t $\leq$ t$^\text{2}$, then t$^\text{2}$ $\leq$ t$^\text{3}$
				so t $\leq$ t$^\text{2}$ $\leq$ t$^\text{3}$.

				Applying the process n times, t $\leq$ t$^\text{n}$.

		\item If 0 $<$ s $<$ t, then s$^\text{n}$ $<$ t$^\text{n}$.

			{ \color{magenta} \underline{Proof} } 
		
				$\underbrace{\text{s} \cdot \text{s} \cdot ... \cdot \text{s}}_\text{n}$
				$<$ t $\cdot$ s $\cdot$ ... $\cdot$ s
				$<$ t $\cdot$ t $\cdot$ ... $\cdot$ s $<$ ... $<$
				$\underbrace{\text{t} \cdot ... \cdot \text{t}}_\text{n}$ \\
	\end{enumerate}


{ \color{red} Theorem 3.1.1: y$^\text{n}$ = x has a unique y }

	\qquad Fix n $\in$ $\mathbb{Z}_+$. For every x $>$ 0, there exists a unique y
	$\in$ $ \mathbb{R} $ such that y$^\text{n}$ = x.

	\qquad Also, such a y is written as y = $\sqrt[n]{\text{x}}$ = x$^{\frac{1}{\text{n}}}$.

{ \color{magenta} \underline{Proof} } 

	{ \color{lblue} Uniqueness: }

	y is unique since if y$_{1}$ $<$ y$_{2}$, then
	x = y$_{1}$$^\text{n}$ $<$ y$_{2}$$^\text{n}$ $\neq$ x.

	{ \color{lblue} Existence: }

	Let set A = \{ t $>$ 0 : t$^\text{n}$ $<$ x \}.

	A $\not =$ $\emptyset$ since let t$_{1}$ = $\frac{x}{x+1}$ $<$ 1 so t$_1$ $<$ x
	and thus, 0 $<$ t$_{1}$$^\text{n}$ $<$ t$_{1}$ $<$ x so t$_{1}$ $\in$ A.

	A is bounded above since if t $ \geq $ x+1, then
	t $>$ 1 so t$^\text{n}$ $ \geq $ t $ \geq $ x+1 $>$ x so t $\not \in$ A.

	So x+1 is an upper bound of A.

	Thus by the least upper bound property, y = sup(A) exists.

	For y$^\text{n}$ = x, show y$^\text{n}$ $<$ x and y$^\text{n}$ $>$ x cannot hold true.

	***(Not an upper bound of A if $<$ and not a least upper bound of A if $>$)***

	For 0 $<$ $\alpha$ $<$ $\beta$:

	\qquad $\beta ^\text{n}$ - $\alpha^\text{n}$
	= ($\beta$ - $\alpha$) ($\underbrace{\beta^{n-1} + \beta^{n-2}\alpha^1 + ... + \alpha^{n-1}}
	_{\beta^{n-1} \hspace{0.7cm} <\beta^{n-1} \hspace{1.3cm} <\beta^{n-1}}$)
	$<$ ($\beta - \alpha$)n$\beta$$^\text{n-1}$ 

	Suppose y$^\text{n}$ $<$ x. Pick 0 $<$ h $<$ 1 and h $<$ $\frac{x-y^n}{n(y+1)^{n-1}}$.

	\qquad From inequality, let $\beta$ = y+h and $\alpha$ = y

	\qquad \qquad (y+h)$^\text{n}$ - y$^\text{n}$ $<$ hn(y+h)$^\text{n-1}$
	$<$ hn(y+1)$^\text{n-1}$ $<$ x - y$^\text{n}$

	\begin{adjustbox}{minipage=14.2cm, right, vspace=0.1cm 0cm}
		Thus, (y+h)$^\text{n}$ $<$ x so y+h $\in$ A and thus, not an upper bound of A
		which is a contradiction since y = sup(A).
	\end{adjustbox}

	Suppose y$^\text{n}$ $>$ x.
	Pick 0 $<$ k = $\frac{y^n - x}{ny^{n-1}}$ $<$ $\frac{y^n}{ny^{n-1}}$
	= $\frac{1}{n}y$ $<$ y.

	\qquad Consider t $ \geq $  y-k, then:
	y$^\text{n}$ - t$^\text{n}$ $ \leq $  y$^\text{n}$ - (y-k)$^\text{n}$ $<$
	kny$^\text{n-1}$ = y$^\text{n}$ - x

	\qquad Thus, t$^\text{n}$ $>$ x so t $\not \in$ A.

	\qquad Thus, y-k is an upper bound of A which is a contradiction since y = sup(A).

	Since y$^\text{n}$ $<$ x and y$^\text{n}$ $>$ x, then y$^\text{n}$ = x. 

\newpage

{ \color{orange} Corollary 3.1.2: n-th root of product = product of n-th root} 

	\begin{adjustbox}{minipage=14cm, right, vspace=0.1cm 0cm}
		If a,b $>$ 0 and n $\in$ $\mathbb{Z}_+$, then
		(ab)$^{\frac{1}{n}}$ = a$^{\frac{1}{n}}$b$^{\frac{1}{n}}$.  
	\end{adjustbox}

{ \color{magenta} \underline{Proof} } 

	Let A = a$^{\frac{1}{n}}$ and B = b$^{\frac{1}{n}}$.

	Then by {\color{red} theorem 3.1.1}, since A is a solution to $y_1^n$ = a, then A$^n$ = a.

	Similarly, B is a solution of $y_2^n$ = b so B$^n$ = b. Thus:

	\hspace{1cm} ab = $A^n$$B^n$ = $A_1A_2...A_nB_1B_2...B_n$

	\hspace{1.6cm} = $A_1A_2...B_1A_nB_2...B_n$ = ... = $A_1B_1A_2...A_{n-1}A_nB_3...B_n$

	\hspace{1.6cm} = ... = $A_1B_1A_2B_2...A_nB_n$ = $(AB)^n$

	Then again by {\color{red} theorem 3.1.1}, there is a unique
	(ab)$^{\frac{1}{n}}$ = AB = a$^{\frac{1}{n}}$b$^{\frac{1}{n}}$.




	
\subsection{Decimals}

	Let n$_0$ be the largest integer such that n$_0$ $\leq$ x for
	x $>$ 0 $\in$ $\mathbb{R}$.

	Then let n$_k$ be the largest integer such that
	d$_k$ = n$_0$ + $\frac{n_1}{10}$ + ... + $\frac{n_k}{10^k}$ $\leq$ x

	Let E be the set of d$_k$ for k = 0, 1, ... $\infty$. Then, x = sup(E).





\subsection{Extended Reals}

	The extended real number system consist of $\mathbb{R}$ and $\pm$$\infty$ such that:

	\qquad -$\infty$ $<$ x $<$ $\infty$	\qquad for every x $\in$ $\mathbb{R}$

	with the properties:
	\begin{itemize}[leftmargin=1cm, itemsep=0.1cm]
		\item x $\pm$ $\infty$ = $\pm$$\infty$
	
		\item x / $\pm$$\infty$ = 0

		\item If x $>$ 0, then x($\pm$$\infty$) = $\pm$$\infty$

		\item If x $<$ 0, then x($\pm$$\infty$) = $\mp$$\infty$
	\end{itemize}





\subsection{Complex Numbers}

{ \color{blue} Definition 3.4.1: Complex } 
	
	\qquad A complex number is an ordered pair (a,b) where a,b $\in$ $ \mathbb{R} $.
	For x,y $\in$ $\mathbb{C}$
	\begin{itemize}[leftmargin=2cm, itemsep=0.1cm]
		\item x + y = (a,b) + (c,d) = (a + c , b + d)
		\item xy = (a,b) (c,d) = (ac - bd , ad + bc)
		\item $\frac{1}{\text{x}}$  = (a$^2$ + b$^2$)(a,-b)
	\end{itemize}

	\qquad Thus, the axioms form a field where (0,0) = 0 and (1,0) = 1 and (0,1) = i. \\

{ \color{blue} Definition 3.4.2: Imaginary i } 
	
	\qquad Let i = (0,1). Then, i$^2$ = -1.

{ \color{magenta} \underline{Proof} } 
	
	i$^\text{2}$ = (0,1)(0,1) = (0-1,0+0) = (-1,0) = -1 \\

{ \color{blue} Definition 3.4.3: Form a + bi } 
	
	\qquad (a,b) = a + bi

{ \color{magenta} \underline{Proof} } 
	
	(a,b) = (a,0) +(0,b) = (a,0) + (b,0)(0,1) = a + bi \\

\newpage

{ \color{blue} Definition 3.4.4: Conjugate }
	
	\qquad Let conjugate: $\bar{z}$ = a - bi where Re(z) = a , Im(z) = b \\

	\qquad Let z = (a,b) and w = (c,d):
	\begin{enumerate}[label=(\alph*), leftmargin=2cm, itemsep=0.1cm]
		\item $\overline{z+w}$ = $\overline{z}$ + $\overline{w}$

			{ \color{magenta} \underline{Proof} } 
			
				$\overline{z+w}$ = $\overline{(a+c,b+d)}$ = (a+c,-b-d)
				= (a,-b) + (c,-d) = $\overline{z}$ + $\overline{w}$

		\item $\overline{zw}$ = $\overline{z}$ $\overline{w}$

			{ \color{magenta} \underline{Proof} } 
			
				$\overline{zw}$ = $\overline{(ac-bd,ad+bc)}$ = (ac-bd,-ad-bc)
				= (a,-b) (c,-d) = $\overline{z}$ $\overline{w}$

		\item z + $\overline{z}$ = 2 Re(z) \qquad \qquad z - $\overline{z}$ = 2i Im(z)

			{ \color{magenta} \underline{Proof} } 
			
				z + $\overline{z}$ = (a,b) + (a,-b) = (2a,0) = 2 Re(z)

				z - $\overline{z}$ = (a,b) - (a,-b) = (0,2b) = (0,2) b = 2i Im(z)

		\item z$\overline{z}$ $\geq$ 0

			{ \color{magenta} \underline{Proof} } 
			
				z$\overline{z}$ = (a,b)(a,-b) = (a$^2$ + b$^2$ , -ab+ab)
				= a$^2$ + b$^2$ $\geq$ 0 \\
	\end{enumerate}

{ \color{blue} Definition 3.4.5: Absolute Value } 
	
	\qquad Let absolute value: $|$ z $|$ = $\sqrt{z \overline{z}}$ \\

	\qquad Let z = (a,b) and w = (c,d):
	\begin{enumerate}[label=(\alph*), leftmargin=2cm, itemsep=0.1cm]
		\item If z $\neq$ 0, then $|$ z $|$ $>$ 0.

			{ \color{magenta} \underline{Proof} } 
			
				$\sqrt{z\overline{z}}$ = $\sqrt{a^2 + b^2}$ $\geq$ 0
				where $|$ z $|$ = 0 only if a,b = 0 so only if z = (0,0).

		\item $|$ $\overline{z}$ $|$ = $|$ z  $|$

			{ \color{magenta} \underline{Proof} } 
			
				$|$ $\overline{z}$ $|$ = $\sqrt{a^2 + (-b)^2}$ = $\sqrt{a^2 + b^2}$ = $|$ z $|$

		\item $|$ zw $|$ = $|$ z $|$ $|$ w $|$

			{ \color{magenta} \underline{Proof} } 
			
				$|$ zw $|$ = $|$ (ac-bd,ad+bc) $|$ = $\sqrt{(ac-bd)^2 + (ad+bc)^2}$
			
				= $\sqrt{a^2c^2 + b^2d^2 + a^2d^2 + b^2c^2}$
				= $\sqrt{(a^2+b^2)(c^2+d^2)}$

				= $\sqrt{a^2+b^2}$ $\sqrt{c^2+d^2}$ = $|$ z $|$ $|$ w $|$

		\item $|$ Re(z) $|$ $\leq$ $|$ z $|$

			{ \color{magenta} \underline{Proof} } 
			
				$|$ Re(z) $|$ = $|$ a $|$ = $\sqrt{a^2}$ $\leq$ $\sqrt{a^2+b^2}$ = $|$ z $|$

		\item $|$ z+w $|$ $ \leq $  $|$ z $|$ + $|$ w $|$

			{ \color{magenta} \underline{Proof} } 
			
				$| z+w |^2$ = (z+w)$\overline{(z+w)}$ = (z+w)($\overline{z} + \overline{w}$)
				= z$\overline{z}$ + z$\overline{w}$ + w$\overline{z}$ + w$\overline{w}$
			
				= $|z|^2$ + $|w|^2$ + 2 Re(z$\overline{w}$)
				$\leq$ $|z|^2$ + $|w|^2$ + 2 $|z\overline{w}|$

				= $|z|^2$ + $|w|^2$ + 2$|z||w|$
				= ($|z|$ + $|w|$)$^2$
	\end{enumerate}


