\newpage

\section[Day 4: Cardinality]{ Cardinality }

\subsection{ Cardinality }

	\begin{definition}{Onto and 1-1 Mapping}{16cm}
		Suppose for every x $\in$ A, there is an associated f(x) $\in$ B.

		Then f maps A into B = f: A $\rightarrow$ B.

		\begin{itemize}[leftmargin=1cm, itemsep=0.1cm]
			\item If f(A) = B, then f maps A onto B.
			
			\item If for each y $\in$ B, f$^{-1}$(y) consist of at most one x $\in$ A
				where f$^{-1}$(y$_1$) = x$_1$ $\neq$ x$_2$ = f$^{-1}$(y$_2$) for
				y$_1$ $\neq$ y$_2$, then f is a 1-1 mapping of A into B.
		\end{itemize}
	\end{definition}



	\begin{figure}[h]
		\centering
		\includegraphics[scale=0.27]{Images/4.1.1.png}
	\end{figure}



	\begin{definition}{1-1 Correspondence}{16cm}
		Sets A and B are equivalent {\color{lblue} (have the same cardinality)}
		if there is a 1-1 onto function f: A $\rightarrow$ B.
		{\color{lblue} (1-1 correspondence between A and B)}
		Then, A $\sim$ B.

		If f: A $\rightarrow$ B is 1-1 and onto, then
		there is a f$^{-1}$: B $\rightarrow$ A that is 1-1 and onto.
	\end{definition}

	\vspace{0.5cm}



	\begin{definition}{Countability}{16cm}
		\begin{itemize}[leftmargin=1cm, itemsep=0.1cm]
			\item A is {\color{lblue} finite}
				if A $\sim$ J$_n$ = \{0, 1, ..., n\} for some n $\in$ $\mathbb{N}$

			\item A is {\color{lblue} infinite}
				if A is not finite
			
			\item A is {\color{lblue} countably infinite}
				if A $\sim$ J = $\mathbb{Z}_+$
			
			\item A is {\color{lblue} uncountable}
				if A is not finite or countably infinite
			
			\item A is {\color{lblue} at most countable}
				if A is finite or countably infinite
		\end{itemize}	
	\end{definition}

	\vspace{0.5cm}



	\begin{example}
		$\mathbb{Z}$ is countably infinite
	\end{example}
	
	\begin{proof}
		Let f(n): $\mathbb{Z}_+$ $\rightarrow$ $\mathbb{Z}$ = 
		$\begin{cases}
			\frac{n}{2} & \text{n is even} \\
			-\frac{n-1}{2} & \text{n is odd} \\
		\end{cases}$

		So 1 $\mapsto$ 0 , 2 $\mapsto$ 1, 3 $\mapsto$ -1, 4 $\mapsto$ 2,
		5 $\mapsto$ -2 , etc.
		Thus, $\mathbb{Z}$ $\sim$ $\mathbb{Z}_+$.
	\end{proof}

	\vspace{0.5cm}



	\begin{definition}{Pigeonhole Principle}{16cm}
		If A is finite, A is not equivalent to any proper set of A.
	\end{definition}
	


	\begin{figure}[h]
		\centering
		\includegraphics[scale=0.4]{Images/4.1.4.png}
	\end{figure}

	\newpage



	\begin{wtheorem}{Infinite subsets of countable sets are countable}{16cm}
		An infinite subset E of a countably infinite set A is countably infinite
	\end{wtheorem}
	
	\begin{proof}
		Let E $\subset$ A be an infinite subset.
		For every distinct $x_i$ $\in$ A, let \{ $x_1, x_2, ...$ \} $\in$ A.

		Let $n_1$ be smallest integer such that x$_{n_1}$ $\in$ E.

		Then let $n_2$ be the smallest integer where $n_2$ $>$ $n_1$
		such that x$_{n_2}$ $\in$ E.

		Repeat the process to create sequence
		f(k) = \{ $x_{n_1}, x_{n_2}, ... , x_{n_k} , ...$ \}.

		Thus, there is a 1-1 correspondence between E and $\mathbb{Z}_+$ so
		E is countably infinite.
	\end{proof}



	\begin{figure}[h]
		\centering
		\includegraphics[scale=0.5]{Images/4.1.5.png}
	\end{figure}





\subsection{ Set of Sets } 

	\begin{definition}{Union and Intersection}{16cm}
		Let sets $\Omega$,B be such that for each x $\in$ $\Omega$,
		there is an associated E$_x$ $\subset$ B.
		\begin{itemize}[leftmargin=1cm, itemsep=0.1cm]
			\item E = $\cup_{x=1}^{n}$ E$_x$ only if for every x $\in$ E,
			x $\in$ E$_x$ for at least one x $\in$ $\Omega$.

			\item P = $\cap_{x=1}^{n}$ E$_x$ only if for every x $\in$ P,
			x $\in$ E$_x$ for all x $\in$ $\Omega$.
		\end{itemize}
		with properties:
	\end{definition}

	\begin{enumerate}[label=(\alph*), leftmargin=2cm, itemsep=0.1cm]
		\item A $\cup_{}^{}$ B = B $\cup_{}^{}$ A
			\hspace{4cm} A $\cap_{}^{}$ B = B $\cap_{}^{}$ A

		\item (A $\cup_{}^{}$ B) $\cup_{}^{}$ C = A $\cup_{}^{}$ (B $\cup_{}^{}$ C)
			\hspace{1.6cm} (A $\cap_{}^{}$ B) $\cap_{}^{}$ C
			= A $\cap_{}^{}$ (B $\cap_{}^{}$ C)

		\item A $\subset$ A $\cup_{}^{}$ B
			\hspace{4.8cm} (A $\cap_{}^{}$ B) $\subset$ A

		\item If A $\subset$ B, then A $\cup_{}^{}$ B = B and A $\cap_{}^{}$ B = A

			\begin{proof}[15cm]
				If x $\in$ A $\cup_{}^{}$ B, then x $\in$ A or/and x $\in$ B.
				\begin{itemize}[leftmargin=1cm, itemsep=0.2cm]
					\item If x $\in$ A, since A $\subset$ B, then x $\in$ B.
						Then, (A $\cup_{}^{}$ B) $\subset$ B.

					\item If x $\in$ B, then immediately (A $\cup_{}^{}$ B) $\subset$ B.
				\end{itemize}
				If x $\in$ B, then x $\in$ A $\cup_{}^{}$ B so B $\subset$ (A $\cup_{}^{}$ B).
				Thus, A $\cup_{}^{}$ B = B.

				\vspace{0.5cm}

				If x $\in$ A $\cap_{}^{}$ B, then x $\in$ A and x $\in$ B.
				Thus, (A $\cap_{}^{}$ B) $\subset$ A.

				If x $\in$ A, since A $\subset$ B, then x $\in$ B so x $\in$ A $\cap_{}^{}$ B.
				Thus, A $\subset$ (A $\cap_{}^{}$ B).

				Thus, A $\cap_{}^{}$ B = A.
			\end{proof}

		\item A $\cap_{}^{}$ (B $\cup_{}^{}$ C)
			= (A $\cap_{}^{}$ B) $\cup_{}^{}$ (A $\cap_{}^{}$ C)

			\begin{proof}[15cm]
				If x $\in$ A $\cap_{}^{}$ (B $\cup_{}^{}$ C), then x $\in$ A
				and (x $\in$ B or/and x $\in$ C).
				\begin{itemize}[leftmargin=1cm, itemsep=0.1cm]
					\item If x $\in$ B, then x $\in$ (A $\cap_{}^{}$ B) so
						x $\in$ (A $\cap_{}^{}$ B) $\cup_{}^{}$ (A $\cap_{}^{}$ C).

					\item If x $\in$ C, then x $\in$ (A $\cap_{}^{}$ C) so
						x $\in$ (A $\cap_{}^{}$ B) $\cup_{}^{}$ (A $\cap_{}^{}$ C).
				\end{itemize}
				Thus, A $\cap_{}^{}$ (B $\cup_{}^{}$ C)
				$\subset$ (A $\cap_{}^{}$ B) $\cup_{}^{}$ (A $\cap_{}^{}$ C).
				
				If x $\in$ (A $\cap_{}^{}$ B) $\cup_{}^{}$ (A $\cap_{}^{}$ C),
				then x $\in$ A and (x $\in$ B or/and x $\in$ C).

				Thus, (A $\cap_{}^{}$ B) $\cup_{}^{}$ (A $\cap_{}^{}$ C)
				$\subset$ A $\cap_{}^{}$ (B $\cup_{}^{}$ C).

				Thus, A $\cap_{}^{}$ (B $\cup_{}^{}$ C)
				= (A $\cap_{}^{}$ B) $\cup_{}^{}$ (A $\cap_{}^{}$ C).
			\end{proof}

			\newpage

		\item A $\cup_{}^{}$ (B $\cap_{}^{}$ C)
			= (A $\cup_{}^{}$ B) $\cap_{}^{}$ (A $\cup_{}^{}$ C)

			\begin{proof}[15cm]
				If x $\in$ A $\cup_{}^{}$ (B $\cap_{}^{}$ C), then
				x $\in$ A or/and (x $\in$ B and x $\in$ C).
				\begin{itemize}[leftmargin=1cm, itemsep=0.1cm]
					\item If x $\in$ A, then x $\in$ (A $\cup_{}^{}$ B)
						and x $\in$ (A $\cup_{}^{}$ C)
						so A $\cup_{}^{}$ (B $\cap_{}^{}$ C) $\subset$
						(A $\cup_{}^{}$ B) $\cap_{}^{}$ (A $\cup_{}^{}$ C).

					\item If x $\in$ B,C, then x $\in$ (A $\cup_{}^{}$ B)
						and x $\in$ (A $\cup_{}^{}$ C)
						so A $\cup_{}^{}$ (B $\cap_{}^{}$ C) $\subset$
						(A $\cup_{}^{}$ B) $\cap_{}^{}$ (A $\cup_{}^{}$ C).
				\end{itemize}
				If x $\in$ (A $\cup_{}^{}$ B) $\cap_{}^{}$ (A $\cup_{}^{}$ C), then
				x $\in$ A or/and (x $\in$ B and x $\in$ C).

				Thus, (A $\cup_{}^{}$ B) $\cap_{}^{}$ (A $\cup_{}^{}$ C)
				$\subset$ A $\cup_{}^{}$ (B $\cap_{}^{}$ C).

				Thus, A $\cup_{}^{}$ (B $\cap_{}^{}$ C)
				= (A $\cup_{}^{}$ B) $\cap_{}^{}$ (A $\cup_{}^{}$ C).
			\end{proof}
	\end{enumerate}

	\vspace{0.5cm}



	\begin{wtheorem}{Union of countably infinite sets is countably infinite}{16cm}
		If $E_1, E_2, ... $ are countably infinite sets, then S = $\cup_{n=1}^{\infty}$ $E_n$
		is countably infinite.		
	\end{wtheorem}
	
	\begin{proof}
		For each $E_n$, there is a sequence \{ $x_{n1}$, $x_{n2}$, ... \}.
		Then construct an array as such:

		{\small $ \hspace{1cm}
		\left(
		\begin{array}{ccc}
			x_{11} & x_{12} & ... \\
			x_{21} & x_{22} & ... \\
			\vdots & \vdots & \ddots \\
		\end{array}
		\right)
		$}
			
		Take elements diagonally, then sequence S$^*$ =
		\{ $x_{11} \ ; \ x_{21}, x_{12} \ ; \ x_{31}, x_{32}, x_{33} \ ; \ ... $ \}.
			
		Since S$^*$ $\sim$ S so S is at most countable and S is infinite since
		$E_1, E_2, ...$ are infinite, then S cannot be finite and
		thus, countably infinite.
	\end{proof}

	\vspace{0.1cm}

	{ \color{magenta} \underline{Alternative Proof} }

	\begin{tbox}
		For each $E_n$, let set $\widetilde{E_n}$ = $E_n$ - $\cup_{m=1}^{\infty}$
		$E_m$ where m $\neq$ n. Thus, S = $\cup_{n=1}^{\infty}$ $\widetilde{E_n}$.

		Since each $E_n$ is countably infinite, there exists a 1-1 mapping
		$\delta_n$: $E_n$ $\rightarrow$ $\mathbb{Z}_+$.

		Thus, for each $\widetilde{E_n}$, there is a 1-1 mapping
		$\delta_n$: $\widetilde{E_n}$ $\rightarrow$ A $\subset$ $\mathbb{Z}_+$.

		Let $p_1, p_2, ... $ be distinct primes.
		Since for s $\in$ S, there exists a unique $\widetilde{E_i}$ such that
		s $\in$ $\widetilde{E_i}$, then let f(s) = $p_1^{\delta_1(s)}
		p_2^{\delta_2(s)}...$ where $p_k^{\delta_k(s)}$ = 1 if k $\neq$ i.

		Then, by the Fundamental theorem of arithmetic, f maps s to a unique
		z $\in$ $\mathbb{Z}_+$ and thus, f is a 1-1 function so S is at most countable.
		Since any $E_n$ $\subset$ S is countably infinite, then S cannot be finite
		and thus, S is countably infinite.
	\end{tbox}



	\begin{figure}[h]
		\centering
		\includegraphics[scale=0.32]{Images/4.2.2.png}
	\end{figure}



	\begin{wtheorem}{The set of countable n-tuples are countable}{16cm}
		Let set A be countably infinite and B$_n$ be the set of all
		n-tuples ($a_1$,...,$a_n$) where $a_k$ $\in$ A.
		Then $B_n$ is countably infinite.
	\end{wtheorem}
	
	\begin{proof}
		The base case $B_1$ is countably infinite since $B_1$ = A.

		Suppose $B_{n-1}$ is countably infinite. Then for every x $\in$ B:

		\qquad x = (b,a) \qquad \qquad b $\in$ $B_{n-1}$ and a $\in$ A

		Since for every fixed b, (b,a) $\sim$ A and thus, countably infinite.

		Since B is a set of countably infinite sets, then $B_{n}$
		is countably infinite.
	\end{proof}
	
	\newpage
	
	
	
	\begin{wtheorem}{$\mathbb{Q}$ is countable}{16cm}
		The set of rational numbers, $\mathbb{Q}$, is countably infinite 
	\end{wtheorem}
	
	\begin{proof}
		Since elements of $\mathbb{Q}$ are of form $\frac{a}{b}$ which is a
		2-tuple, then by the {\color{red} theorem 4.2.3},
		$\mathbb{Q}$ is countably infinite. 
	\end{proof}

	\vspace{0.1cm}

	{ \color{magenta} \underline{Alternative Proof} }

	\begin{tbox}
		For every x $\in$ $\mathbb{Q}$, let x = $(-1)^i$ $\frac{p}{q}$ where
		p,q $\in$ $\mathbb{Z}_+$.

		Let f(x) = $2^i$ $3^p$ $5^q$. Then by the Fundamental theorem of arithmetic,
		f is a 1-1 mapping of x to E $\subset$ $\mathbb{Z}_+$.

		Thus, $\mathbb{Q}$ is at most countable, but since p,q $\in$ $\mathbb{Z}_+$,
		then $\mathbb{Q}$ cannot be finite and thus, is countably infinite. 
	\end{tbox}
	
	\vspace{0.5cm}


	 
	\begin{example}
		Let A be the set of all sequences whose elements are digits 0 and 1.
		Then A is uncountable. 
	\end{example}

	\vspace{0.1cm}
	 
	{ \color{magenta} \underline{Proof: Cantor's Diagonalization Proof} }
	 
	\begin{tbox}
		Let set E be a countably infinite subset of A which consist of
		sequences $s_1,s_2,...$.

		Then construct a sequence s as follows:

		\qquad If the n-th digit in $s_n$ is 1, then let the n-th digit of s be 0
		and vice versa.

		Thus. s differs from every $s_n$ $\in$ E so s $\not \in$ E.

		But, s $\in$ A so E is a proper subset of A.

		Thus, every countably infinite subset of A is a proper subset of A.

		If A is countably infinite, then A is a proper subset of A which
		is a contradiction.
	\end{tbox}
	



	