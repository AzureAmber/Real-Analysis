\newpage

\section[Day 7: Perfect \& Connected Sets]{ Perfect and Connected Sets }

\subsection{ Perfect Sets }

	\begin{definition}{Perfect Set}{14cm}
		E $\subset$ X is perfect if E is closed and if every p $\in$ E
		is p $\in$ E'
	\end{definition}

	\vspace{0.5cm}



	\begin{wtheorem}{Perfect sets are uncountable}{14cm}
		Let P be a nonempty perfect set in $\mathbb{R}^k$.
		Then, P is uncountable.
	\end{wtheorem}

	\begin{proof}
		Since P has limit points, then by {\color{red} theorem 5.1.4},
		P is infinite.

		Suppose P is countable. Then let $x_1, x_2, ...$ $\in$ P.

		Let $V_i$ be a neighborhood of $x_i$ where y $\in$ $V_i$ for any
		y $\in$ $\mathbb{R}^k$ such that $|y-x_i|$ $<$ r.

		Thus, the $\overline{V_i}$ is the set of all y $\in$ $\mathbb{R}^k$
		such that $|y-x_i|$ $\leq$ r.

		Since every $x_i$ are limit points, then any $V_i$ $\cap_{}^{}$ P
		is not empty where there is a $V_{i+1}$

		\begin{enumerate}[label=(\alph*), leftmargin=1cm, itemsep=0.1cm]
			\item $\overline{V_{i+1}}$ $\subset$ $V_i$
			\item $x_i$ $\not \in$ $\overline{V_{i+1}}$
			\item $V_{i+1}$ $\cap_{}^{}$ P is nonempty
		\end{enumerate}

		Let $K_i$ = $\overline{V_i}$ $\cap_{}^{}$ P.
		Since $\overline{V_i}$ is closed and bounded, then by
		{\color{red} theorem 6.3.11}, $\overline{V_i}$ is compact.
		Since $x_i$ $\not \in$ $K_{i+1}$, then no $x_i$ $\in$ P is
		$x_i$ $\in$ $\cap_{}^{}$ $K_i$.
		Since $K_n$ $\subset$ P, then $\cap_{}^{}$ $K_i$ is empty
		which contradicts {\color{orange} corollary 6.3.8 } since
		each $K_i$ is nonempty and $K_{i+1}$ $\subset$ $K_i$.
	\end{proof}



	\begin{figure}[h]
		\centering
		\includegraphics[scale=0.3]{Images/7.1.2.png}
	\end{figure}



	\begin{corollary}{$\mathbb{R}$ is not countable}{14cm}
		Every interval [a,b] is uncountable.
		Thus, $\mathbb{R}$ is uncountable.
	\end{corollary}
	
	\begin{proof}
		Since [a,b] is closed and every p $\in$ [a,b] is a limit point,
		then nonempty set [a,b] is perfect.
		Thus, by {\color{red} theorem 7.1.2}, [a,b] is uncountable.
	\end{proof}

	\newpage



	\begin{definition}{Cantor Set}{14cm}
		There exists perfect segments in $\mathbb{R}^1$ which contain no segment.

		Let $E_0$ = [0,1].

		For $E_1$, remove ($\frac{1}{3}$,$\frac{2}{3}$).
		Thus, $E_1$ = [0,$\frac{1}{3}$] $\cup_{}^{}$ [$\frac{2}{3}$,1].

		For $E_2$, remove ($\frac{1}{9}$,$\frac{2}{9}$) and ($\frac{7}{9}$,$\frac{8}{9}$).
		Thus, $E_2$ = [0,$\frac{1}{9}$] $\cup_{}^{}$ [$\frac{2}{9}$,$\frac{3}{9}$] $\cup_{}^{}$
		[$\frac{6}{9}$,$\frac{7}{9}$] $\cup_{}^{}$ [$\frac{8}{9}$,1].

		Continuing such a sequence, the set of compact sets $E_n$ are such that:

		\begin{enumerate}[label=(\alph*), leftmargin=2cm, itemsep=0.1cm]
			\item $E_{n+1}$ $\subset$ $E_n$
			\item $E_n$ is the union of $2^n$ intervals each of length $3^{-n}$.
		\end{enumerate}

		P = $\cap_{}^{}$ $E_n$ is called the Cantor set.
		P is compact and nonempty.

		Thus, any segment of form $(\frac{3k+1}{3^m},\frac{3k+2}{3^m})$
		where k,m $\in$ $\mathbb{Z}_+$ has no points in common with P.
		Since any segment (a,b) contain a segment of such a form since
		$3^{-m}$ $<$ $\frac{b-a}{6}$, then P contains no segment.

		Let x $\in$ P and segment S contain x. Let $I_n$ be an interval of
		$E_n$ containing x. Then choose a large enough n so $I_n$ $\subset$ S.
		
		Let $x_n$ be an endpoint of $I_n$ where $x_n$ $\not =$ x and thus,
		x is a limit point. Since P is closed and every p $\in$ P is p $\in$ P',
		then P is perfect.
	\end{definition}
	
	\vspace{0.5cm}





\subsection{ Connected Sets }

	\begin{definition}{Connected Set}{14cm}
		A,B $\subset$ X are separated if both
		A $\cap_{}^{}$ $\overline{B}$ and
		$\overline{A}$ $\cap_{}^{}$ B are empty.

		E $\subset$ X is connected if E is not the union of two nonempty
		separated sets.

		Separated sets are disjoint, but disjoint sets need not be separated.
	\end{definition}

	\vspace{0.5cm}



	\begin{wtheorem}{All points between points in connected sets exists}{14cm}
		E $\subset$ $\mathbb{R}^1$ is connected if and only if:

		\hspace{0.5cm}
		If x,y $\in$ E and x $<$ z $<$ y, then z $\in$ E.
	\end{wtheorem}
	
	\begin{proof}
		If there exists x,y $\in$ E and z $\in$ (x,y) such that z $\not \in$ E,
		then E = $A_z$ $\cup_{}^{}$ $B_z$ where
		
		$A_z$ = E $\cap_{}^{}$ ($-\infty,z$)
		and $B_z$ = E $\cap_{}^{}$ ($z,\infty$).

		Since x $\in$ $A_z$ and y $\in$ $B_z$, then A and B are nonempty.
		Since $A_z$ $\subset$ ($-\infty,z$) and
		
		$B_z$ $\subset$ ($z_,\infty$),
		then $A_z$ and $B_z$ are separated. Thus, E is not connected.

		\vspace{0.2cm}

		Suppose E is not connected. Then, there are nonempty separated sets
		A and B such that A $\cup_{}^{}$ B = E. Pick x $\in$ A, y $\in$ B
		where x $<$ y. Let z = sup(A $\cap_{}^{}$ [x,y]).

		Since, z $\in$ $\overline{A}$ so z $\not \in$ B, then x $\leq$ z $<$ y.
		If z $\not \in$ A, then x $<$ z $<$ y so z $\not \in$ E.

		If z $\in$ A, then z $\not \in$ $\overline{B}$ and thus,
		there exists a $z_1$ such that z $<$ $z_1$ $<$ y and $z_1$ $\not \in$ B.
		Then, x $<$ $z_1$ $<$ y so $z_1$ $\not \in$ E.
	\end{proof}




