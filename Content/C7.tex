\newpage

\section[Day 7: Balls and Convex]{ Balls and Convex }

\subsection{ Intervals and Balls } 

{ \color{blue} Definition 7.1.1: Segments and Intervals } 

	\begin{adjustbox}{minipage=14cm, right, vspace=0.1cm 0cm}
		In $\mathbb{R}$, a segement is an open interval (a,b) = \{ x $\in$ $\mathbb{R}$ : a $<$ x $<$ b \}

		In $\mathbb{R}$, a interval is a closed interval [a,b] = \{ x $\in$ $\mathbb{R}$ : a $\leq$ x $<$ b \} \\
	\end{adjustbox}

{ \color{blue} Definition 7.1.2: Open Balls } 

	\begin{adjustbox}{minipage=14cm, right, vspace=0.1cm 0cm}
		In $\mathbb{R}^k$, an open ball of radius r $>$ 0 centered at p is:

		\qquad N$_r(p)$ = \{ x $\in$ $\mathbb{R}^k$ : $|x-p|$ $<$ r \}
		= \{ x $\in$ $\mathbb{R}^k$ : d(x,p) $<$ r \}

		A closed ball has d(y,p) $\leq$ r. \\
	\end{adjustbox}

{ \color{blue} Definition 7.1.3: Convex } 

	\begin{adjustbox}{minipage=14cm, right, vspace=0.1cm 0cm}
		E $\subset$ $\mathbb{R}^k$ is convex if for all
		x,y $\in$ E and t $\in$ [0,1]:

		\qquad tx + (1-t)y $\in$ E \\
	\end{adjustbox}

{ \color{purple} Example 7.1.4: Balls are convex }

	\begin{adjustbox}{minipage=14cm, right, vspace=0.1cm 0cm}
		Balls in $\mathbb{R}^k$ are convex.
	\end{adjustbox}

{ \color{magenta} \underline{Proof} }

	Let x,y $\in$ open ball N$_r(p)$. Let z = tx + (1-t)y for t $\in$ [0,1].
	
	Since $|x - p| < r$ and $|y - p| < r$:

	\hspace{1cm} $|z - p|$ = $|tx + (1-t)y - p|$ = $|tx + (1-t)y - tp + (t-1)p|$

	\hspace{2.3cm} = $|t(x-p) + (1-t)(y-p)|$ $\leq$ $t|(x-p)|$ + $(1-t)|(y-p)|$

	\hspace{2.3cm} $<$ tr + (1-t)r = r

	Thus, z $\in$ N$_r(p)$ so balls are convex. Same proof applies to closed balls. \\

{ \color{blue} Definition 7.1.5: Dense } 

	\begin{adjustbox}{minipage=14cm, right, vspace=0.1cm 0cm}
		E $\subset$ X is dense if every x $\in$ X is either in E or
		a limit point of E. \\
	\end{adjustbox}

{ \color{purple} Example 7.1.6: $\mathbb{Q}$ is dense in $\mathbb{R}$ } 

	\qquad Let X = $\mathbb{R}$.
	Then, E = $\mathbb{Q}$ is dense in $\mathbb{R}$.

{ \color{magenta} \underline{Proof} } 

	Fix x $\in$ $\mathbb{R}$ and r $>$ 0.

	There is a q $\in$ $\mathbb{Q}$ such that x-r $<$ q $<$ x.

	So for any r $>$ 0 and q $\in$ $\mathbb{Q}$, q $\neq$ x and
	q $\in$ N$_r(x)$.

	Thus, every x $\in$ $\mathbb{R}$ is a limit point of $\mathbb{Q}$. \\





































